\chapter{CONSIDERAÇÕES FINAIS}\label{Sec:Conclusoes}

Neste trabalho, foi apresentado um estudo detalhado sobre a viabilidade da implementação do algoritmo SHA-256 em FPGA, utilizando técnicas de High-Level Synthesis (HLS) para mineração de Bitcoin. Foram explorados os aspectos teóricos relacionados à criptografia, às tecnologias de circuito e processadores, a aceleração em hardware, assim como a arquitetura de hardware e software proposta para o sistema.

O projeto desenvolvido até o momento fornece uma base sólida para o avanço para a próxima etapa, a ser realizada no TCC 3. No próximo semestre, será iniciada a implementação prática do sistema proposto em um FPGA da linha \textit{Digilent ZedBoard Zynq-7000 Development Board}, cedido pela UNIVALI. A programação será realizada utilizando a ferramenta \textit{Vivado Design Suite}, enquanto os testes de desempenho incluirão medições de \textit{hashrate} com base em logs, consumo energético mediante um Wattímetro Digital, consequentemente, o cálculo da eficiência energética e a preocupação térmica do dispositivo em funcionamento.

A fase de implementação também permitirá validar os conceitos apresentados neste trabalho, com a análise comparativa do desempenho do FPGA em relação a outras tecnologias de mineração, como CPUs e equipamentos baseados em ASICs. Adicionalmente, serão realizadas melhorias na arquitetura do sistema com base nos resultados obtidos, buscando otimizar o desempenho e a eficiência energética.