%\chapter{Trabalhos Relacionados}\label{Sec:Trabalhos_Relacionados}

\section{Trabalhos Relacionados}\label{Sec:Trabalhos_Relacionados}

Nesta seção, apresenta-se uma análise dos estudos relacionados, começando pela exposição dos repositórios utilizados, seguida da descrição dos métodos e dos critérios empregados para a seleção desses estudos. Ao final, será realizada uma comparação entre as pesquisas encontradas e o estudo atual.

\subsection{Procedimento de Análise}
Esta subseção descreve os repositórios consultados para a identificação e avaliação dos estudos relevantes. São apresentados, em detalhes, a metodologia aplicada e os critérios definidos para a seleção dos trabalhos analisados.

\subsubsection{Fontes de Pesquisa}

A pesquisa por estudos relacionados foi conduzida através da plataforma CAPES, com acesso fornecido pela UNIVALI, abrangendo diversas bases de dados, incluindo IEEE, arXiv, Science Direct e ACM. Além disso, foram realizadas consultas no Google Scholar para ampliar o alcance da pesquisa e incluir referências acadêmicas adicionais.

%Qual o propósito do trabalho, que tecnologias foram usadas, como foi implementado, resultados obtidos.

\subsubsection{Critérios de seleção}
A seleção dos trabalhos foi baseada em várias características-chave, que incluem:
(\textit{i}) o propósito do trabalho;
(\textit{ii}) quais tecnologias foram usadas;
(\textit{iii}) como foi implementado e
(\textit{iv}) os resultados obtidos. 

O presente trabalho tem como foco principal a implementação do algoritmo SHA-256 em um FPGA, utilizando técnicas avançadas de Síntese de Alto Nível (HLS) para potencializar a eficiência computacional na geração de \textit{hashrate}, essencial para a mineração de Bitcoin. 

Além disso, o trabalho visa avaliar os impactos da implementação do SHA-256 em termos de consumo de energia, desempenho e viabilidade, comparando os resultados obtidos com outras soluções de hardware dedicadas à mineração de criptomoedas.


%%%%%%%%%%%%%%%%%%%%%%%%%%%%%%%%%%%%%%%%
%%%%%%%%%% PRIMEIRO TRABALHO %%%%%%%%%%%
%%%%%%%%%%%%%%%%%%%%%%%%%%%%%%%%%%%%%%%%

\subsection{Acceleration of Frequent Itemset Mining on FPGA Using SDAccel and Vivado HLS}

O estudo realizado por \citeonline{dang2017} aborda a aceleração de algoritmos de conjuntos frequentes (Frequent Itemset Mining — FIM) utilizando FPGAs e técnicas de Síntese de Alto Nível (HLS), especificamente com as ferramentas SDAccel e Vivado HLS. A mineração de conjuntos frequentes é uma técnica importante na descoberta de padrões em grandes bases de dados, porém, é extremamente exigente em termos de processamento, principalmente com o aumento do tamanho dos conjuntos de dados. Neste contexto, o uso de FPGAs se mostra promissor por sua capacidade de acelerar algoritmos computacionalmente intensivos.

Para o desenvolvimento, o algoritmo de expansão de fronteira (Frontier Expansion — FE) foi implementado em um FPGA, utilizando SDAccel e a linguagem C/C++. Os resultados obtidos indicaram que a implementação proposta oferece uma aceleração de até 3,2 vezes em comparação com uma CPU de 6 núcleos, além de apresentar uma eficiência energética superior à de uma GPU. Em testes preliminares com o FPGA XCKU115, a implementação demonstrou desempenho comparável com implementações em HDL e superou o desempenho da GPU, destacando o potencial dos FPGAs como alternativa viável em cenários de mineração de dados.

A \autoref{fig:fluxo_dang}, os ganhos de velocidade são apresentados à medida que várias otimizações são aplicadas à implementação no FPGA. A base de comparação é a implementação \textit{Non-Opt}, que representa o FPGA sem nenhuma otimização aplicada.

Vale destacar que as otimizações (2) a (4) têm como foco principal a melhoria da taxa de transferência, sendo agrupadas com a otimização (1) sob o rótulo Opt1\_to\_4. Esse agrupamento permite uma comparação clara entre configurações parcialmente otimizadas e a configuração totalmente otimizada.

O rótulo All-Opt na figura representa o desempenho alcançado quando todas as otimizações são implementadas simultaneamente, mostrando o potencial máximo de aceleração do FPGA com uma otimização completa.


\begin{figure}[H]
    \centering
    \caption{Técnicas de otimização e desempenho em um FPGA.}
    \includegraphics[width=0.75\linewidth]{trabalhos//img/fluxo_dang.png}
    \caption*{Fonte: \cite{dang2017}.}
    \label{fig:fluxo_dang}
\end{figure}

A \autoref{fig:energy_dang} apresenta o consumo de potência dinâmica de diferentes plataformas de processamento, incluindo CPU, GPU e FPGA, durante a execução do algoritmo FE em dois conjuntos de dados distintos (Accidents e T80I20N0\_3D2000K) com um valor de suporte mínimo de 0,1. Devido à falta de suporte da ferramenta SDAccel para medições de potência diretamente na placa e ao fato de GPU e FPGA estarem em sistemas diferentes, a análise de potência foi realizada medindo o consumo de potência dinâmica. Esse processo envolveu a utilização de um medidor de energia \textit{Watts PRO} para registrar o consumo do sistema em intervalos de um segundo. A potência dinâmica foi calculada subtraindo o consumo de potência do sistema em estado ocioso do consumo durante a execução do algoritmo.

\begin{figure}[H]
    \centering
    \caption{Eficiência energética de GPU e FPGA, normalizada em relação aos resultados da CPU}
    \includegraphics[width=0.75\linewidth]{trabalhos//img/energy_dang.png}
    \caption*{Fonte: \cite{dang2017}. }
    \label{fig:energy_dang}
\end{figure}

Os resultados mostram que as implementações em FPGA com diferentes configurações de unidades computacionais (CU), como 1CU, 2CU e 4CU, apresentam um consumo de energia significativamente menor em comparação com a CPU e a GPU, especialmente quando várias CUs são usadas. No conjunto de dados \textit{Accidents}, a configuração de 6 núcleos de CPU (6CPU) registrou um consumo de 74,1W em comparação com 48,9W na configuração de 4 unidades computacionais (4CU) da FPGA. Já para o conjunto de dados T80I20N0\_3D2000K, a configuração 6CPU apresentou um consumo de 152,3W, enquanto a FPGA com 4CU consumiu apenas 50,9W. 

Esses resultados indicam que, embora a GPU ofereça maior velocidade de processamento, as implementações em FPGA são mais eficientes energeticamente, especialmente quando se utiliza múltiplas CUs. Esse aspecto torna a FPGA uma alternativa atraente em termos de eficiência energética, especialmente em cenários onde o consumo de energia é um fator crítico.


%%%%%%%%%%%%%%%%%%%%%%%%%%%%%%%%%%%%%%%%
%%%%%%%%%%% SEGUNDO TRABALHO %%%%%%%%%%%
%%%%%%%%%%%%%%%%%%%%%%%%%%%%%%%%%%%%%%%%

\subsection{FPGA-based implementation of the SHA-256 hash algorithm}

O estudo realizado por \citeonline{kammoun2020} aborda a implementação do algoritmo de hash SHA-256 em FPGAs, com foco em otimizar o desempenho e reduzir o consumo de energia para aplicações que exigem alta segurança, como assinaturas digitais e criptografia em transações eletrônicas. 

A implementação proposta utiliza a ferramenta Vivado HLS da Xilinx para realizar a Síntese de Alto Nível (HLS), permitindo que o código em C seja convertido em uma descrição de hardware, facilitando o desenvolvimento e melhorando o desempenho. Essa abordagem permite aplicar diretivas de otimização, como \textit{pipeline} e \textit{unroll}, que reduzem a latência e aumentam a taxa de transferência \textit{(throughput)}, aspectos essenciais para a mineração de Bitcoin.

Os resultados obtidos indicam uma redução de 73\% no consumo de energia e um aumento de 17\% na velocidade de execução em comparação com uma implementação puramente em software. Além disso, uma arquitetura otimizada alcançou um incremento de 66\% na taxa de transferência em relação a estudos anteriores, demonstrando a eficácia do uso de FPGA para tarefas intensivas em computação.

O estudo também explora uma arquitetura híbrida SW/HW, combinando o processador ARM Cortex-A9 e o acelerador em FPGA, permitindo que o FPGA seja utilizado exclusivamente para o cálculo do SHA-256, enquanto o processador ARM gerencia outras operações. Esse modelo híbrido proporciona um bom equilíbrio entre flexibilidade e eficiência, confirmando o potencial do FPGA para aplicações de mineração de Bitcoin com eficiência energética e alta taxa de \textit{hash}.

A \autoref{fig:kammoun_execution} apresenta os resultados do tempo de execução do algoritmo SHA-256 em duas abordagens: uma implementação puramente em software (SW) e uma implementação híbrida de software e hardware (SW/HW HLS) utilizando Síntese de Alto Nível (HLS). Observa-se que a abordagem SW/HW HLS proporciona uma redução significativa no tempo de execução em comparação com a solução apenas em software. Essa melhoria em picossegundos destaca a vantagem de utilizar técnicas de HLS para otimizar o desempenho, o que é particularmente relevante em aplicações de alta demanda computacional, como a mineração de criptomoedas.

\begin{figure}[H]
    \centering
    \caption{Resultados dos tempos de execução do algorítimos SHA-256.}
    \includegraphics[width=0.75\linewidth]{trabalhos//img/kammoun_execution.png}
    %\caption{Execution time results of the SHA-256 algorithm 
    \caption*{Fonte: \cite{kammoun2020}.}
    \label{fig:kammoun_execution}
\end{figure}

Enquanto na \autoref{fig:kammoun_energy} exibe o consumo de energia das duas abordagens de implementação do algoritmo SHA-256: a implementação puramente em software (SW) e a implementação híbrida de software e hardware (SW/HW HLS). A abordagem SW/HW HLS demonstra uma considerável redução no consumo energético em relação à solução apenas em software. Esse resultado evidencia a eficiência energética da utilização de FPGAs com HLS, mostrando-se uma alternativa mais sustentável para operações intensivas em processamento, como a mineração de Bitcoin, onde o consumo de energia é um fator decisivo.

\begin{figure}[H]
    \centering
    \caption{Power consumption estimation.}
    \includegraphics[width=0.75\linewidth]{trabalhos//img/kammoun_energy.png}
    \caption*{Fonte: \cite{kammoun2020}.}
    \label{fig:kammoun_energy}
\end{figure}


Os testes experimentais validaram a eficácia da implementação, destacando o FPGA como uma solução viável para melhorar o desempenho e reduzir o consumo de energia em comparação com arquiteturas baseadas apenas em software.


%%%%%%%%%%%%%%%%%%%%%%%%%%%%%%%%%%%%%%%%
%%%%%%%%%%% TERCEIRO TRABALHO %%%%%%%%%%
%%%%%%%%%%%%%%%%%%%%%%%%%%%%%%%%%%%%%%%%

\subsection{The Evolution of Bitcoin Hardware}

O estudo realizado por \citeonline{bedford2017} explora a evolução do hardware de mineração de Bitcoin, desde o uso inicial de CPUs até o desenvolvimento de circuitos integrados específicos para aplicações (ASICs). O autor examina o impacto do crescimento do valor do Bitcoin no aprimoramento da tecnologia de mineração, destacando como a eficiência energética e o desempenho foram melhorados ao longo das gerações de hardware.

Taylor detalha as transições tecnológicas que ocorreram no hardware de mineração, com foco especial nos ASICs, que atualmente dominam a mineração de Bitcoin. Ele descreve a ascensão dos datacenters especializados, conhecidos como “ASIC clouds”, que permitem uma escala de computação em nível planetário, otimizada para reduzir custos e maximizar a eficiência. O autor também aborda como a integração vertical da indústria permitiu que empresas controlassem cada aspecto do processo, desde o design dos chips até a manutenção dos datacenters.

Com o tempo, a dificuldade continuou a subir, levando ao uso de FPGAs, que proporcionaram uma melhoria significativa em desempenho e eficiência energética em relação às GPUs. No entanto, a demanda crescente e a competitividade da mineração resultaram na criação de máquinas baseadas em ASICs, dispositivos projetados especificamente para a mineração de Bitcoin. Os ASICs passaram por avanços rápidos na miniaturização de transistores, com processos de fabricação em diferentes nós de VLSI (Very-Large-Scale Integration), como 130 nm, 110 nm, 65 nm, 55 nm, 28 nm, 22 nm, 20 nm e, finalmente, 16 nm.

A \autoref{fig:bedford_diff} ilustra o aumento exponencial da dificuldade de mineração de Bitcoin ao longo dos anos, desde 2009 até 2018, e mostra como a introdução de novas tecnologias de hardware foi essencial para acompanhar esse crescimento. Inicialmente, a mineração era realizada em CPUs, mas rapidamente se tornou inviável devido ao aumento da dificuldade. Como resposta, as GPUs foram adotadas, oferecendo maior poder de processamento para os cálculos do algoritmo SHA-256.

\begin{figure}[H]
    \centering
    \caption{Nível de dificuldade da mineração ao longo dos anos.}
    \includegraphics[width=0.70\linewidth]{trabalhos//img/bedford_diff.png}
    \caption*{Fonte: \cite{bedford2017}}
    \label{fig:bedford_diff}
\end{figure}

Essas evoluções tecnológicas permitiram que a mineração de Bitcoin acompanhasse o aumento da dificuldade, tornando-se cerca de 850 bilhões de vezes mais difícil do que no início da rede. A figura destaca como cada inovação em hardware foi introduzida em resposta ao aumento da dificuldade, enfatizando a importância da adaptação tecnológica para manter a competitividade e a eficiência energética na mineração.

Outro ponto importante discutido é a viabilidade econômica da mineração, onde os custos de operação e manutenção de \textit{rigs} de mineração são comparados com o retorno financeiro da mineração. Taylor destaca como a dificuldade crescente da mineração exige melhorias contínuas no hardware para manter a lucratividade, forçando os mineradores a substituir equipamentos com frequência para acompanhar a evolução tecnológica.

Esses tópicos são particularmente relevantes para o contexto da mineração de criptomoedas, ao mostrarem a importância da inovação em hardware e da eficiência energética, aspectos centrais para a viabilidade de uma arquitetura baseada em FPGA para a mineração de Bitcoin.







%%%%%%%%%%%%%%%%%%%%%%%%%%%%%%%%%%%%%%%%
%%%%%%%%%%% ANALISE %%%%%%%%%%
%%%%%%%%%%%%%%%%%%%%%%%%%%%%%%%%%%%%%%%%





\subsection{Análise Comparativa}

Nesta seção, são apresentadas as principais características dos estudos discutidos anteriormente. As informações principais de cada trabalho, como propósito, tecnologias utilizadas, implementação e resultados obtidos, são dispostas no \autoref{Q:Trabalhos_Relacionados}. Os trabalhos \citeonline{dang2017}, \citeonline{kammoun2020} e \citeonline{bedford2017} abordam diferentes aspectos da utilização de FPGAs e tecnologias associadas para otimização e eficiência em tarefas intensivas de processamento.

\begin{quadro}[H]
    \centering
    \caption{Análise comparativa dos trabalhos relacionados.}
    \vspace{-6pt}
    \resizebox{\textwidth}{!}{%
    \begin{tabular}{|p{3.0cm}|p{3.5cm}|p{3.5cm}|p{3.5cm}|p{3.5cm}|}
\hline
\rowcolor[HTML]{EFEFEF} 
\textbf{Trabalho} &
  \textbf{Propósito do Trabalho} &
  \textbf{\begin{tabular}[c]{@{}c@{}}Tecnologias Usadas\end{tabular}} &
  \textbf{\begin{tabular}[c]{@{}c@{}}Implementação\end{tabular}} &
  \textbf{\begin{tabular}[c]{@{}c@{}}Resultados Obtidos\end{tabular}} \\ \hline

\textbf{Dang e Skadron (2017)} &
  Acelerar algoritmos de mineração de conjuntos frequentes (Frequent Itemset Mining) em FPGA &
  Xilinx FPGA XCKU115: Lógica Reconfigurável e SPP &
  Implementação do algoritmo de expansão de fronteira (Frontier Expansion - FE) em C/C++ com HLS &
  Aceleração de até 3,2x em relação à CPU, com eficiência energética superior à da GPU \\ \hline

\textbf{Kammoun et al. (2020)} &
  Implementação do algoritmo SHA-256 em FPGA para aplicações de segurança e eficiência energética &
  %Vivado HLS, FPGA, ARM Cortex A9 
  FPGA: Lógica Reconfigurável e SPP; ARM Cortex A9: Full Custon/ASIC e GPP &
  Arquitetura híbrida SW/HW utilizando HLS para SHA-256 com pipeline e unroll &
  Redução de 73\% no consumo de energia e aumento de 17\% na velocidade em relação ao software puro \\ \hline

\textbf{Taylor (2017)} &
  Explorar a evolução do hardware para mineração de Bitcoin, desde CPUs até ASICs &
  CPUs, GPUs, FPGAs, ASICs em diferentes nós de VLSI &
  Estudo comparativo e histórico dos avanços de hardware para mineração &
  Identificação da eficiência dos ASICs em data centers e a necessidade de inovação contínua para acompanhar a dificuldade de mineração \\ \hline
\textbf{Este Trabalho} & Explorar o uso de FPGA para a mineração de Bitcoin utilizando o algoritmo SHA-256 & FPGA: Lógica Reconfigurável. Simulação em IDE & Utilização da arquitetura HW/SW com técnica em HLS & Previsão: aumentar o hashrate em relação à CPU com baixo custo de energia \\\hline
    \end{tabular}%
    }
\label{Q:Trabalhos_Relacionados}
\caption*{Fonte: O autor (2024).}
\end{quadro}


Para uma análise comparativa detalhada, cada trabalho contribui com perspectivas específicas no uso de FPGAs e outras tecnologias de hardware:

\begin{itemize}
     


\item O trabalho de \citeonline{dang2017} foca na aceleração de algoritmos de mineração de conjuntos frequentes utilizando FPGAs, onde foram aplicadas técnicas de HLS com SDAccel e Vivado HLS para melhorar a desempenho de processamento. A implementação do algoritmo FE em FPGA resultou em uma aceleração de até 3,2 vezes em comparação com a CPU, além de oferecer uma eficiência energética superior à da GPU.

\item No estudo de \citeonline{kammoun2020}, a ênfase está na implementação do SHA-256, o algoritmo base da mineração de Bitcoin, com uma arquitetura híbrida SW/HW em FPGA. Utilizando o Vivado HLS e um processador ARM, a abordagem adotada otimizou o uso de pipeline e unroll para aumentar a taxa de transferência e reduzir o consumo de energia, resultando em uma economia de 73\% de energia e um aumento de 17\% na velocidade de execução em relação a uma solução de software puro.

\item Por fim, o trabalho de \citeonline{bedford2017} aborda a evolução dos hardwares de mineração de Bitcoin, descrevendo a transição das CPUs para GPUs, FPGAs e, finalmente, ASICs. Taylor destaca a eficiência dos ASICs em data centers dedicados, os chamados "ASIC clouds", que permitem uma escala massiva de mineração. Esse trabalho é fundamental para compreender a importância da inovação em hardware e eficiência energética, especialmente à medida que a dificuldade da mineração aumenta exponencialmente.
\end{itemize}

Esses estudos reforçam a relevância do uso de FPGAs e outras tecnologias específicas para o contexto da mineração de Bitcoin e outras aplicações intensivas, destacando os avanços em desempenho e eficiência energética proporcionados por implementações dedicadas.









%Fazer um levamento dos pontos positivos e negavos de cada trabalho dentro do contexto do trabalho
%Tabela comparativa com cada coluna sendo um atributado pro seu trabalho

% Trabalho | Tecnologia Circuito Usada | Algoritmos Usados | Métricas Usadas