\chapter{Introdução}\label{Sec:Introducao}


No atual cenário de avanços tecnológicos, as tecnologias de processamento digital desempenham um papel fundamental na solução de problemas computacionais complexos. Com o crescimento exponencial de dados e a necessidade de processamento eficiente, dispositivos como FPGAs (\textit{Field-Programmable Gate Arrays}) têm ganhado destaque por sua flexibilidade e capacidade de personalização na sua configuração \cite{vahid2001embedded}. Esses dispositivos são amplamente utilizados em aplicações que exigem alto desempenho, como sistemas embarcados, aprendizado de máquina e, mais recentemente, em operações relacionadas à tecnologia \textit{blockchain} \cite{Cocco2016BitcoinMining}.

A tecnologia \textit{blockchain}, conhecida por sua capacidade de descentralização e segurança, tem se consolidado como um dos pilares das criptomoedas. Dentre elas, o Bitcoin é o maior representante, sendo amplamente reconhecido como o precursor do uso dessa tecnologia em sistemas financeiros descentralizados. O Bitcoin utiliza o algoritmo SHA-256 (\textit{Secure Hash Algorithm 256-bit}) para assegurar a integridade e a segurança de sua rede. Este algoritmo, embora altamente confiável, exige grande capacidade de processamento, especialmente no contexto da mineração de blocos, um processo essencial para o funcionamento da rede \textit{blockchain} \cite{Zhang2022BlockchainDecentralization}.

Diante da crescente complexidade das operações de mineração e do aumento da demanda por soluções eficientes, técnicas de síntese em alto nível (\textit{High-Level Synthesis — HLS}) têm emergido como uma alternativa para extrair mais desempenho. O HLS permite a tradução de algoritmos escritos em linguagens de alto nível, como C/C++, para descrições de hardware compatíveis com FPGAs, otimizando o desempenho e explorando o paralelismo intrínseco do hardware.

Neste contexto, este trabalho propõe a exploração do uso de FPGAs, configurados por meio de técnicas de HLS, para a implementação do algoritmo SHA-256 visando implementar no processo de mineração de Bitcoin. A proposta visa aliar eficiência energética e desempenho computacional, contribuindo para o desenvolvimento de sistemas mais acessíveis e sustentáveis no campo da mineração de criptomoedas.


A tecnologia blockchain constitui um sistema, analogamente como um livro-razão, no qual cada computador participante atua como um nó responsável por registrar todas as transações realizadas para armazenar o histórico completo dessas operações. Esse modelo descentralizado elimina a necessidade de uma autoridade central ou instituição financeira para garantir o funcionamento da rede \cite{nakamoto2008bitcoin}. A segurança é assegurada pelo fato de que qualquer tentativa de fraude exigiria a alteração de dados em mais de 50\% dos nós simultaneamente, uma tarefa que demanda um poder computacional extremamente alto, tornando-a impraticável dado o tamanho e a robustez da rede atual.


\section{Problematização}
\subsection{A Dificuldade da Mineração}

O processo da mineração de Bitcoin, de adicionar novos blocos à blockchain, se dá pela resolução complexa de problemas matemáticos que garantem a segurança e a integridade da rede. Com o passar do tempo, a dificuldade desses problemas aumentou significativamente. O grau de dificuldade é ajustado aproximadamente a cada duas semanas para garantir que um novo bloco seja encontrado em aproximadamente 10 minutos, independentemente da quantidade de capacidade de processamento na rede de computadores \cite{nakamoto2008bitcoin}.



Como resultado, minerar Bitcoin de forma caseira, isto é, sem grandes \textit{clusters} de computação acessíveis e com hardware convencional, tornou-se impraticável. No início, computadores pessoais, usando seus Processadores e depois Placas de Vídeo, que antes eram suficientes para minerar Bitcoin, sendo que agora são incapazes de competir com as poderosas fazendas de mineração equipadas com ASICs dedicados. Esses dispositivos são projetados especificamente para a mineração de Bitcoin, oferecendo uma eficiência muito superior em processamento, mas também apresentando desafios significativos em termos de custo monetário e consumo de energia \cite{antminer2020bitcoin}.

\subsection{O Custo Monetários e o Consumo Energético}

Os hardwares baseados em ASIC, embora eficientes em processamento para mineração, consomem uma grande quantidade energia elétrica e ocupam um espaço significativo, além do seu alto nível de ruído sonoro e temperatura de operação. Por exemplo, o Bitmain Antminer S21 Pro \cite{S21Pro}, um dos modelos mais eficientes em questão de Watt/TH no mercado, computa 234 TH/s no algoritmo SHA-256. O \textit{hashrate} é uma métrica de poder computacional total de cálculos de \textit{hash} por segundo. Essa máquina tem seu valor de venda de aproximadamente U\$6.318, com um consumo energético de 3.510 Watts, ou seja, uma eficiência energética de 15 Watt/TH \cite{asicminervalue}. Operar várias dessas máquinas em uma fazenda de mineração requer um investimento inicial significativo, além de um custo operacional elevado devido ao grande consumo de energia e à necessidade de sistemas de resfriamento para o hardware de processamento \cite{minerset2021energy}.

\subsection{Espaço e Infraestrutura}

Além do custo e do consumo de energia, as fazendas de mineração requerem um espaço físico considerável e uma infraestrutura adequada para operar, como um grande fornecimento de energia elétrica e estrutura de arrefecimento e exaustão térmica. Grandes galpões são frequentemente utilizados para abrigar centenas ou até milhares de máquinas ASICs. Esse nível de escala está fora do alcance da maioria das pessoas, limitando a participação na mineração de Bitcoin a grandes operadores e investidores com capital significativo \cite{arthurinc}.


\subsection{Solução Proposta}

Diante desse cenário, este trabalho propõe explorar a viabilidade de utilizar dispositivos FPGA, como, por exemplo, o \textit{ZedBoard Zynq-7000 ARM/FPGA SoC Development Board} \cite{Zedboard2024}. 

Para o desenvolvimento deste trabalho, usaremos componentes comerciais prontos (COTS, do inglês \textit{“Commercial off-the-shelf”}), como demonstrado na \autoref{fig:zedboard_png} disponível nos laboratórios da instituição, os quais são produtos de hardware já disponíveis no mercado e projetados para serem utilizados em uma ampla gama de aplicações. A escolha por COTS visa reduzir os custos de desenvolvimento e acelerar o processo de implementação, uma vez que esses componentes já foram testados e validados por fabricantes especializados. Para o desenvolvimento do projeto em FPGA, utilizaremos a técnica de HLS \textit{(High-Level Synthesis)}, no qual permite que algoritmos descritos em linguagens de programação de alto nível, como C ou C++, sejam automaticamente convertidos em descrições de hardware em nível de registrador (RTL) \cite{coussy2009hls}. Essa abordagem simplifica o processo de design de hardware, aumentando a eficiência do desenvolvimento e permitindo ajustes rápidos no projeto para otimização de desempenho e consumo de energia.

\begin{figure}[H]
    \centering
    \includegraphics[width=0.75\linewidth]{1_introducao/ZedBoard Zynq-7000.png}
    \caption{ZedBoard Zynq-7000 ARM/FPGA SoC Development Board}
    \label{fig:zedboard_png}
\end{figure}


O estudo irá consumir de competências adquiridas nas disciplinas de Arquitetura e Organização de Processadores, Projeto de Sistemas Digitais e Projeto de Sistemas Embarcados, abordadas no curso de Engenharia de Computação, em prática \cite{univali_disciplinas}. Também são explorados conteúdos não comuns no curso, como HLS para FPGAs e o uso de componentes COTS para implementação de soluções de mineração de Bitcoin.

\section{Objetivos}

Esta seção formaliza o objetivo geral do trabalho, bem como os objetivos específicos para a validação do projeto proposto, conforme descrito nas subseções a seguir.

\subsection{Objetivo Geral}

Explorar o uso de FPGAs COTS para a mineração de criptomoedas utilizando o algoritmo SHA-256, com a implementação de técnicas de HLS para a configuração do hardware.

\subsection{Objetivos Específicos}
\begin{enumerate}
    \item Analisar as técnicas atuais de mineração de Bitcoin, destacando as vantagens e desvantagens de cada abordagem.
    \item Identificar técnicas e métodos para implementação do algoritmo SHA-256 em FPGA.
    \item Alcançar eficiência energética melhor que em soluções baseadas em CPU e GPU.
    \item Disponibilizar um comparativo com métodos tradicionais de mineração, ASIC\footnote{Na área específica de mineração de criptomoedas, os profissionais desta área chamam equipamentos baseados em ASICs de apenas ASIC.}, CPU e GPU, considerando o \textit{hashrate} final.
\end{enumerate}

O trabalho buscará demonstrar a viabilidade técnica e econômica da utilização do FPGA para a mineração de Bitcoin, contribuindo para a otimização do processo de mineração em termos de desempenho e eficiência energética. A abordagem detalhada em cada objetivo específico deverá garantir uma compreensão profunda dos desafios e soluções possíveis, estabelecendo um quadro claro para futuras pesquisas e desenvolvimentos na área.

%%%%%%%%%%%%%%%%%%%%%%%%%%%%%%%%%%%%%%%%%%%%%%%%%%%%%%%%%%%%%%%%%%%%%%%%%%%%%%%%%%%%%%%%%%
%%%%%%%%%%%%%%%%%%%%%%%%%%%%%    PLANO DE TRABALHO    %%%%%%%%%%%%%%%%%%%%%%%%%%%%%%%%%%%%
%%%%%%%%%%%%%%%%%%%%%%%%%%%%%%%%%%%%%%%%%%%%%%%%%%%%%%%%%%%%%%%%%%%%%%%%%%%%%%%%%%%%%%%%%%
\begin{comment}
\section{PLANO DE TRABALHO}

Este plano de trabalho apresenta as etapas que serão seguidas para atingir os objetivos propostos neste projeto utilizando FPGA e conceitos de HLS. Cada etapa no plano de trabalho, é detalhada com suas respectivas atividades, demonstrando o caminho a ser seguido para construir a solução proposta.

\begin{enumerate}
    \item Levantamento teórico: permitirá identificar e analisar as principais obras e autores que tratam problemas semelhantes.
        \begin{enumerate}
            \item[a.] \underline{Pesquisa bibliográfica:} verificação e levantamento teórico relacionados à pesquisa;
            \item[b.] \underline{Análise dos materiais:} análise dos conteúdos levantados sobre FPGA e HLS;
        \end{enumerate}
    \item Implementação do Algoritmo SHA-256 em C/C++: permitirá desenvolver a base do algoritmo de mineração.
    \begin{enumerate}
        \item[a.] \underline{Desenvolvimento:} codificação do algoritmo SHA-256 em C/C++;
        \item[b.] \underline{Testes de unidade:} realização de testes de unidade para garantir a correta implementação do algoritmo;
    \end{enumerate}
    \item Conversão para Hardware com HLS: permitirá a conversão do código de software para uma descrição de hardware.
    \begin{enumerate}
        \item[a.] \underline{Conversão:} utilização do HLS para converter o código C/C++ em VHDL;
        \item[b.] \underline{Otimização:} aplicação de técnicas de otimização para melhorar o desempenho e a eficiência energética do design;
    \end{enumerate}
    \item Síntese e Implementação no FPGA: permitirá a programação do FPGA com o design desenvolvido.
    \begin{enumerate}
        \item[a.] \underline{Síntese:} utilização do Intel Quartus Prime para a síntese do design em VHDL;
        \item[b.] \underline{Implementação:} programação do FPGA com o design sintetizado;
    \end{enumerate}
    \item Simulação e Verificação: permitirá garantir que o design funcione corretamente antes da implementação física.
    \begin{enumerate}
        \item[a.] \underline{Simulação:} utilização de programa de simulação para o teste do design sintetizado;
        \item[b.] \underline{Verificação funcional:} verificação da funcionalidade e correção de possíveis erros;
    \end{enumerate}
    \item Avaliação de Desempenho e Eficiência Energética: permitirá avaliar a eficiência da solução desenvolvida.
    \begin{enumerate}
        \item[a.] \underline{Benchmarking:} execução de testes de benchmark para avaliar o \textit{hashrate} do sistema;
        \item[b.] \underline{Análise de Consumo:} monitoramento do consumo de energia durante os testes;
        \item[c.] \underline{Eficiência Energética:} cálculo da eficiência energética (Watt/TH) e comparação com outras tecnologias de mineração (CPU, GPU, ASIC);
    \end{enumerate}
    \item Monografia: permitirá a escrita do texto da monografia.
    \begin{enumerate}
        \item[a.] \underline{Monografia:} escrita do texto da monografia;
    \end{enumerate}
\end{enumerate}

\end{comment}


