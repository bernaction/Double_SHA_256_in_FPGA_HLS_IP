\begin{Info}
% Universidade
{UNIVERSIDADE DO VALE DO ITAJAÍ}
% Escola
%{ESCOLA DO MAR, CIÊNCIA E TECNOLOGIA}
{ESCOLA POLITÉCNICA}
% Curso
{CURSO DE ENGENHARIA DE COMPUTAÇÃO}
% Titulo
{IMPLEMENTAÇÃO DO ALGORITMO SHA-256 EM HLS PARA MINERAÇÃO DE BITCOIN EM FPGA}
% Autor
{Bernardo Vannier Soares Pinto}
% Cidade e Data
{Itajaí (SC), Dezembro de 2024}
% Nome da Área de concentração
{Algoritmo SHA-256}
% Orientador(a)
{Felipe Viel, MSc.}
% Coorientador(a) <Nome do Coorientador(a)>, <Titulação> %%%%%%%% Se não tiver coorientador deixe vazio
\end{Info}

%\begin{Dedicatoria}
% Dedicatória
% \end{Dedicatoria}

% \begin{Agradecimentos}
% Agradeço a todos.
% \end{Agradecimentos}

% \begin{Epigrafe}
% Epígrafe
% \end{Epigrafe}

\begin{Resumo}
PINTO, Bernardo V. S. Exploração da aceleração em hardware do SHA-256 em COTS. Itajaí, 2024. \pageref{LastPage} f. Trabalho Técnico-científico de Conclusão de Curso (Graduação em Engenharia de Computação) -- Escola Politécnica, Universidade do Vale do Itajaí, Itajaí, 2024.


Este trabalho explora a utilização de FPGA na mineração de Bitcoin utilizando o algoritmo SHA-256. A mineração de Bitcoin requer equipamentos designados, como ASIC, os quais são altamente eficientes e exigem grande infraestrutura, além de consumirem muita energia, o que os torna inviáveis e inacessíveis para a grande maioria dos entusiastas e mineradores caseiros. Neste contexto, este estudo aborda uma solução alternativa, baseada no uso de FPGA, que são dispositivos facilmente programáveis e acessíveis, como uma alternativa para mineração em menor escala.
O trabalho estuda a aplicação do algoritmo SHA-256 em dispositivos de hardware FPGA por meio de métodos de Síntese de Alto Nível (HLS), que possibilitam a conversão de descrições em linguagens de programação de alto nível, como C/C++. Os objetivos principais envolvem avaliar o rendimento do FPGA em relação ao hashrate (número de hashes por segundo) e a eficiência energética, em comparação com outras tecnologias, tais como ASICs, GPUs e CPUs.
Os resultados alcançados trazem o potencial do FPGA como uma opção para a mineração de Bitcoin, mesmo com restrições no que diz respeito ao desempenho em grande escala. Esta pesquisa auxilia na diversificação das tecnologias de mineração, proporcionando uma alternativa mais viável e adaptável para entusiastas e pequenos mineradores interessados em integrar o ecossistema Bitcoin sem um alto investimento em infraestrutura. A opção de se utilizar um FPGA nesse estudo vem da possibilidade de obter eficiência energética na mineração, além da capacidade técnica lecionada neste curso.

Palavras-chave: Mineração de Bitcoin, FPGA, SHA-256, Síntese de Alto Nível, Eficiência Energética.
\end{Resumo}

\begin{Abstract}

\textit{This work explores the use of FPGA in Bitcoin mining using the SHA-256 algorithm. Bitcoin mining requires dedicated equipment, such as ASICs, which are highly efficient but demand extensive infrastructure and consume significant amounts of energy, making them unfeasible and inaccessible for most enthusiasts and home miners. In this context, this study presents an alternative solution based on the use of FPGA, which are easily programmable and accessible devices, as a viable option for small-scale mining.
The study investigates the application of the SHA-256 algorithm on FPGA hardware devices through High-Level Synthesis (HLS) methods, which enable the conversion of descriptions in high-level programming languages, such as C/C++, into optimized hardware representations. The main objectives include evaluating the FPGA's performance in terms of hashrate (hashes per second) and energy efficiency, in comparison with other technologies such as ASICs, GPUs, and CPUs.
The results demonstrate the potential of FPGA as an option for Bitcoin mining, despite limitations regarding large-scale performance. This research contributes to the diversification of mining technologies, providing a more viable and adaptable alternative for enthusiasts and small-scale miners interested in integrating into the Bitcoin ecosystem without a high infrastructure investment. The choice of FPGA in this study highlights the potential for energy efficiency in mining, along with the technical capabilities developed in this course.}

\textit{Keywords: Bitcoin Mining, FPGA, SHA-256, High-Level Synthesis, Energy Efficiency.}
\end{Abstract}

