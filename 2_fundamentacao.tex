\chapter{Fundamentação Teórica}\label{Sec:Fundamentacao_Teorica}

Neste capítulo, são apresentados os conceitos e definições relevantes para o embasamento teórico dos aspectos necessários para o desenvolvimento da solução proposta neste trabalho. Sendo apresentados também os trabalhos relacionados, que demonstram pontos de similaridade com a presente monografia.

\section{Criptomoedas} 
Criptomoedas são ativos digitais que utilizam criptografia para garantir a segurança, integridade e verificabilidade das transações. Embora muitas criptomoedas, como o Bitcoin, se destaquem pela ausência de uma entidade central controladora e operem em redes descentralizadas, algumas criptomoedas possuem elementos de controle centralizado, seja por desenvolvedores, empresas ou consórcios. As transações em redes descentralizadas são validadas por um sistema de consenso distribuído, enquanto em modelos mais centralizados, podem depender de intermediários específicos para registro e validação \cite{StLouisFed2018}. 


As criptomoedas têm diversas funcionalidades além de serem apenas uma forma de pagamento. Elas também podem ser usadas como ativos de investimento devido à sua alta volatilidade, como meio de transferir riqueza de forma anônima, e até para financiar startups, por meio de ICOs \textit{(Initial Coin Offerings)} \cite{Cryptoeconomics2019}. 

\subsection{Bitcoin}
O Bitcoin é a primeira e mais conhecida criptomoeda, criada em 2008 por Satoshi Nakamoto. Seu objetivo era fornecer uma forma de dinheiro eletrônico descentralizado, permitindo que transações pudessem ser realizadas diretamente entre pessoas, sem a necessidade de intermediários como bancos ou instituições financeiras. O sistema é suportado por um livro-razão público e imutável, conhecido como \textit{blockchain}, que registra todas as transações seguramente \cite{nakamoto2008bitcoin}.

O funcionamento do Bitcoin é baseado na tecnologia \textit{blockchain}, uma estrutura de dados que atua como um livro-razão distribuído e imutável, que registra todas as transações de maneira pública e auditável. A \textit{blockchain} é mantida por uma rede global de computadores, conhecidos como nós (ou \textit{nodes}), que trabalham em conjunto para validar e armazenar as transações em blocos, usando algoritmos criptográficos, garantindo a mesma informação descentralizada e segurança nos dados \cite{nakamoto2008bitcoin}.

Cada transação realizada no sistema Bitcoin é agrupada em um bloco, que contém diversas informações, incluindo a assinatura digital do remetente, o endereço do destinatário e o valor transferido, conforme demonstra na \autoref{fig:blockchain_verify}. Esses blocos são encadeados em uma sequência cronológica e linear por meio de um algoritmo criptográfico, formando uma cadeia de blocos (\textit{blockchain}). Cada bloco contém a \textit{hash} (ou impressão digital criptográfica) do bloco anterior, o que garante que a ordem e a integridade das transações não possam ser alteradas sem invalidar toda a cadeia subsequente \cite{Soltani2022}.

\begin{figure}[H]
    \centering
    \caption{Verificação por uma assinatura digital por \textit{hash} anterior.}
    \includegraphics[width=1\linewidth]{2_fundamentacao/verificacao_blockchain.png}
    \caption*{Fonte: Adaptado de \cite{nakamoto2008bitcoin}}
    \label{fig:blockchain_verify}
\end{figure}

O processo que torna o Bitcoin seguro e resistente a fraudes é chamado de mineração. A mineração é essencial para validar as transações, inserir novos blocos na \textit{blockchain} e introduzir novos bitcoins no sistema. Esse processo utiliza o algoritmo de prova de trabalho (do inglês Proof of Work — PoW), que envolve a resolução de complexos problemas matemáticos pelos mineradores. Para adicionar um novo bloco à \textit{blockchain}, os mineradores competem para encontrar uma solução válida para o problema, usando grande capacidade computacional. O primeiro minerador a encontrar a solução válida é recompensado com novos bitcoins, servindo como incentivo econômico para continuar validando as transações \cite{Ibanez2023}. 

A mineração desempenha um papel vital na segurança da rede Bitcoin, ao tornar o sistema altamente resistente a ataques. Alterar qualquer bloco anterior na \textit{blockchain} exigiria que um atacante refizesse todo o trabalho computacional necessário para resolver os problemas dos blocos subsequentes, o que, na prática, seria extremamente caro e difícil de realizar, devido ao crescente poder computacional da rede \cite{antonopoulos2017mastering}. 

Uma das inovações mais importantes do Bitcoin é a resolução do problema do “gasto duplo”, que impede que uma mesma moeda digital seja utilizada mais de uma vez. Isso é possível graças à verificação realizada pelos mineradores, que competem para resolver complexos problemas matemáticos a fim de adicionar novos blocos à \textit{blockchain} e, em troca, são recompensados com novos bitcoins \cite{Kou2022}.

Além disso, o processo de mineração regula a emissão de novos bitcoins, sendo o Bitcoin projetado com um limite máximo de 21 milhões de unidades. Ainda, dentre as regras de emissão limitada de bitcoins, a cada quatro anos, ocorre um evento conhecido como \textit{halving}, onde a recompensa por mineração é reduzida pela metade, garantindo uma taxa de emissão controlada e previsível \cite{ulrich2014bitcoin}.  O Bitcoin também se diferencia por características como sua divisibilidade, onde a menor fração é chamada de “\textit{satoshi}”, e sua portabilidade digital, permitindo transações globais instantâneas e seguras. Com o passar do tempo, o Bitcoin ampliou suas funcionalidades, sendo utilizado tanto como reserva de valor quanto como meio de pagamento, sendo aceito por empresas e até por países, como El Salvador \cite{ElSalvador2021}. 

\subsection{Blockchain}
A \textit{blockchain} é uma tecnologia de livro-razão distribuído que organiza dados em blocos criptografados e cronologicamente conectados, formando uma cadeia imutável de informações. Cada transação na rede é verificada por nós voluntários e descentralizados, que garantem a validade das informações antes de adicioná-las à cadeia. Esse processo elimina a necessidade de intermediários, proporcionando segurança e transparência. Um bloco contém o \textit{hash} do bloco anterior e dados da transação, garantindo que qualquer tentativa de alteração seja imediatamente detectada. Uma característica fundamental da \textit{blockchain} é a sua descentralização, onde o controle não está em uma única entidade, mas distribuído entre todos os participantes da rede \cite{Bencic2018}.

Além disso, as \textit{blockchains} podem ser públicas (como usadas para as criptomoedas como Bitcoin, Litecoin, Ethereum, Monero), onde qualquer pessoa pode participar, ou privadas, onde apenas usuários autorizados têm acesso (empresas, bancos, \textit{datacenters}). A tecnologia tem sido amplamente aplicada não apenas em criptomoedas, mas também em setores como logística, saúde e finanças \cite{swan2015blockchain}.

Por fim, a verificação das transações em uma \textit{blockchain} é feita por meio de mecanismos de consenso, como a Prova de Trabalho (PoW — Proof of Work) ou Prova de Participação (PoS — Proof of Stake), que determinam como os nós concordam sobre o estado atual deste livro-razão \cite{bashir2017mastering}.

\subsection{Prova de Trabalho (PoW)}
O \textit{Hashcash} foi um sistema de prova de trabalho desenvolvido por Adam Back em 1997 \cite{Back1997}, inicialmente projetado para combater o spam em e-mails. O algoritmo exigia que o remetente de um e-mail resolvesse um desafio computacional para gerar um cabeçalho de e-mail único, o que criava um custo computacional mínimo para cada mensagem enviada. Esse custo seria insignificante para usuários legítimos, mas se tornaria inviável para \textit{spammers} que enviam milhares de e-mails.

No contexto das criptomoedas, o \textit{Hashcash} pode ser visto como um precursor do bitcoin, ao introduzir o conceito de usar poder computacional para resolver problemas criptográficos como um meio de verificação, algo essencial para o sistema de prova de trabalho utilizado no Bitcoin. No entanto, \textit{Hashcash} em si não era uma criptomoeda, mas o mecanismo que ele introduziu acabou sendo adaptado por Satoshi Nakamoto no protocolo Bitcoin, para validar transações e garantir a segurança da rede \textit{blockchain} \cite{nakamoto2008bitcoin}.

\subsection{Dificuldade da Rede}
Para aprofundar a compreensão sobre a dificuldade da rede Bitcoin, mais conhecida na área como \textit{diff}, uma fonte para esta definição é o próprio \textit{whitepaper} do Bitcoin, escrito por \cite{nakamoto2008bitcoin}, que apresenta os fundamentos do protocolo e menciona o ajuste de dificuldade para manter o intervalo de geração de blocos constante em aproximadamente 10 minutos. Seu principal objetivo é garantir que novos blocos sejam adicionados neste intervalo de tempo, independentemente do poder computacional total disponível na rede.

A dificuldade é ajustada automaticamente a cada 2016 blocos, correspondendo a cerca de duas semanas, considerando o intervalo médio de 10 minutos por bloco. Durante esse ajuste, a dificuldade é recalculada com base no tempo gasto para minerar os últimos 2016 blocos. Se o tempo total foi menor que 14 dias, a dificuldade aumenta, tornando o processo de mineração mais desafiador. Caso contrário, a dificuldade diminui, facilitando a mineração. Esse mecanismo é essencial para manter a estabilidade e previsibilidade da emissão de novos bitcoins, além de assegurar a segurança da rede contra possíveis ataques.

\newpage
\begin{lstlisting}[language=C++, caption={Exemplo de código para o cálculo da dificuldade com ajuste de bits}, label={lst:difficulty_code}]
#include <iostream>
#include <cmath>

inline float fast_log(float val){
   int * const exp_ptr = reinterpret_cast <int *>(&val);
   int x = *exp_ptr;
   const int log_2 = ((x >> 23) & 255) - 128;
   x &= ~(255 << 23);
   x += 127 << 23;
   *exp_ptr = x;

   val = ((-1.0f/3) * val + 2) * val - 2.0f/3;
   return ((val + log_2) * 0.69314718f);
} 

float difficulty(unsigned int bits){
    static double max_body = fast_log(0x00ffff), scaland = fast_log(256);
    return exp(max_body - fast_log(bits & 0x00ffffff) + scaland * (0x1d - ((bits & 0xff000000) >> 24)));
}

int main(){
    std::cout << difficulty(0x1b0404cb) << std::endl;
    return 0;
}
\end{lstlisting}

Para simplificar o entendimento do cálculo da dificuldade, podemos descrever basicamente pela fórmula:

\begin{equation}
    \text{Dificuldade} = \text{Dificuldade\_Anterior} \times \frac{\text{Tempo\_Real}}{\text{Tempo\_Esperado}}
\end{equation}
\addEquacao{Dificuldade atual}

Onde:

\begin{itemize}
    \item \textbf{Tempo\_Real}: é o tempo total levado para minerar os últimos 2016 blocos.
    \item \textbf{Tempo\_Esperado}: é o tempo ideal, equivalente a \(2016 \times 10\) minutos.
\end{itemize}

Essa métrica é representada como um número escalar que define o alvo de dificuldade, um valor numérico que os mineradores precisam atingir ou superar ao gerar o \textit{hash} do bloco. O alvo é inversamente proporcional à dificuldade: quanto maior a dificuldade, menor será o alvo, tornando mais desafiador encontrar um \textit{hash} válido.

A dificuldade também desempenha um papel crítico na segurança da rede. Ao tornar o processo de mineração mais difícil quando há um aumento no poder computacional, o sistema se protege contra tentativas de ataques coordenados, como o ataque de 51\%. Além disso, o ajuste dinâmico da dificuldade garante que o sistema permaneça funcional e eficiente, mesmo com mudanças significativas no número de mineradores ativos.

\subsection{Bifurcação da Rede}
A dificuldade de mineração na rede Bitcoin não apenas regula o tempo médio para a mineração de blocos, mas também desempenha um papel essencial na prevenção de bifurcações indesejadas na \textit{blockchain}, mais conhecido na área como \textit{fork}. Como descrito por \cite{bouabdo2022evolutionary}, a probabilidade de bifurcação pode ser modelada como:

\begin{equation}
    1 - \prod_{i=1}^{\lfloor d_{\text{evol-rand}} \rfloor} e^{-shd \cdot \lambda_{\text{mine}} \cdot \psi^i} \leq P(\text{forking}) \leq 1 - \prod_{i=1}^{\lceil d_{\text{evol-rand}} \rceil} e^{-shd \cdot \lambda_{\text{mine}} \cdot \psi^i},
\end{equation}
\addEquacao{Probabilidade da bifurcação}

onde:

\begin{itemize}
    \item \( \psi \) representa a influência das conexões entre os nós, computada como uma combinação de somatórios e distribuições de Poisson.
    \item \( shd \) é o atraso de propagação entre dois mineradores.
    \item \( \lambda_{\text{mine}} \) é a probabilidade de um minerador gerar um bloco dentro de um período de tempo, dada por:
\end{itemize}

\begin{equation}
\lambda_{\text{mine}} = \frac{\text{computational-speed}}{\text{mining-difficulty}}.
\end{equation}
\addEquacao{Probabilidade de gerar um bloco}

A partir dessa relação, a dificuldade mínima necessária para manter a probabilidade de bifurcação abaixo de um limite aceitável pode ser expressa como:

\begin{equation}
\text{mining-diff} \geq -\frac{\text{computational-speed} \cdot \sum_{j=1}^{\lceil d_{\text{evol-rand}} \rceil} shd \cdot \psi^j}{\ln(1-\text{threshold})}.
\end{equation}
\addEquacao{Dificuldade mínima aceitável}



\section{Algoritmos de criptografia aplicada a mineração}
A criptografia é crucial para a proteção e integridade dos sistemas de \textit{blockchain} do qual foi projetado, constituindo-se em um dos alicerces fundamentais da tecnologia de criptomoedas. A utilização de algoritmos criptográficos assegura a segurança e a imutabilidade das transações realizadas em uma \textit{blockchain}. 

\begin{quote}
    \textit{A prova-de-trabalho envolve a procura por um valor que quando calculado o hash,
tal como utilizando o SHA-256, o mesmo comece com um número de bits zero. A média do
trabalho requerida é exponencial ao número de bits zero requeridos e pode ser verificada por
meio da execução de um único hash.} \cite{nakamoto2008bitcoin}
\end{quote}

Esses algoritmos têm a função de validar as transações e também de adicionar novos blocos à rede. O algoritmo SHA-256 é o mais comumente usado na mineração de forma geral, sendo o escolhido pelo Bitcoin e um dos mais relevantes no processo de PoW \cite{FrontiersBlockchain2022}. Já na criptomoeda Ethereum, era minerada pelo seu algoritmo chamado \textit{Ethash} (e suas variantes, como \textit{Etchash}), porém, hoje em dia, essa criptomoeda tornou-se prova de participação (PoS), não sendo mais de força bruta (PoW) como o SHA-256 \cite{coinbase_ethereum}.

\subsection{SHA-256}
O algoritmo SHA-256 (Secure Hash Algorithm 256 bits) é um dos mais empregados em sistemas criptográficos contemporâneos, particularmente na mineração de Bitcoin. Ele foi criado pela Agência Nacional de Segurança (do inglês National Security Agency — NSA) e integra a família de algoritmos SHA-2. O SHA-256 transforma qualquer entrada em uma saída de 256 bits, também chamada de \textit{hash}, como apresentado na \autoref{fig:sha-256_funct}. Na mineração de Bitcoin, esse \textit{hash} permite validar as transações a serem registradas naquele bloco e conferir se houve blocos anteriores, se mantiveram íntegros (\textit{hashes} imutáveis) e assegurar a segurança das informações armazenadas no bloco \cite{Crosby2016}. 

\begin{figure}[H]
    \centering
    \caption{Diagrama de blocos do algoritmo SHA-256.}
    \caption*{(a) Fluxo de execução do SHA-256, incluindo a etapa de pré-processamento, programação da mensagem, que gera 64 palavras de 32 bits, e 64 estágios de compressão.}
    \caption*{(b) Diagrama detalhado de dois estágios de compressão de 256 bits de largura.}
    \label{fig:sha-256_funct}
    \includegraphics[width=1\linewidth]{2_fundamentacao/sha-256_funct.png}
    \caption*{Fonte: Adaptado de \cite{tches10955}.}
    \label{fig:sha-256_funct}
\end{figure}

O SHA-256 opera por meio de operações de \textit{hashing}, que são determinísticas e resistem a colisões. Isso implica que, dada uma entrada, o algoritmo sempre gerará a mesma saída, mas mesmo uma mínima mudança na entrada levará a um resultado completamente distinto. Essa propriedade faz do SHA-256 assegurar a integridade dos dados nas \textit{blockchains}, pois qualquer modificação na informação de um bloco invalidaria todos os blocos subsequentes \cite{Haber1991}. 

Por consequência, o SHA-256 tem um papel fundamental no procedimento de mineração. Os mineradores precisam encontrar um \textit{hash} que cumpra determinadas condições pré-estabelecidas para adicionar um novo bloco à \textit{blockchain}, demandando um grande poder de processamento computacional. O minerador que primeiro identificar um \textit{hash} válido recebe novos bitcoins, assegurando que o processo seja economicamente estimulado. Isso faz do SHA-256 um componente fundamental para a segurança e descentralização do sistema \cite{Crosby2016}. 

O algoritmo SHA-256 é bastante conhecido pela capacidade de resistir a ataques. Até agora, não foram identificadas vulnerabilidades práticas que coloquem em risco sua segurança, tornando-o uma opção segura para sistemas vitais como a \textit{blockchain}. Contudo, uma das objeções ao uso do SHA-256 no Bitcoin é o elevado gasto energético necessário para solucionar as questões criptográficas, o que tem gerado debates acerca de opções mais eficazes, como o PoS \cite{FrontiersBlockchain2022}. 

Em suma, o algoritmo SHA-256 é um componente crucial no processo de mineração de Bitcoin. Ele não só assegura a integridade e a segurança da \textit{blockchain}, como também oferece um sistema equitativo de pagamento aos mineradores. Sua eficácia e resistência a ataques fazem dele uma opção perfeita para sistemas descentralizados, mesmo que o impacto ambiental do processo de mineração seja um assunto em constante discussão \cite{Haber1991} \cite{Crosby2016}. 

\subsection*{Função Hash}
O algoritmo SHA-256 funciona por meio de um processo iterativo, dividindo a entrada em blocos de 512 bits. Cada bloco é processado em 64 rodadas de compressão, que utilizam uma combinação de funções lógicas, deslocamentos e operações aritméticas, como demonstrado na \autoref{fig:sha-256-diagram}. 

\begin{figure}[H]
    \centering
    \caption{Diagrama da função de compressão da família SHA-2}
    \includegraphics[width=0.85\linewidth]{2_fundamentacao/diagram_sha-256.png}
    \caption*{Fonte: Adaptado de \cite{sim2012submission46}}
    \label{fig:sha-256-diagram}
\end{figure}

A estrutura básica das operações é a seguinte:

\begin{itemize}
    \item \textbf{Divisão dos Blocos:} Cada bloco de 512 bits é dividido em palavras de 32 bits (\(W_0, W_1, \dots, W_{15}\)). Outras palavras (\(W_{16}, \dots, W_{63}\)) são geradas dinamicamente utilizando deslocamentos circulares e \textit{XOR}.

    \item \textbf{Estados Internos:} O SHA-256 utiliza 8 registradores principais (\(A, B, C, D, E, F, G, H\)) para armazenar os estados intermediários, quos quais são atualizados a cada rodada.

    \item \textbf{Funções Lógicas:}
    \begin{itemize}
        \item \(Ch(E, F, G) = (E \land F) \oplus (\neg E \land G)\): Seleção condicional baseada em \(E\).
        \item \(Ma(A, B, C) = (A \land B) \oplus (A \land C) \oplus (B \land C)\): Seleção majoritária.
        \item Deslocamentos circulares e shifts (\(\Sigma_0\), \(\Sigma_1\)) para misturar os bits dos estados internos.
    \end{itemize}

    \item \textbf{Atualização dos Registradores:} Em cada rodada, os valores de \(A\) a \(H\) são atualizados com base em combinações das funções \(Ch\), \(Ma\), deslocamentos e constantes específicas (\(K_t\)).
\end{itemize}


\subsection*{Pseudocódigo}

Como exemplo, o pseudocódigo apresentado na Listagem \ref{lst:sha256_transform} será utilizado como base para a configuração do algoritmo SHA-256 no FPGA. Esse algoritmo, descrito em linguagem C, será adaptado para o desenvolvimento em HLS, possibilitando a conversão para descrição de hardware em nível de registro RTL. A utilização dessa abordagem permite explorar o desempenho do FPGA.

\begin{lstlisting}[language=C, caption={Função SHA256\_transform implementada em C.}, label={lst:sha256_transform}]

void sha256_transform(unsigned int data[8], unsigned int data[16], unsigned int ss[8]) {
    WORD reg[8], i, j, t1, t2, m[64];
    int n = 0;

    label0: for (i = 0; i < 16; i++) {
        m[i] = data[i];
    }
    
    label1: for (i = 16; i < 64; i++) {
        m[i] = SIG1(m[i - 2]) + m[i - 7] + SIG0(m[i - 15]) + m[i - 16];
    }

    label2: for (n = 0; n < 8; n++) {
        reg[n] = data[n];
    }
}
\end{lstlisting}
Fonte: Adaptado de \cite{kammoun2020}.

\section{Aceleração em Hardware}
A aceleração em hardware é um princípio aplicado para melhorar a eficiência e a desempenho de sistemas de computação, atribuindo funções específicas a componentes especializados, tais como ASICs, FPGAs e GPUs. Em contraste com o processamento convencional em Unidade de Processamento Central (do inglês Central Processing Unit — CPU), projetados para realizar uma variedade de tarefas, a aceleração em hardware emprega componentes específicos para operações específicas, como a mineração de criptomoedas e outras atividades que necessitam de desempenho elevado no tempo para computação \cite{WevolverASICvsFPGA2024}.

No contexto de mineração de criptomoedas, a demanda por aceleração em hardware na mineração de criptomoedas, particularmente em sistemas como o Bitcoin, surge devido ao aumento da dificuldade da rede. Conforme a complexidade da mineração cresce, o poder de processamento bruto requerido para solucionar a chave do bloco seguinte aumenta \cite{Treiblmaier2021}.

A especialização é uma das principais vantagens da aceleração em hardware. Embora as CPUs sejam concebidas para executar diversas operações, os hardwares voltados para a  aceleração em hardware, tais como FPGAs e ASICs, são desenvolvidos especificamente para executar um número limitado de operações de maneira mais rápida e eficiente. Em relação à mineração de criptomoedas, isso implica que algoritmos como o SHA-256 podem ser executados consideravelmente mais eficiente em um FPGA ou ASIC do que numa CPU convencional \cite{AnalyticsFPGA2024}. 

Ademais, a aceleração por hardware proporciona adaptabilidade no caso dos FPGAs, que podem ser reprogramados para diversas funções conforme a necessidade. Apesar de os FPGAs não serem tão otimizados em questão de tecnologia de circuito quanto os ASICs, eles oferecem uma opção viável ao procurar um equilíbrio entre flexibilidade e desempenho em métricas como tempo de desenvolvimento, tempo de processamento e consumo de energia. Por outro lado, os ASICs são altamente otimizados para uma tarefa específica, como a mineração de criptomoedas, podendo executar cálculos de \textit{hashing} eficientemente \cite{Treiblmaier2021}. 

A aplicação de aceleração em hardware também pode levar a uma eficiência energética superior. No processo de mineração de criptomoedas, o consumo de energia é uma questão relevante. Ao empregar componentes especializados para executar as tarefas de mineração, conseguimos diminuir o uso de energia em relação ao uso de CPUs ou GPUs, tornando a operação mais sustentável e economicamente viável a longo prazo \cite{Jablczynska2023}. 


\subsection{Instruções Dedicadas}

As instruções dedicadas referem-se ao conjunto específico de operações que um núcleo de processamento especializado pode executar. Essas instruções são otimizadas para realizar tarefas específicas, como operações matemáticas complexas, comuns no processo de mineração de criptomoedas. Por exemplo, na mineração de Bitcoin, as instruções focam em operações de \textit{hashing}, comparações e cálculos binários, acelerando o processo de mineração quando implementadas diretamente no hardware.

Em plataformas como FPGAs e ASICs, essas instruções são incorporadas diretamente nos circuitos, proporcionando um aumento significativo na eficiência. Isso ocorre porque o sistema não precisa interpretar uma ampla gama de instruções de uso geral, resultando em economia de recursos computacionais e melhoria na eficiência energética, fatores cruciais na mineração de criptomoedas \cite{8404574, 7967745}.


\subsection{Algoritmo dedicado}
Um algoritmo dedicado é desenvolvido e aprimorado para funcionar apenas em hardware específico, como FPGAs e ASICs, em vez de recorrer a implementações em software que seriam menos eficientes. Esses algoritmos, como o SHA-256 empregado na mineração de Bitcoin, são organizados de maneira a maximizar o potencial dos núcleos de processamento dedicados. A utilização de algoritmos específicos para hardware leva a tempos de execução reduzidos e a uma administração mais eficiente dos recursos de computação disponíveis \cite{sha256_fpga_optimization, sha256_asic_energy}.

\section{Tecnologia de Circuito e de Processador}
As tecnologias de circuitos integrados desempenham um papel essencial na mineração de criptomoedas, principalmente com o uso de dispositivos especializados, como ASICs e FPGAs. Ambas as tecnologias oferecem vantagens distintas em termos de desempenho, eficiência e flexibilidade, dependendo da aplicação e do nível de otimização necessário para o processo de mineração \cite{hennessy2017computer}.

\subsection{Tecnologia de Circuito}

\subsubsection{Totalmente Customizável (Full-custom)}
Na tecnologia full-custom, todas as camadas do circuito, incluindo transistores, portas lógicas e conexões, são personalizadas. Essa abordagem visa otimizar o desempenho do circuito final em diversas métricas, como eficiência energética e alta frequência de operação. Aspectos como o tamanho do canal, a dimensão e a posição dos transistores, bem como o roteamento das interconexões, são configurados de forma específica. Após a finalização do projeto, as máscaras são geradas e enviadas para a fabricação do circuito integrado (CI). Embora essa técnica tenha um custo NRE elevado e um tempo de produção mais longo, ela proporciona superioridade em métricas de desempenho quando comparada a outras tecnologias \cite{vahid2001embedded}.

\subsubsection{Semi Customizável (ASIC)}
Os ASICs são circuitos integrados projetados especificamente para executar uma única função de forma extremamente eficiente. Na mineração de criptomoedas, como o Bitcoin, os ASICs são otimizados para o algoritmo SHA-256, tornando-os a escolha mais eficiente em termos de taxa de \textit{hash} e consumo de energia. Por serem projetados exclusivamente para essa função, os ASICs superam amplamente outros tipos de hardware, como GPUs e FPGAs, em desempenho e economia de energia \cite{hennessy2017computer}.

Uma das principais vantagens dos ASICs é a sua eficiência energética. Por serem otimizados para uma tarefa específica, eles consomem muito menos energia em comparação com processadores de propósito geral. No entanto, essa especialização também é uma desvantagem, pois os ASICs não podem ser reconfigurados para realizar outras funções, tornando-os inúteis se a criptomoeda ou o algoritmo de mineração mudar \cite{barr2006asic}.

Além disso, os ASICs modernos avançam em termos de miniaturização de chips, com tecnologias de semicondutores mais recentes permitindo a criação de chips menores e mais eficientes, melhorando ainda mais o desempenho e o resfriamento desses dispositivos \cite{hennessy2017computer}.

\subsubsection{Lógica Programável (FPGA)}
Os FPGAs são dispositivos semicondutores que podem ser programados após a fabricação para realizar uma ampla gama de funções. Ao contrário dos ASICs, projetados para uma tarefa específica, os FPGAs podem ser reprogramados conforme necessário, o que os torna uma opção mais flexível para mineradores que desejam adaptar suas operações a diferentes criptomoedas ou algoritmos \cite{maxfield2004field}. Embora os FPGAs não sejam tão eficientes quanto os ASICs em termos de taxa de \textit{hash}, eles oferecem uma flexibilidade de desenvolvimento e implementação que os ASICs não conseguem igualar. Isso é particularmente útil em cenários onde a criptomoeda minerada pode mudar ou quando novos algoritmos de mineração são introduzidos. Outra vantagem dos FPGAs é sua capacidade de realizar outras tarefas além da mineração, o que os torna uma opção mais versátil para diversas aplicações.

Os elementos que compõe um FPGA podem ser descritos conforme \citeonline{harris2013projeto}:

\begin{quote}
    \textit{As FPGAs são construídas como uma matriz de elementos lógicos configuráveis (LE — logic elements), também conhecidos como configurable logic blocks (CLB). Cada LE pode ser configurado para desempenhar funções combinatórias ou sequenciais. A \autoref{fig:fpga_layout} apresenta um diagrama de blocos geral de uma FPGA. Os LE estão rodeados por elementos de entrada/saída (IOE — input output elements) para a interface com o mundo exterior. Os IOE ligam as entradas e as saídas dos LE aos pinos de empacotamento de chip. OS LE podem-se ligar-se a outros LE e IOE através de canais de encaminhamento programáveis.}
\end{quote} 

\begin{figure}[H]
    \centering
    \caption{Layout genérico de uma FPGA.}
    \includegraphics[width=0.6\linewidth]{2_fundamentacao/fpga_layout.png}
    \label{fig:fpga_layout}
    \caption*{Fonte: \citeonline{harris2013projeto}.}
\end{figure}


\subsection{Tecnologia de processador}
Os processadores desempenham um papel central na computação, com diferentes arquiteturas projetadas para atender a necessidades específicas. Para o contexto de mineração de criptomoedas e tarefas que demandam aceleração em hardware, destacam-se três tipos principais de tecnologia de processador: General Purpose Processors (GPP), SPP e Application-Specific Instruction Processors (ASIP). Cada um oferece vantagens e limitações conforme o tipo de tarefa e o nível de otimização necessário \cite{vahid2001embedded}. Na \autoref{fig:gpp_asip_spp} temos um comparativo sobre cada tecnologia de processador.

\begin{figure}[H]
    \centering
    \caption{A independência das tecnologias de processador e CI: qualquer tecnologia de processador pode ser mapeada para qualquer tecnologia de CI.}
    \includegraphics[width=1\linewidth]{2_fundamentacao/gpp_asip_spp.png}
    \caption*{Fonte: Adaptado de \citeonline{vahid2001embedded}.}
    \label{fig:gpp_asip_spp}
\end{figure}

\subsubsection{GPP}
Os GPPs são processadores projetados para realizar uma ampla gama de tarefas, sendo flexíveis e aplicáveis em diversas aplicações. Eles são comumente usados em computadores pessoais e servidores, onde a versatilidade é importante. No entanto, para aplicações como mineração de criptomoedas, onde o desempenho e a eficiência energética são críticos, os GPPs tendem a ser menos eficientes em comparação com SPPs \cite{vahid2001embedded}. Em geral, os GPPs não são a primeira escolha para mineração devido ao seu consumo de energia elevado e menor capacidade de processamento por Watt. 

\subsubsection{SPP}
Os SPPs são processadores projetados para executar um conjunto limitado de tarefas de maneira eficiente. Eles são otimizados para operações específicas, como a criptografia ou o processamento de dados gráficos, e, por isso, têm desempenho superior em determinadas métricas em suas áreas de especialização \cite{vahid2001embedded}. No contexto da mineração, os SPPs são frequentemente usados em dispositivos como GPUs (Graphic Processing Units), que se tornaram populares para minerar certas criptomoedas antes da dominância dos ASICs. Ao contrário dos GPPs, os SPPs são menos flexíveis, mas oferecem uma melhor relação entre desempenho e consumo de energia para tarefas específicas.

\subsubsection{ASIP}
Os ASIPs são projetados para serem reconfiguráveis, oferecendo um equilíbrio entre flexibilidade e otimização. Eles podem ser ajustados para executar uma variedade de tarefas em um domínio específico, como algoritmos de mineração de criptomoedas. Os ASIPs combinam a eficiência de um processador especializado com a adaptabilidade de um GPP, permitindo que sejam usados em diferentes criptomoedas ou algoritmos com mudanças mínimas no hardware. A principal vantagem dos ASIPs é a sua capacidade de personalização, o que os torna adequados para mineradores que buscam otimizar suas operações sem investir em ASICs, os quais são menos flexíveis \cite{vahid2001embedded}.


\section{Comparação sobre tecnologias de circuito para \\ mineração de criptomoedas}
As tecnologias de circuitos desempenham um papel fundamental na mineração de criptomoedas, oferecendo diferentes características de desempenho, custo e flexibilidade. Neste contexto, destacam-se duas tecnologias principais: FPGA  e ASIC.

Os FPGAs são dispositivos semicondutores programáveis, que permitem uma ampla gama de reconfigurações após a fabricação. Isso os torna altamente versáteis e uma escolha ideal para aplicações onde a flexibilidade é essencial, como na mineração de criptomoedas, onde o algoritmo de mineração pode mudar com o tempo. No entanto, essa flexibilidade tem um custo em termos de desempenho e consumo de energia, já que os FPGAs geralmente não são tão eficientes quanto os ASICs para tarefas específicas.

Por outro lado, os ASICs são projetados para realizar uma função específica de forma extremamente eficiente. A desvantagem é a falta de flexibilidade, já que os ASICs não podem ser reconfigurados para outros algoritmos \cite{vahid2001embedded}. Na mineração de criptomoedas, os ASICs são frequentemente otimizados para o algoritmo de mineração SHA-256 (utilizado pelo Bitcoin), proporcionando uma taxa de hash muito maior com um consumo de energia reduzido em comparação com os FPGAs.