\documentclass[
	% -- opções da classe memoir --
	12pt,				% tamanho da fonte
	openright,			% capítulos começam em pág ímpar (insere página vazia caso preciso)
	oneside,			% para impressão em um lado. Oposto a twoside, verso e anverso
	a4paper,			% tamanho do papel. 
	% -- opções da classe abntex2 --
	chapter=TITLE,		% títulos de capítulos convertidos em letras maiúsculas
	section=TITLE,		% títulos de seções convertidos em letras maiúsculas
	%subsection=TITLE,	% títulos de subseções convertidos em letras maiúsculas
	%subsubsection=TITLE,% títulos de subsubseções convertidos em letras maiúsculas
	% -- opções do pacote babel --
	english,			% idioma adicional para hifenização
	%french,				% idioma adicional para hifenização
	%spanish,			% idioma adicional para hifenização
	brazil,				% o último idioma é o principal do documento
	]{z_styles/abntex2}

\usepackage[alf,
%versalete,
abnt-emphasize = bf, % destaca o titulo da revista ou livro em negrito;
abnt-etal-list = 3, % trabalhos com mais de 3 autores recebem et al.,;
%abnt-etal-text = it, % escreve o et al., em italico;
%abnt-and-type = e, % usa o caracter '&' no lugar de 'e' para mais de um autor;
abnt-last-names = abnt, % trata sobrenomes 'estritamente' conforme a ABNT;
abnt-repeated-author-omit = no % autores com + de uma entrada recebem '____.'
]{z_styles/abntex2cite}

\usepackage[utf8]{inputenc}

%\usepackage[dvipsnames]{xcolor}
\usepackage[table,xcdraw,dvipsnames]{xcolor}
%\usepackage{subfigure}   % Pacote para subfiguras
\usepackage{amsmath}
\usepackage{lastpage}
\usepackage{times}
\usepackage{graphicx}
\usepackage{caption}
\usepackage{tocloft}
\usepackage{tabularx}
\usepackage{booktabs}
\usepackage{enumitem}
\usepackage{dirtytalk}
\usepackage{tikz}
\usetikzlibrary{tikzmark}
%\usepackage{array} 
\usepackage{multirow}
\usepackage{threeparttable} % to use table notes


%%%%%%%%%%%%%%%%%%%%%%%%%%%%%%%%%%%%%%%%%%%%%%%%%%%%%%%%%%%%%%%%%
%%%%%%%%%%%%%%%%%%%%%%%%%%%%%%%%%%%%%%%%%%%%%%%%%%%%%%%%%%%%%%%%%
%%%%%%%%%%%%%%%%%%%%%%%%%%%%%%%%%%%%%%%%%%%%%%%%%%%%%%%%%%%%%%%%%
\usepackage{listings}
\usepackage{xcolor}

%New colors defined below
\definecolor{codegreen}{rgb}{0,0.6,0}
\definecolor{codegray}{rgb}{0.5,0.5,0.5}
\definecolor{codepurple}{rgb}{0.58,0,0.82}
\definecolor{backcolour}{rgb}{0.95,0.95,0.92}

%Code listing style named "mystyle"
\lstdefinestyle{mystyle}{
  backgroundcolor=\color{backcolour}, commentstyle=\color{codegreen},
  keywordstyle=\color{magenta},
  numberstyle=\tiny\color{codegray},
  stringstyle=\color{codepurple},
  basicstyle=\ttfamily\footnotesize,
  breakatwhitespace=false,         
  breaklines=true,                 
  captionpos=b,                    
  keepspaces=true,                 
  numbers=left,                    
  numbersep=5pt,                  
  showspaces=false,                
  showstringspaces=false,
  showtabs=false,                  
  tabsize=2
}

%"mystyle" code listing set
\lstset{style=mystyle}
%%%%%%%%%%%%%%%%%%%%%%%%%%%%%%%%%%%%%%%%%%%%%%%%%%%%%%%%%%%%%%%%%
%%%%%%%%%%%%%%%%%%%%%%%%%%%%%%%%%%%%%%%%%%%%%%%%%%%%%%%%%%%%%%%%%
%%%%%%%%%%%%%%%%%%%%%%%%%%%%%%%%%%%%%%%%%%%%%%%%%%%%%%%%%%%%%%%%%


% Criação do symbol da função sign
\DeclareMathOperator{\sign}{sign}

\newcolumntype{C}[1]{>{\centering\let\newline\\\arraybackslash\hspace{0pt}}m{#1}}
\newcolumntype{L}[1]{>{\RaggedRight\let\newline\\\arraybackslash\hspace{0pt}}m{#1}}

% Alinhamento de legenda à esquerda
\captionsetup{justification=raggedright,singlelinecheck=false}


\usepackage{z_styles/ttc_univali}


\begin{document}

\pretextual

\begin{Info}
% Universidade
{UNIVERSIDADE DO VALE DO ITAJAÍ}
% Escola
%{ESCOLA DO MAR, CIÊNCIA E TECNOLOGIA}
{ESCOLA POLITÉCNICA}
% Curso
{CURSO DE ENGENHARIA DE COMPUTAÇÃO}
% Titulo
{IMPLEMENTAÇÃO DO DOUBLE SHA-256 EM HLS PARA FPGA UTILIZANDO BLOCOS REAIS DO BITCOIN}
% Autor
{Bernardo Vannier Soares Pinto}
% Cidade e Data
{Itajaí (SC), Junho de 2025}
% Nome da Área de concentração
{Algoritmo SHA-256}
% Orientador(a)
{Felipe Viel, MSc.}
% Coorientador(a) <Nome do Coorientador(a)>, <Titulação> %%%%%%%% Se não tiver coorientador deixe vazio1

\end{Info}

%\begin{Dedicatoria}
% Dedicatória
% \end{Dedicatoria}

% \begin{Agradecimentos}
% Agradeço a todos.
% \end{Agradecimentos}

% \begin{Epigrafe}
% Epígrafe
% \end{Epigrafe}

\begin{Resumo}
PINTO, Bernardo V. S. Implementação do double SHA-256 em FPGA com HLS utilizando blocos reais do Bitcoin. Itajaí, 2024. \pageref{LastPage} f. Trabalho Técnico-científico de Conclusão de Curso (Graduação em Engenharia de Computação) -- Escola Politécnica, Universidade do Vale do Itajaí, Itajaí, 2024.

O algoritmo criptográfico SHA-256 desempenha papel central na integridade e segurança da blockchain do Bitcoin, sendo aplicado duas vezes consecutivas no processo conhecido como Double SHA-256. Apesar de sua confiabilidade, o algoritmo demanda elevado poder computacional, o que o torna desafiador para sistemas embarcados com restrições de energia e desempenho. Este trabalho teve como objetivo avaliar a viabilidade de implementar o Double SHA-256 em hardware reconfigurável por meio de High-Level Synthesis (HLS), utilizando plataformas FPGA comerciais. A arquitetura foi desenvolvida como um IP personalizado e integrado a sistemas bare-metal nas plataformas ZedBoard (Zynq-7000) e ZCU104 (Zynq UltraScale+ MPSoC), validando blocos reais da blockchain para garantir fidelidade funcional. Os resultados mostraram que a ZCU104 atingiu desempenho de 319 kH/s com tempo de execução de 3,1~\textmu{}s por hash, enquanto a ZedBoard se destacou em eficiência energética, com 29,625 hashes por watt. Ambas as implementações utilizaram apenas um núcleo lógico, com consumo reduzido de recursos da FPGA. Concluiu-se que a abordagem proposta é eficaz e energeticamente eficiente, adequada para aplicações embarcadas que exigem criptografia de alto desempenho.

\textbf{Palavras-chave}: SHA-256, FPGA, HLS, Blockchain, Eficiência energética, Sistemas embarcados, Validação de blocos.
\end{Resumo}

\begin{comment}
Problema: SHA-256 é custoso computacionalmente e usado no núcleo da blockchain;
Objetivo: avaliar viabilidade e eficiência da implementação em FPGA usando HLS;
Proposta: validação de blocos reais com Double SHA-256 em hardware reconfigurável;
Resultados: desempenho, eficiência energética e comparação com CPU embarcada;
Conclusão: FPGA com HLS é uma solução viável para aplicações embarcadas de alto desempenho.
\end{comment}



\begin{Abstract}

The SHA-256 cryptographic algorithm plays a central role in ensuring the integrity and security of the Bitcoin blockchain and is applied twice in succession in the Double SHA-256 process. Despite its robustness, the algorithm demands significant computational power, posing challenges for embedded systems with constraints on energy and performance. This work aimed to evaluate the feasibility of implementing Double SHA-256 on reconfigurable hardware using High-Level Synthesis (HLS), targeting commercially available FPGA platforms. The architecture was developed as a custom IP and integrated into bare-metal systems on ZedBoard (Zynq-7000) and ZCU104 (Zynq UltraScale+ MPSoC) platforms, validating real blockchain headers to ensure functional fidelity. The results showed that the ZCU104 reached a performance of 319 kH/s with an execution time of 3.1~\textmu{}s per hash, while the ZedBoard excelled in energy efficiency with 29,625 hashes per watt. Both implementations used a single logic core with low FPGA resource usage. The results confirmed that the proposed approach is effective and energy-efficient, making it suitable for high-performance cryptographic applications in embedded systems.

\textbf{Keywords}: SHA-256, FPGA, HLS, Blockchain, Energy efficiency, Embedded systems, Block validation.
\end{Abstract}


 \clearpage

% ---
% inserir lista de ilustrações
% ---
\pdfbookmark[0]{\listfigurename}{lof}
\listoffigures*
\cleardoublepage
% ---

% ---
% inserir lista de tabelas
% ---
%\pdfbookmark[0]{\listtablename}{lot}
%\listoftables*
%\cleardoublepage
% ---

% ---
% inserir lista de quadros
% ---
\pdfbookmark[0]{\listofquadrosname}{loq}
\listofquadros*
\cleardoublepage
% ---

% ---
% inserir lista de equações
% ---
\pdfbookmark[0]{\listofequacaoname}{loe}
\listofequacao*
\cleardoublepage
% ---

% ---
% inserir lista de abreviaturas e siglas
% ---
\begin{siglas}
\setlength{\itemsep}{0pt}%
\setlength{\parskip}{0pt}%
    \item[ASIC] Application-Specific Integrated Circuit
    \item[ASIP] Application-Specific Instruction Processor
    \item[BTC] Bitcoin
    \item[CLB] Configurable Logic Blocks
    \item[COTS] Commercial off-the-shelf
    \item[CPU] Central Processing Unit
    \item[ETH] Ethereum
    \item[FPGA] Field-Programmable Gate Array
    \item[GB] Gigabyte
    \item[GPU] Graphics Processing Unit
    \item[GPP] General-Purpose Processor
    \item[HLS] High-Level Synthesis
    \item[HPS] Hard Processor System
    \item[IOE] Input Output Elements
    \item[LE] Logic Elements
    \item[LUTs] Look-Up Tables
    \item[PoW] Proof of Work
    \item[PoS] Proof of Stake
    \item[SHA-256] Secure Hash Algorithm 256-bit
    \item[SPP] Single-Purpose Processor
    \item[VLSI] Very Large Scale Integration
    \item[VHDL] VHSIC Hardware Description Language
    \item[VHSIC] Very High Speed Integrated Circuits
\end{siglas}

% ---

% ---
%inserir lista de símbolos
% ---
\begin{simbolos}
\setlength{\itemsep}{0pt}%
\setlength{\parskip}{0pt}%
    \item[$TH/W$] Eficiência Energética (Terahash por Watt)
    \item[$MHz$] Megahertz
    \item[$mm$] Milímetro
    \item[$ms$] Milissegundo
    \item[$nm$] Nanômetro

\end{simbolos}
% ---

% ---
% inserir o sumario
% ---
\pdfbookmark[0]{\contentsname}{toc}
\tableofcontents*
\cleardoublepage
% --- \clearpage

\textual % Indica inicio dos elementos textuais
\pagestyle{simple} % remove o cabecalho
\chapter{Introdução}\label{Sec:Introducao}


No atual cenário de avanços tecnológicos, as tecnologias de processamento digital desempenham um papel fundamental na solução de problemas computacionais complexos. Com o crescimento exponencial de dados e a necessidade de processamento eficiente, dispositivos como FPGAs (\textit{Field-Programmable Gate Arrays}) têm ganhado destaque por sua flexibilidade e capacidade de personalização na sua configuração \cite{vahid2001embedded}. Esses dispositivos são amplamente utilizados em aplicações que exigem alto desempenho, como sistemas embarcados, aprendizado de máquina e, mais recentemente, em operações relacionadas à tecnologia \textit{blockchain} \cite{Cocco2016BitcoinMining}.

A tecnologia \textit{blockchain}, conhecida por sua capacidade de descentralização e segurança, tem se consolidado como um dos pilares das criptomoedas. Dentre elas, o Bitcoin é o maior representante, sendo amplamente reconhecido como o precursor do uso dessa tecnologia em sistemas financeiros descentralizados. O Bitcoin utiliza o algoritmo SHA-256 (\textit{Secure Hash Algorithm 256-bit}) para assegurar a integridade e a segurança de sua rede. Este algoritmo, embora altamente confiável, exige grande capacidade de processamento, especialmente no contexto da mineração de blocos, um processo essencial para o funcionamento da rede \textit{blockchain} \cite{Zhang2022BlockchainDecentralization}.

Diante da crescente complexidade das operações de mineração e do aumento da demanda por soluções eficientes, técnicas de síntese em alto nível (\textit{High-Level Synthesis — HLS}) têm emergido como uma alternativa para extrair mais desempenho. O HLS permite a tradução de algoritmos escritos em linguagens de alto nível, como C/C++, para descrições de hardware compatíveis com FPGAs, otimizando o desempenho e explorando o paralelismo intrínseco do hardware.

Neste contexto, este trabalho propõe a exploração do uso de FPGAs, configurados por meio de técnicas de HLS, para a implementação do algoritmo SHA-256 visando implementar no processo de mineração de Bitcoin. A proposta visa aliar eficiência energética e desempenho computacional, contribuindo para o desenvolvimento de sistemas mais acessíveis e sustentáveis no campo da mineração de criptomoedas.


A tecnologia blockchain constitui um sistema, analogamente como um livro-razão, no qual cada computador participante atua como um nó responsável por registrar todas as transações realizadas para armazenar o histórico completo dessas operações. Esse modelo descentralizado elimina a necessidade de uma autoridade central ou instituição financeira para garantir o funcionamento da rede \cite{nakamoto2008bitcoin}. A segurança é assegurada pelo fato de que qualquer tentativa de fraude exigiria a alteração de dados em mais de 50\% dos nós simultaneamente, uma tarefa que demanda um poder computacional extremamente alto, tornando-a impraticável dado o tamanho e a robustez da rede atual.


\section{Problematização}
\subsection{A Dificuldade da Mineração}

O processo da mineração de Bitcoin, de adicionar novos blocos à blockchain, se dá pela resolução complexa de problemas matemáticos que garantem a segurança e a integridade da rede. Com o passar do tempo, a dificuldade desses problemas aumentou significativamente. O grau de dificuldade é ajustado aproximadamente a cada duas semanas para garantir que um novo bloco seja encontrado em aproximadamente 10 minutos, independentemente da quantidade de capacidade de processamento na rede de computadores \cite{nakamoto2008bitcoin}.



Como resultado, minerar Bitcoin de forma caseira, isto é, sem grandes \textit{clusters} de computação acessíveis e com hardware convencional, tornou-se impraticável. No início, computadores pessoais, usando seus Processadores e depois Placas de Vídeo, que antes eram suficientes para minerar Bitcoin, sendo que agora são incapazes de competir com as poderosas fazendas de mineração equipadas com ASICs dedicados. Esses dispositivos são projetados especificamente para a mineração de Bitcoin, oferecendo uma eficiência muito superior em processamento, mas também apresentando desafios significativos em termos de custo monetário e consumo de energia \cite{antminer2020bitcoin}.

\subsection{O Custo Monetários e o Consumo Energético}

Os hardwares baseados em ASIC, embora eficientes em processamento para mineração, consomem uma grande quantidade energia elétrica e ocupam um espaço significativo, além do seu alto nível de ruído sonoro e temperatura de operação. Por exemplo, o Bitmain Antminer S21 Pro \cite{S21Pro}, um dos modelos mais eficientes em questão de Watt/TH no mercado, computa 234 TH/s no algoritmo SHA-256. O \textit{hashrate} é uma métrica de poder computacional total de cálculos de \textit{hash} por segundo. Essa máquina tem seu valor de venda de aproximadamente U\$6.318, com um consumo energético de 3.510 Watts, ou seja, uma eficiência energética de 15 Watt/TH \cite{asicminervalue}. Operar várias dessas máquinas em uma fazenda de mineração requer um investimento inicial significativo, além de um custo operacional elevado devido ao grande consumo de energia e à necessidade de sistemas de resfriamento para o hardware de processamento \cite{minerset2021energy}.

\subsection{Espaço e Infraestrutura}

Além do custo e do consumo de energia, as fazendas de mineração requerem um espaço físico considerável e uma infraestrutura adequada para operar, como um grande fornecimento de energia elétrica e estrutura de arrefecimento e exaustão térmica. Grandes galpões são frequentemente utilizados para abrigar centenas ou até milhares de máquinas ASICs. Esse nível de escala está fora do alcance da maioria das pessoas, limitando a participação na mineração de Bitcoin a grandes operadores e investidores com capital significativo \cite{arthurinc}.


\subsection{Solução Proposta}

Diante desse cenário, este trabalho propõe explorar a viabilidade de utilizar dispositivos FPGA, como, por exemplo, o \textit{ZedBoard Zynq-7000 ARM/FPGA SoC Development Board} \cite{Zedboard2024}. 

Para o desenvolvimento deste trabalho, usaremos componentes comerciais prontos (COTS, do inglês \textit{“Commercial off-the-shelf”}), como demonstrado na \autoref{fig:zedboard_png} disponível nos laboratórios da instituição, os quais são produtos de hardware já disponíveis no mercado e projetados para serem utilizados em uma ampla gama de aplicações. A escolha por COTS visa reduzir os custos de desenvolvimento e acelerar o processo de implementação, uma vez que esses componentes já foram testados e validados por fabricantes especializados. Para o desenvolvimento do projeto em FPGA, utilizaremos a técnica de HLS \textit{(High-Level Synthesis)}, no qual permite que algoritmos descritos em linguagens de programação de alto nível, como C ou C++, sejam automaticamente convertidos em descrições de hardware em nível de registrador (RTL) \cite{coussy2009hls}. Essa abordagem simplifica o processo de design de hardware, aumentando a eficiência do desenvolvimento e permitindo ajustes rápidos no projeto para otimização de desempenho e consumo de energia.

\begin{figure}[H]
    \centering
    \includegraphics[width=0.75\linewidth]{1_introducao/ZedBoard Zynq-7000.png}
    \caption{ZedBoard Zynq-7000 ARM/FPGA SoC Development Board}
    \label{fig:zedboard_png}
\end{figure}


O estudo irá consumir de competências adquiridas nas disciplinas de Arquitetura e Organização de Processadores, Projeto de Sistemas Digitais e Projeto de Sistemas Embarcados, abordadas no curso de Engenharia de Computação, em prática \cite{univali_disciplinas}. Também são explorados conteúdos não comuns no curso, como HLS para FPGAs e o uso de componentes COTS para implementação de soluções de mineração de Bitcoin.

\section{Objetivos}

Esta seção formaliza o objetivo geral do trabalho, bem como os objetivos específicos para a validação do projeto proposto, conforme descrito nas subseções a seguir.

\subsection{Objetivo Geral}

Explorar o uso de FPGAs COTS para a mineração de criptomoedas utilizando o algoritmo SHA-256, com a implementação de técnicas de HLS para a configuração do hardware.

\subsection{Objetivos Específicos}
\begin{enumerate}
    \item Analisar as técnicas atuais de mineração de Bitcoin, destacando as vantagens e desvantagens de cada abordagem.
    \item Identificar técnicas e métodos para implementação do algoritmo SHA-256 em FPGA.
    \item Alcançar eficiência energética melhor que em soluções baseadas em CPU e GPU.
    \item Disponibilizar um comparativo com métodos tradicionais de mineração, ASIC\footnote{Na área específica de mineração de criptomoedas, os profissionais desta área chamam equipamentos baseados em ASICs de apenas ASIC.}, CPU e GPU, considerando o \textit{hashrate} final.
\end{enumerate}

O trabalho buscará demonstrar a viabilidade técnica e econômica da utilização do FPGA para a mineração de Bitcoin, contribuindo para a otimização do processo de mineração em termos de desempenho e eficiência energética. A abordagem detalhada em cada objetivo específico deverá garantir uma compreensão profunda dos desafios e soluções possíveis, estabelecendo um quadro claro para futuras pesquisas e desenvolvimentos na área.

%%%%%%%%%%%%%%%%%%%%%%%%%%%%%%%%%%%%%%%%%%%%%%%%%%%%%%%%%%%%%%%%%%%%%%%%%%%%%%%%%%%%%%%%%%
%%%%%%%%%%%%%%%%%%%%%%%%%%%%%    PLANO DE TRABALHO    %%%%%%%%%%%%%%%%%%%%%%%%%%%%%%%%%%%%
%%%%%%%%%%%%%%%%%%%%%%%%%%%%%%%%%%%%%%%%%%%%%%%%%%%%%%%%%%%%%%%%%%%%%%%%%%%%%%%%%%%%%%%%%%
\begin{comment}
\section{PLANO DE TRABALHO}

Este plano de trabalho apresenta as etapas que serão seguidas para atingir os objetivos propostos neste projeto utilizando FPGA e conceitos de HLS. Cada etapa no plano de trabalho, é detalhada com suas respectivas atividades, demonstrando o caminho a ser seguido para construir a solução proposta.

\begin{enumerate}
    \item Levantamento teórico: permitirá identificar e analisar as principais obras e autores que tratam problemas semelhantes.
        \begin{enumerate}
            \item[a.] \underline{Pesquisa bibliográfica:} verificação e levantamento teórico relacionados à pesquisa;
            \item[b.] \underline{Análise dos materiais:} análise dos conteúdos levantados sobre FPGA e HLS;
        \end{enumerate}
    \item Implementação do Algoritmo SHA-256 em C/C++: permitirá desenvolver a base do algoritmo de mineração.
    \begin{enumerate}
        \item[a.] \underline{Desenvolvimento:} codificação do algoritmo SHA-256 em C/C++;
        \item[b.] \underline{Testes de unidade:} realização de testes de unidade para garantir a correta implementação do algoritmo;
    \end{enumerate}
    \item Conversão para Hardware com HLS: permitirá a conversão do código de software para uma descrição de hardware.
    \begin{enumerate}
        \item[a.] \underline{Conversão:} utilização do HLS para converter o código C/C++ em VHDL;
        \item[b.] \underline{Otimização:} aplicação de técnicas de otimização para melhorar o desempenho e a eficiência energética do design;
    \end{enumerate}
    \item Síntese e Implementação no FPGA: permitirá a programação do FPGA com o design desenvolvido.
    \begin{enumerate}
        \item[a.] \underline{Síntese:} utilização do Intel Quartus Prime para a síntese do design em VHDL;
        \item[b.] \underline{Implementação:} programação do FPGA com o design sintetizado;
    \end{enumerate}
    \item Simulação e Verificação: permitirá garantir que o design funcione corretamente antes da implementação física.
    \begin{enumerate}
        \item[a.] \underline{Simulação:} utilização de programa de simulação para o teste do design sintetizado;
        \item[b.] \underline{Verificação funcional:} verificação da funcionalidade e correção de possíveis erros;
    \end{enumerate}
    \item Avaliação de Desempenho e Eficiência Energética: permitirá avaliar a eficiência da solução desenvolvida.
    \begin{enumerate}
        \item[a.] \underline{Benchmarking:} execução de testes de benchmark para avaliar o \textit{hashrate} do sistema;
        \item[b.] \underline{Análise de Consumo:} monitoramento do consumo de energia durante os testes;
        \item[c.] \underline{Eficiência Energética:} cálculo da eficiência energética (Watt/TH) e comparação com outras tecnologias de mineração (CPU, GPU, ASIC);
    \end{enumerate}
    \item Monografia: permitirá a escrita do texto da monografia.
    \begin{enumerate}
        \item[a.] \underline{Monografia:} escrita do texto da monografia;
    \end{enumerate}
\end{enumerate}

\end{comment}


 \clearpage
\chapter{Fundamentação Teórica}\label{Sec:Fundamentacao_Teorica}

Neste capítulo, são apresentados os conceitos e definições relevantes para o embasamento teórico dos aspectos necessários para o desenvolvimento da solução proposta neste trabalho. Sendo apresentados também os trabalhos relacionados, que demonstram pontos de similaridade com a presente monografia.

\section{Criptomoedas} 
Criptomoedas são ativos digitais que utilizam criptografia para garantir a segurança, integridade e verificabilidade das transações. Embora muitas criptomoedas, como o Bitcoin, se destaquem pela ausência de uma entidade central controladora e operem em redes descentralizadas, algumas criptomoedas possuem elementos de controle centralizado, seja por desenvolvedores, empresas ou consórcios. As transações em redes descentralizadas são validadas por um sistema de consenso distribuído, enquanto em modelos mais centralizados, podem depender de intermediários específicos para registro e validação \cite{StLouisFed2018}. 


As criptomoedas têm diversas funcionalidades além de serem apenas uma forma de pagamento. Elas também podem ser usadas como ativos de investimento devido à sua alta volatilidade, como meio de transferir riqueza de forma anônima, e até para financiar startups, por meio de ICOs \textit{(Initial Coin Offerings)} \cite{Cryptoeconomics2019}. 

\subsection{Bitcoin}
O Bitcoin é a primeira e mais conhecida criptomoeda, criada em 2008 por Satoshi Nakamoto. Seu objetivo era fornecer uma forma de dinheiro eletrônico descentralizado, permitindo que transações pudessem ser realizadas diretamente entre pessoas, sem a necessidade de intermediários como bancos ou instituições financeiras. O sistema é suportado por um livro-razão público e imutável, conhecido como \textit{blockchain}, que registra todas as transações seguramente \cite{nakamoto2008bitcoin}.

O funcionamento do Bitcoin é baseado na tecnologia \textit{blockchain}, uma estrutura de dados que atua como um livro-razão distribuído e imutável, que registra todas as transações de maneira pública e auditável. A \textit{blockchain} é mantida por uma rede global de computadores, conhecidos como nós (ou \textit{nodes}), que trabalham em conjunto para validar e armazenar as transações em blocos, usando algoritmos criptográficos, garantindo a mesma informação descentralizada e segurança nos dados \cite{nakamoto2008bitcoin}.

Cada transação realizada no sistema Bitcoin é agrupada em um bloco, que contém diversas informações, incluindo a assinatura digital do remetente, o endereço do destinatário e o valor transferido, conforme demonstra na \autoref{fig:blockchain_verify}. Esses blocos são encadeados em uma sequência cronológica e linear por meio de um algoritmo criptográfico, formando uma cadeia de blocos (\textit{blockchain}). Cada bloco contém a \textit{hash} (ou impressão digital criptográfica) do bloco anterior, o que garante que a ordem e a integridade das transações não possam ser alteradas sem invalidar toda a cadeia subsequente \cite{Soltani2022}.

\begin{figure}[H]
    \centering
    \caption{Verificação por uma assinatura digital por \textit{hash} anterior.}
    \includegraphics[width=1\linewidth]{2_fundamentacao/verificacao_blockchain.png}
    \caption*{Fonte: Adaptado de \cite{nakamoto2008bitcoin}}
    \label{fig:blockchain_verify}
\end{figure}

O processo que torna o Bitcoin seguro e resistente a fraudes é chamado de mineração. A mineração é essencial para validar as transações, inserir novos blocos na \textit{blockchain} e introduzir novos bitcoins no sistema. Esse processo utiliza o algoritmo de prova de trabalho (do inglês Proof of Work — PoW), que envolve a resolução de complexos problemas matemáticos pelos mineradores. Para adicionar um novo bloco à \textit{blockchain}, os mineradores competem para encontrar uma solução válida para o problema, usando grande capacidade computacional. O primeiro minerador a encontrar a solução válida é recompensado com novos bitcoins, servindo como incentivo econômico para continuar validando as transações \cite{Ibanez2023}. 

A mineração desempenha um papel vital na segurança da rede Bitcoin, ao tornar o sistema altamente resistente a ataques. Alterar qualquer bloco anterior na \textit{blockchain} exigiria que um atacante refizesse todo o trabalho computacional necessário para resolver os problemas dos blocos subsequentes, o que, na prática, seria extremamente caro e difícil de realizar, devido ao crescente poder computacional da rede \cite{antonopoulos2017mastering}. 

Uma das inovações mais importantes do Bitcoin é a resolução do problema do “gasto duplo”, que impede que uma mesma moeda digital seja utilizada mais de uma vez. Isso é possível graças à verificação realizada pelos mineradores, que competem para resolver complexos problemas matemáticos a fim de adicionar novos blocos à \textit{blockchain} e, em troca, são recompensados com novos bitcoins \cite{Kou2022}.

Além disso, o processo de mineração regula a emissão de novos bitcoins, sendo o Bitcoin projetado com um limite máximo de 21 milhões de unidades. Ainda, dentre as regras de emissão limitada de bitcoins, a cada quatro anos, ocorre um evento conhecido como \textit{halving}, onde a recompensa por mineração é reduzida pela metade, garantindo uma taxa de emissão controlada e previsível \cite{ulrich2014bitcoin}.  O Bitcoin também se diferencia por características como sua divisibilidade, onde a menor fração é chamada de “\textit{satoshi}”, e sua portabilidade digital, permitindo transações globais instantâneas e seguras. Com o passar do tempo, o Bitcoin ampliou suas funcionalidades, sendo utilizado tanto como reserva de valor quanto como meio de pagamento, sendo aceito por empresas e até por países, como El Salvador \cite{ElSalvador2021}. 

\subsection{Blockchain}
A \textit{blockchain} é uma tecnologia de livro-razão distribuído que organiza dados em blocos criptografados e cronologicamente conectados, formando uma cadeia imutável de informações. Cada transação na rede é verificada por nós voluntários e descentralizados, que garantem a validade das informações antes de adicioná-las à cadeia. Esse processo elimina a necessidade de intermediários, proporcionando segurança e transparência. Um bloco contém o \textit{hash} do bloco anterior e dados da transação, garantindo que qualquer tentativa de alteração seja imediatamente detectada. Uma característica fundamental da \textit{blockchain} é a sua descentralização, onde o controle não está em uma única entidade, mas distribuído entre todos os participantes da rede \cite{Bencic2018}.

Além disso, as \textit{blockchains} podem ser públicas (como usadas para as criptomoedas como Bitcoin, Litecoin, Ethereum, Monero), onde qualquer pessoa pode participar, ou privadas, onde apenas usuários autorizados têm acesso (empresas, bancos, \textit{datacenters}). A tecnologia tem sido amplamente aplicada não apenas em criptomoedas, mas também em setores como logística, saúde e finanças \cite{swan2015blockchain}.

Por fim, a verificação das transações em uma \textit{blockchain} é feita por meio de mecanismos de consenso, como a Prova de Trabalho (PoW — Proof of Work) ou Prova de Participação (PoS — Proof of Stake), que determinam como os nós concordam sobre o estado atual deste livro-razão \cite{bashir2017mastering}.

\subsection{Prova de Trabalho (PoW)}
O \textit{Hashcash} foi um sistema de prova de trabalho desenvolvido por Adam Back em 1997 \cite{Back1997}, inicialmente projetado para combater o spam em e-mails. O algoritmo exigia que o remetente de um e-mail resolvesse um desafio computacional para gerar um cabeçalho de e-mail único, o que criava um custo computacional mínimo para cada mensagem enviada. Esse custo seria insignificante para usuários legítimos, mas se tornaria inviável para \textit{spammers} que enviam milhares de e-mails.

No contexto das criptomoedas, o \textit{Hashcash} pode ser visto como um precursor do bitcoin, ao introduzir o conceito de usar poder computacional para resolver problemas criptográficos como um meio de verificação, algo essencial para o sistema de prova de trabalho utilizado no Bitcoin. No entanto, \textit{Hashcash} em si não era uma criptomoeda, mas o mecanismo que ele introduziu acabou sendo adaptado por Satoshi Nakamoto no protocolo Bitcoin, para validar transações e garantir a segurança da rede \textit{blockchain} \cite{nakamoto2008bitcoin}.

\subsection{Dificuldade da Rede}
Para aprofundar a compreensão sobre a dificuldade da rede Bitcoin, mais conhecida na área como \textit{diff}, uma fonte para esta definição é o próprio \textit{whitepaper} do Bitcoin, escrito por \cite{nakamoto2008bitcoin}, que apresenta os fundamentos do protocolo e menciona o ajuste de dificuldade para manter o intervalo de geração de blocos constante em aproximadamente 10 minutos. Seu principal objetivo é garantir que novos blocos sejam adicionados neste intervalo de tempo, independentemente do poder computacional total disponível na rede.

A dificuldade é ajustada automaticamente a cada 2016 blocos, correspondendo a cerca de duas semanas, considerando o intervalo médio de 10 minutos por bloco. Durante esse ajuste, a dificuldade é recalculada com base no tempo gasto para minerar os últimos 2016 blocos. Se o tempo total foi menor que 14 dias, a dificuldade aumenta, tornando o processo de mineração mais desafiador. Caso contrário, a dificuldade diminui, facilitando a mineração. Esse mecanismo é essencial para manter a estabilidade e previsibilidade da emissão de novos bitcoins, além de assegurar a segurança da rede contra possíveis ataques.

\newpage
\begin{lstlisting}[language=C++, caption={Exemplo de código para o cálculo da dificuldade com ajuste de bits}, label={lst:difficulty_code}]
#include <iostream>
#include <cmath>

inline float fast_log(float val){
   int * const exp_ptr = reinterpret_cast <int *>(&val);
   int x = *exp_ptr;
   const int log_2 = ((x >> 23) & 255) - 128;
   x &= ~(255 << 23);
   x += 127 << 23;
   *exp_ptr = x;

   val = ((-1.0f/3) * val + 2) * val - 2.0f/3;
   return ((val + log_2) * 0.69314718f);
} 

float difficulty(unsigned int bits){
    static double max_body = fast_log(0x00ffff), scaland = fast_log(256);
    return exp(max_body - fast_log(bits & 0x00ffffff) + scaland * (0x1d - ((bits & 0xff000000) >> 24)));
}

int main(){
    std::cout << difficulty(0x1b0404cb) << std::endl;
    return 0;
}
\end{lstlisting}

Para simplificar o entendimento do cálculo da dificuldade, podemos descrever basicamente pela fórmula:

\begin{equation}
    \text{Dificuldade} = \text{Dificuldade\_Anterior} \times \frac{\text{Tempo\_Real}}{\text{Tempo\_Esperado}}
\end{equation}
\addEquacao{Dificuldade atual}

Onde:

\begin{itemize}
    \item \textbf{Tempo\_Real}: é o tempo total levado para minerar os últimos 2016 blocos.
    \item \textbf{Tempo\_Esperado}: é o tempo ideal, equivalente a \(2016 \times 10\) minutos.
\end{itemize}

Essa métrica é representada como um número escalar que define o alvo de dificuldade, um valor numérico que os mineradores precisam atingir ou superar ao gerar o \textit{hash} do bloco. O alvo é inversamente proporcional à dificuldade: quanto maior a dificuldade, menor será o alvo, tornando mais desafiador encontrar um \textit{hash} válido.

A dificuldade também desempenha um papel crítico na segurança da rede. Ao tornar o processo de mineração mais difícil quando há um aumento no poder computacional, o sistema se protege contra tentativas de ataques coordenados, como o ataque de 51\%. Além disso, o ajuste dinâmico da dificuldade garante que o sistema permaneça funcional e eficiente, mesmo com mudanças significativas no número de mineradores ativos.

\subsection{Bifurcação da Rede}
A dificuldade de mineração na rede Bitcoin não apenas regula o tempo médio para a mineração de blocos, mas também desempenha um papel essencial na prevenção de bifurcações indesejadas na \textit{blockchain}, mais conhecido na área como \textit{fork}. Como descrito por \cite{bouabdo2022evolutionary}, a probabilidade de bifurcação pode ser modelada como:

\begin{equation}
    1 - \prod_{i=1}^{\lfloor d_{\text{evol-rand}} \rfloor} e^{-shd \cdot \lambda_{\text{mine}} \cdot \psi^i} \leq P(\text{forking}) \leq 1 - \prod_{i=1}^{\lceil d_{\text{evol-rand}} \rceil} e^{-shd \cdot \lambda_{\text{mine}} \cdot \psi^i},
\end{equation}
\addEquacao{Probabilidade da bifurcação}

onde:

\begin{itemize}
    \item \( \psi \) representa a influência das conexões entre os nós, computada como uma combinação de somatórios e distribuições de Poisson.
    \item \( shd \) é o atraso de propagação entre dois mineradores.
    \item \( \lambda_{\text{mine}} \) é a probabilidade de um minerador gerar um bloco dentro de um período de tempo, dada por:
\end{itemize}

\begin{equation}
\lambda_{\text{mine}} = \frac{\text{computational-speed}}{\text{mining-difficulty}}.
\end{equation}
\addEquacao{Probabilidade de gerar um bloco}

A partir dessa relação, a dificuldade mínima necessária para manter a probabilidade de bifurcação abaixo de um limite aceitável pode ser expressa como:

\begin{equation}
\text{mining-diff} \geq -\frac{\text{computational-speed} \cdot \sum_{j=1}^{\lceil d_{\text{evol-rand}} \rceil} shd \cdot \psi^j}{\ln(1-\text{threshold})}.
\end{equation}
\addEquacao{Dificuldade mínima aceitável}



\section{Algoritmos de criptografia aplicada a mineração}
A criptografia é crucial para a proteção e integridade dos sistemas de \textit{blockchain} do qual foi projetado, constituindo-se em um dos alicerces fundamentais da tecnologia de criptomoedas. A utilização de algoritmos criptográficos assegura a segurança e a imutabilidade das transações realizadas em uma \textit{blockchain}. 

\begin{quote}
    \textit{A prova-de-trabalho envolve a procura por um valor que quando calculado o hash,
tal como utilizando o SHA-256, o mesmo comece com um número de bits zero. A média do
trabalho requerida é exponencial ao número de bits zero requeridos e pode ser verificada por
meio da execução de um único hash.} \cite{nakamoto2008bitcoin}
\end{quote}

Esses algoritmos têm a função de validar as transações e também de adicionar novos blocos à rede. O algoritmo SHA-256 é o mais comumente usado na mineração de forma geral, sendo o escolhido pelo Bitcoin e um dos mais relevantes no processo de PoW \cite{FrontiersBlockchain2022}. Já na criptomoeda Ethereum, era minerada pelo seu algoritmo chamado \textit{Ethash} (e suas variantes, como \textit{Etchash}), porém, hoje em dia, essa criptomoeda tornou-se prova de participação (PoS), não sendo mais de força bruta (PoW) como o SHA-256 \cite{coinbase_ethereum}.

\subsection{SHA-256}
O algoritmo SHA-256 (Secure Hash Algorithm 256 bits) é um dos mais empregados em sistemas criptográficos contemporâneos, particularmente na mineração de Bitcoin. Ele foi criado pela Agência Nacional de Segurança (do inglês National Security Agency — NSA) e integra a família de algoritmos SHA-2. O SHA-256 transforma qualquer entrada em uma saída de 256 bits, também chamada de \textit{hash}, como apresentado na \autoref{fig:sha-256_funct}. Na mineração de Bitcoin, esse \textit{hash} permite validar as transações a serem registradas naquele bloco e conferir se houve blocos anteriores, se mantiveram íntegros (\textit{hashes} imutáveis) e assegurar a segurança das informações armazenadas no bloco \cite{Crosby2016}. 

\begin{figure}[H]
    \centering
    \caption{Diagrama de blocos do algoritmo SHA-256.}
    \caption*{(a) Fluxo de execução do SHA-256, incluindo a etapa de pré-processamento, programação da mensagem, que gera 64 palavras de 32 bits, e 64 estágios de compressão.}
    \caption*{(b) Diagrama detalhado de dois estágios de compressão de 256 bits de largura.}
    \label{fig:sha-256_funct}
    \includegraphics[width=1\linewidth]{2_fundamentacao/sha-256_funct.png}
    \caption*{Fonte: Adaptado de \cite{tches10955}.}
    \label{fig:sha-256_funct}
\end{figure}

O SHA-256 opera por meio de operações de \textit{hashing}, que são determinísticas e resistem a colisões. Isso implica que, dada uma entrada, o algoritmo sempre gerará a mesma saída, mas mesmo uma mínima mudança na entrada levará a um resultado completamente distinto. Essa propriedade faz do SHA-256 assegurar a integridade dos dados nas \textit{blockchains}, pois qualquer modificação na informação de um bloco invalidaria todos os blocos subsequentes \cite{Haber1991}. 

Por consequência, o SHA-256 tem um papel fundamental no procedimento de mineração. Os mineradores precisam encontrar um \textit{hash} que cumpra determinadas condições pré-estabelecidas para adicionar um novo bloco à \textit{blockchain}, demandando um grande poder de processamento computacional. O minerador que primeiro identificar um \textit{hash} válido recebe novos bitcoins, assegurando que o processo seja economicamente estimulado. Isso faz do SHA-256 um componente fundamental para a segurança e descentralização do sistema \cite{Crosby2016}. 

O algoritmo SHA-256 é bastante conhecido pela capacidade de resistir a ataques. Até agora, não foram identificadas vulnerabilidades práticas que coloquem em risco sua segurança, tornando-o uma opção segura para sistemas vitais como a \textit{blockchain}. Contudo, uma das objeções ao uso do SHA-256 no Bitcoin é o elevado gasto energético necessário para solucionar as questões criptográficas, o que tem gerado debates acerca de opções mais eficazes, como o PoS \cite{FrontiersBlockchain2022}. 

Em suma, o algoritmo SHA-256 é um componente crucial no processo de mineração de Bitcoin. Ele não só assegura a integridade e a segurança da \textit{blockchain}, como também oferece um sistema equitativo de pagamento aos mineradores. Sua eficácia e resistência a ataques fazem dele uma opção perfeita para sistemas descentralizados, mesmo que o impacto ambiental do processo de mineração seja um assunto em constante discussão \cite{Haber1991} \cite{Crosby2016}. 

\subsection*{Função Hash}
O algoritmo SHA-256 funciona por meio de um processo iterativo, dividindo a entrada em blocos de 512 bits. Cada bloco é processado em 64 rodadas de compressão, que utilizam uma combinação de funções lógicas, deslocamentos e operações aritméticas, como demonstrado na \autoref{fig:sha-256-diagram}. 

\begin{figure}[H]
    \centering
    \caption{Diagrama da função de compressão da família SHA-2}
    \includegraphics[width=0.85\linewidth]{2_fundamentacao/diagram_sha-256.png}
    \caption*{Fonte: Adaptado de \cite{sim2012submission46}}
    \label{fig:sha-256-diagram}
\end{figure}

A estrutura básica das operações é a seguinte:

\begin{itemize}
    \item \textbf{Divisão dos Blocos:} Cada bloco de 512 bits é dividido em palavras de 32 bits (\(W_0, W_1, \dots, W_{15}\)). Outras palavras (\(W_{16}, \dots, W_{63}\)) são geradas dinamicamente utilizando deslocamentos circulares e \textit{XOR}.

    \item \textbf{Estados Internos:} O SHA-256 utiliza 8 registradores principais (\(A, B, C, D, E, F, G, H\)) para armazenar os estados intermediários, quos quais são atualizados a cada rodada.

    \item \textbf{Funções Lógicas:}
    \begin{itemize}
        \item \(Ch(E, F, G) = (E \land F) \oplus (\neg E \land G)\): Seleção condicional baseada em \(E\).
        \item \(Ma(A, B, C) = (A \land B) \oplus (A \land C) \oplus (B \land C)\): Seleção majoritária.
        \item Deslocamentos circulares e shifts (\(\Sigma_0\), \(\Sigma_1\)) para misturar os bits dos estados internos.
    \end{itemize}

    \item \textbf{Atualização dos Registradores:} Em cada rodada, os valores de \(A\) a \(H\) são atualizados com base em combinações das funções \(Ch\), \(Ma\), deslocamentos e constantes específicas (\(K_t\)).
\end{itemize}


\subsection*{Pseudocódigo}

Como exemplo, o pseudocódigo apresentado na Listagem \ref{lst:sha256_transform} será utilizado como base para a configuração do algoritmo SHA-256 no FPGA. Esse algoritmo, descrito em linguagem C, será adaptado para o desenvolvimento em HLS, possibilitando a conversão para descrição de hardware em nível de registro RTL. A utilização dessa abordagem permite explorar o desempenho do FPGA.

\begin{lstlisting}[language=C, caption={Função SHA256\_transform implementada em C.}, label={lst:sha256_transform}]

void sha256_transform(unsigned int data[8], unsigned int data[16], unsigned int ss[8]) {
    WORD reg[8], i, j, t1, t2, m[64];
    int n = 0;

    label0: for (i = 0; i < 16; i++) {
        m[i] = data[i];
    }
    
    label1: for (i = 16; i < 64; i++) {
        m[i] = SIG1(m[i - 2]) + m[i - 7] + SIG0(m[i - 15]) + m[i - 16];
    }

    label2: for (n = 0; n < 8; n++) {
        reg[n] = data[n];
    }
}
\end{lstlisting}
Fonte: Adaptado de \cite{kammoun2020}.

\section{Aceleração em Hardware}
A aceleração em hardware é um princípio aplicado para melhorar a eficiência e a desempenho de sistemas de computação, atribuindo funções específicas a componentes especializados, tais como ASICs, FPGAs e GPUs. Em contraste com o processamento convencional em Unidade de Processamento Central (do inglês Central Processing Unit — CPU), projetados para realizar uma variedade de tarefas, a aceleração em hardware emprega componentes específicos para operações específicas, como a mineração de criptomoedas e outras atividades que necessitam de desempenho elevado no tempo para computação \cite{WevolverASICvsFPGA2024}.

No contexto de mineração de criptomoedas, a demanda por aceleração em hardware na mineração de criptomoedas, particularmente em sistemas como o Bitcoin, surge devido ao aumento da dificuldade da rede. Conforme a complexidade da mineração cresce, o poder de processamento bruto requerido para solucionar a chave do bloco seguinte aumenta \cite{Treiblmaier2021}.

A especialização é uma das principais vantagens da aceleração em hardware. Embora as CPUs sejam concebidas para executar diversas operações, os hardwares voltados para a  aceleração em hardware, tais como FPGAs e ASICs, são desenvolvidos especificamente para executar um número limitado de operações de maneira mais rápida e eficiente. Em relação à mineração de criptomoedas, isso implica que algoritmos como o SHA-256 podem ser executados consideravelmente mais eficiente em um FPGA ou ASIC do que numa CPU convencional \cite{AnalyticsFPGA2024}. 

Ademais, a aceleração por hardware proporciona adaptabilidade no caso dos FPGAs, que podem ser reprogramados para diversas funções conforme a necessidade. Apesar de os FPGAs não serem tão otimizados em questão de tecnologia de circuito quanto os ASICs, eles oferecem uma opção viável ao procurar um equilíbrio entre flexibilidade e desempenho em métricas como tempo de desenvolvimento, tempo de processamento e consumo de energia. Por outro lado, os ASICs são altamente otimizados para uma tarefa específica, como a mineração de criptomoedas, podendo executar cálculos de \textit{hashing} eficientemente \cite{Treiblmaier2021}. 

A aplicação de aceleração em hardware também pode levar a uma eficiência energética superior. No processo de mineração de criptomoedas, o consumo de energia é uma questão relevante. Ao empregar componentes especializados para executar as tarefas de mineração, conseguimos diminuir o uso de energia em relação ao uso de CPUs ou GPUs, tornando a operação mais sustentável e economicamente viável a longo prazo \cite{Jablczynska2023}. 


\subsection{Instruções Dedicadas}

As instruções dedicadas referem-se ao conjunto específico de operações que um núcleo de processamento especializado pode executar. Essas instruções são otimizadas para realizar tarefas específicas, como operações matemáticas complexas, comuns no processo de mineração de criptomoedas. Por exemplo, na mineração de Bitcoin, as instruções focam em operações de \textit{hashing}, comparações e cálculos binários, acelerando o processo de mineração quando implementadas diretamente no hardware.

Em plataformas como FPGAs e ASICs, essas instruções são incorporadas diretamente nos circuitos, proporcionando um aumento significativo na eficiência. Isso ocorre porque o sistema não precisa interpretar uma ampla gama de instruções de uso geral, resultando em economia de recursos computacionais e melhoria na eficiência energética, fatores cruciais na mineração de criptomoedas \cite{8404574, 7967745}.


\subsection{Algoritmo dedicado}
Um algoritmo dedicado é desenvolvido e aprimorado para funcionar apenas em hardware específico, como FPGAs e ASICs, em vez de recorrer a implementações em software que seriam menos eficientes. Esses algoritmos, como o SHA-256 empregado na mineração de Bitcoin, são organizados de maneira a maximizar o potencial dos núcleos de processamento dedicados. A utilização de algoritmos específicos para hardware leva a tempos de execução reduzidos e a uma administração mais eficiente dos recursos de computação disponíveis \cite{sha256_fpga_optimization, sha256_asic_energy}.

\section{Tecnologia de Circuito e de Processador}
As tecnologias de circuitos integrados desempenham um papel essencial na mineração de criptomoedas, principalmente com o uso de dispositivos especializados, como ASICs e FPGAs. Ambas as tecnologias oferecem vantagens distintas em termos de desempenho, eficiência e flexibilidade, dependendo da aplicação e do nível de otimização necessário para o processo de mineração \cite{hennessy2017computer}.

\subsection{Tecnologia de Circuito}

\subsubsection{Totalmente Customizável (Full-custom)}
Na tecnologia full-custom, todas as camadas do circuito, incluindo transistores, portas lógicas e conexões, são personalizadas. Essa abordagem visa otimizar o desempenho do circuito final em diversas métricas, como eficiência energética e alta frequência de operação. Aspectos como o tamanho do canal, a dimensão e a posição dos transistores, bem como o roteamento das interconexões, são configurados de forma específica. Após a finalização do projeto, as máscaras são geradas e enviadas para a fabricação do circuito integrado (CI). Embora essa técnica tenha um custo NRE elevado e um tempo de produção mais longo, ela proporciona superioridade em métricas de desempenho quando comparada a outras tecnologias \cite{vahid2001embedded}.

\subsubsection{Semi Customizável (ASIC)}
Os ASICs são circuitos integrados projetados especificamente para executar uma única função de forma extremamente eficiente. Na mineração de criptomoedas, como o Bitcoin, os ASICs são otimizados para o algoritmo SHA-256, tornando-os a escolha mais eficiente em termos de taxa de \textit{hash} e consumo de energia. Por serem projetados exclusivamente para essa função, os ASICs superam amplamente outros tipos de hardware, como GPUs e FPGAs, em desempenho e economia de energia \cite{hennessy2017computer}.

Uma das principais vantagens dos ASICs é a sua eficiência energética. Por serem otimizados para uma tarefa específica, eles consomem muito menos energia em comparação com processadores de propósito geral. No entanto, essa especialização também é uma desvantagem, pois os ASICs não podem ser reconfigurados para realizar outras funções, tornando-os inúteis se a criptomoeda ou o algoritmo de mineração mudar \cite{barr2006asic}.

Além disso, os ASICs modernos avançam em termos de miniaturização de chips, com tecnologias de semicondutores mais recentes permitindo a criação de chips menores e mais eficientes, melhorando ainda mais o desempenho e o resfriamento desses dispositivos \cite{hennessy2017computer}.

\subsubsection{Lógica Programável (FPGA)}
Os FPGAs são dispositivos semicondutores que podem ser programados após a fabricação para realizar uma ampla gama de funções. Ao contrário dos ASICs, projetados para uma tarefa específica, os FPGAs podem ser reprogramados conforme necessário, o que os torna uma opção mais flexível para mineradores que desejam adaptar suas operações a diferentes criptomoedas ou algoritmos \cite{maxfield2004field}. Embora os FPGAs não sejam tão eficientes quanto os ASICs em termos de taxa de \textit{hash}, eles oferecem uma flexibilidade de desenvolvimento e implementação que os ASICs não conseguem igualar. Isso é particularmente útil em cenários onde a criptomoeda minerada pode mudar ou quando novos algoritmos de mineração são introduzidos. Outra vantagem dos FPGAs é sua capacidade de realizar outras tarefas além da mineração, o que os torna uma opção mais versátil para diversas aplicações.

Os elementos que compõe um FPGA podem ser descritos conforme \citeonline{harris2013projeto}:

\begin{quote}
    \textit{As FPGAs são construídas como uma matriz de elementos lógicos configuráveis (LE — logic elements), também conhecidos como configurable logic blocks (CLB). Cada LE pode ser configurado para desempenhar funções combinatórias ou sequenciais. A \autoref{fig:fpga_layout} apresenta um diagrama de blocos geral de uma FPGA. Os LE estão rodeados por elementos de entrada/saída (IOE — input output elements) para a interface com o mundo exterior. Os IOE ligam as entradas e as saídas dos LE aos pinos de empacotamento de chip. OS LE podem-se ligar-se a outros LE e IOE através de canais de encaminhamento programáveis.}
\end{quote} 

\begin{figure}[H]
    \centering
    \caption{Layout genérico de uma FPGA.}
    \includegraphics[width=0.6\linewidth]{2_fundamentacao/fpga_layout.png}
    \label{fig:fpga_layout}
    \caption*{Fonte: \citeonline{harris2013projeto}.}
\end{figure}


\subsection{Tecnologia de processador}
Os processadores desempenham um papel central na computação, com diferentes arquiteturas projetadas para atender a necessidades específicas. Para o contexto de mineração de criptomoedas e tarefas que demandam aceleração em hardware, destacam-se três tipos principais de tecnologia de processador: General Purpose Processors (GPP), SPP e Application-Specific Instruction Processors (ASIP). Cada um oferece vantagens e limitações conforme o tipo de tarefa e o nível de otimização necessário \cite{vahid2001embedded}. Na \autoref{fig:gpp_asip_spp} temos um comparativo sobre cada tecnologia de processador.

\begin{figure}[H]
    \centering
    \caption{A independência das tecnologias de processador e CI: qualquer tecnologia de processador pode ser mapeada para qualquer tecnologia de CI.}
    \includegraphics[width=1\linewidth]{2_fundamentacao/gpp_asip_spp.png}
    \caption*{Fonte: Adaptado de \citeonline{vahid2001embedded}.}
    \label{fig:gpp_asip_spp}
\end{figure}

\subsubsection{GPP}
Os GPPs são processadores projetados para realizar uma ampla gama de tarefas, sendo flexíveis e aplicáveis em diversas aplicações. Eles são comumente usados em computadores pessoais e servidores, onde a versatilidade é importante. No entanto, para aplicações como mineração de criptomoedas, onde o desempenho e a eficiência energética são críticos, os GPPs tendem a ser menos eficientes em comparação com SPPs \cite{vahid2001embedded}. Em geral, os GPPs não são a primeira escolha para mineração devido ao seu consumo de energia elevado e menor capacidade de processamento por Watt. 

\subsubsection{SPP}
Os SPPs são processadores projetados para executar um conjunto limitado de tarefas de maneira eficiente. Eles são otimizados para operações específicas, como a criptografia ou o processamento de dados gráficos, e, por isso, têm desempenho superior em determinadas métricas em suas áreas de especialização \cite{vahid2001embedded}. No contexto da mineração, os SPPs são frequentemente usados em dispositivos como GPUs (Graphic Processing Units), que se tornaram populares para minerar certas criptomoedas antes da dominância dos ASICs. Ao contrário dos GPPs, os SPPs são menos flexíveis, mas oferecem uma melhor relação entre desempenho e consumo de energia para tarefas específicas.

\subsubsection{ASIP}
Os ASIPs são projetados para serem reconfiguráveis, oferecendo um equilíbrio entre flexibilidade e otimização. Eles podem ser ajustados para executar uma variedade de tarefas em um domínio específico, como algoritmos de mineração de criptomoedas. Os ASIPs combinam a eficiência de um processador especializado com a adaptabilidade de um GPP, permitindo que sejam usados em diferentes criptomoedas ou algoritmos com mudanças mínimas no hardware. A principal vantagem dos ASIPs é a sua capacidade de personalização, o que os torna adequados para mineradores que buscam otimizar suas operações sem investir em ASICs, os quais são menos flexíveis \cite{vahid2001embedded}.


\section{Comparação sobre tecnologias de circuito para \\ mineração de criptomoedas}
As tecnologias de circuitos desempenham um papel fundamental na mineração de criptomoedas, oferecendo diferentes características de desempenho, custo e flexibilidade. Neste contexto, destacam-se duas tecnologias principais: FPGA  e ASIC.

Os FPGAs são dispositivos semicondutores programáveis, que permitem uma ampla gama de reconfigurações após a fabricação. Isso os torna altamente versáteis e uma escolha ideal para aplicações onde a flexibilidade é essencial, como na mineração de criptomoedas, onde o algoritmo de mineração pode mudar com o tempo. No entanto, essa flexibilidade tem um custo em termos de desempenho e consumo de energia, já que os FPGAs geralmente não são tão eficientes quanto os ASICs para tarefas específicas.

Por outro lado, os ASICs são projetados para realizar uma função específica de forma extremamente eficiente. A desvantagem é a falta de flexibilidade, já que os ASICs não podem ser reconfigurados para outros algoritmos \cite{vahid2001embedded}. Na mineração de criptomoedas, os ASICs são frequentemente otimizados para o algoritmo de mineração SHA-256 (utilizado pelo Bitcoin), proporcionando uma taxa de hash muito maior com um consumo de energia reduzido em comparação com os FPGAs. %\clearpage
%\chapter{Trabalhos Relacionados}\label{Sec:Trabalhos_Relacionados}

\section{Trabalhos Relacionados}\label{Sec:Trabalhos_Relacionados}

Nesta seção, apresenta-se uma análise dos estudos relacionados, começando pela exposição dos repositórios utilizados, seguida da descrição dos métodos e dos critérios empregados para a seleção desses estudos. Ao final, será realizada uma comparação entre as pesquisas encontradas e o estudo atual.

\subsection{Procedimento de Análise}
Esta subseção descreve os repositórios consultados para a identificação e avaliação dos estudos relevantes. São apresentados, em detalhes, a metodologia aplicada e os critérios definidos para a seleção dos trabalhos analisados.

\subsubsection{Fontes de Pesquisa}

A pesquisa por estudos relacionados foi conduzida através da plataforma CAPES, com acesso fornecido pela UNIVALI, abrangendo diversas bases de dados, incluindo IEEE, arXiv, Science Direct e ACM. Além disso, foram realizadas consultas no Google Scholar para ampliar o alcance da pesquisa e incluir referências acadêmicas adicionais.

%Qual o propósito do trabalho, que tecnologias foram usadas, como foi implementado, resultados obtidos.

\subsubsection{Critérios de seleção}
A seleção dos trabalhos foi baseada em várias características-chave, que incluem:
(\textit{i}) o propósito do trabalho;
(\textit{ii}) quais tecnologias foram usadas;
(\textit{iii}) como foi implementado e
(\textit{iv}) os resultados obtidos. 

O presente trabalho tem como foco principal a implementação do algoritmo SHA-256 em um FPGA, utilizando técnicas avançadas de Síntese de Alto Nível (HLS) para potencializar a eficiência computacional na geração de \textit{hashrate}, essencial para a mineração de Bitcoin. 

Além disso, o trabalho visa avaliar os impactos da implementação do SHA-256 em termos de consumo de energia, desempenho e viabilidade, comparando os resultados obtidos com outras soluções de hardware dedicadas à mineração de criptomoedas.


%%%%%%%%%%%%%%%%%%%%%%%%%%%%%%%%%%%%%%%%
%%%%%%%%%% PRIMEIRO TRABALHO %%%%%%%%%%%
%%%%%%%%%%%%%%%%%%%%%%%%%%%%%%%%%%%%%%%%

\subsection{Acceleration of Frequent Itemset Mining on FPGA Using SDAccel and Vivado HLS}

O estudo realizado por \citeonline{dang2017} aborda a aceleração de algoritmos de conjuntos frequentes (Frequent Itemset Mining — FIM) utilizando FPGAs e técnicas de Síntese de Alto Nível (HLS), especificamente com as ferramentas SDAccel e Vivado HLS. A mineração de conjuntos frequentes é uma técnica importante na descoberta de padrões em grandes bases de dados, porém, é extremamente exigente em termos de processamento, principalmente com o aumento do tamanho dos conjuntos de dados. Neste contexto, o uso de FPGAs se mostra promissor por sua capacidade de acelerar algoritmos computacionalmente intensivos.

Para o desenvolvimento, o algoritmo de expansão de fronteira (Frontier Expansion — FE) foi implementado em um FPGA, utilizando SDAccel e a linguagem C/C++. Os resultados obtidos indicaram que a implementação proposta oferece uma aceleração de até 3,2 vezes em comparação com uma CPU de 6 núcleos, além de apresentar uma eficiência energética superior à de uma GPU. Em testes preliminares com o FPGA XCKU115, a implementação demonstrou desempenho comparável com implementações em HDL e superou o desempenho da GPU, destacando o potencial dos FPGAs como alternativa viável em cenários de mineração de dados.

A \autoref{fig:fluxo_dang}, os ganhos de velocidade são apresentados à medida que várias otimizações são aplicadas à implementação no FPGA. A base de comparação é a implementação \textit{Non-Opt}, que representa o FPGA sem nenhuma otimização aplicada.

Vale destacar que as otimizações (2) a (4) têm como foco principal a melhoria da taxa de transferência, sendo agrupadas com a otimização (1) sob o rótulo Opt1\_to\_4. Esse agrupamento permite uma comparação clara entre configurações parcialmente otimizadas e a configuração totalmente otimizada.

O rótulo All-Opt na figura representa o desempenho alcançado quando todas as otimizações são implementadas simultaneamente, mostrando o potencial máximo de aceleração do FPGA com uma otimização completa.


\begin{figure}[H]
    \centering
    \caption{Técnicas de otimização e desempenho em um FPGA.}
    \includegraphics[width=0.75\linewidth]{trabalhos//img/fluxo_dang.png}
    \caption*{Fonte: \cite{dang2017}.}
    \label{fig:fluxo_dang}
\end{figure}

A \autoref{fig:energy_dang} apresenta o consumo de potência dinâmica de diferentes plataformas de processamento, incluindo CPU, GPU e FPGA, durante a execução do algoritmo FE em dois conjuntos de dados distintos (Accidents e T80I20N0\_3D2000K) com um valor de suporte mínimo de 0,1. Devido à falta de suporte da ferramenta SDAccel para medições de potência diretamente na placa e ao fato de GPU e FPGA estarem em sistemas diferentes, a análise de potência foi realizada medindo o consumo de potência dinâmica. Esse processo envolveu a utilização de um medidor de energia \textit{Watts PRO} para registrar o consumo do sistema em intervalos de um segundo. A potência dinâmica foi calculada subtraindo o consumo de potência do sistema em estado ocioso do consumo durante a execução do algoritmo.

\begin{figure}[H]
    \centering
    \caption{Eficiência energética de GPU e FPGA, normalizada em relação aos resultados da CPU}
    \includegraphics[width=0.75\linewidth]{trabalhos//img/energy_dang.png}
    \caption*{Fonte: \cite{dang2017}. }
    \label{fig:energy_dang}
\end{figure}

Os resultados mostram que as implementações em FPGA com diferentes configurações de unidades computacionais (CU), como 1CU, 2CU e 4CU, apresentam um consumo de energia significativamente menor em comparação com a CPU e a GPU, especialmente quando várias CUs são usadas. No conjunto de dados \textit{Accidents}, a configuração de 6 núcleos de CPU (6CPU) registrou um consumo de 74,1W em comparação com 48,9W na configuração de 4 unidades computacionais (4CU) da FPGA. Já para o conjunto de dados T80I20N0\_3D2000K, a configuração 6CPU apresentou um consumo de 152,3W, enquanto a FPGA com 4CU consumiu apenas 50,9W. 

Esses resultados indicam que, embora a GPU ofereça maior velocidade de processamento, as implementações em FPGA são mais eficientes energeticamente, especialmente quando se utiliza múltiplas CUs. Esse aspecto torna a FPGA uma alternativa atraente em termos de eficiência energética, especialmente em cenários onde o consumo de energia é um fator crítico.


%%%%%%%%%%%%%%%%%%%%%%%%%%%%%%%%%%%%%%%%
%%%%%%%%%%% SEGUNDO TRABALHO %%%%%%%%%%%
%%%%%%%%%%%%%%%%%%%%%%%%%%%%%%%%%%%%%%%%

\subsection{FPGA-based implementation of the SHA-256 hash algorithm}

O estudo realizado por \citeonline{kammoun2020} aborda a implementação do algoritmo de hash SHA-256 em FPGAs, com foco em otimizar o desempenho e reduzir o consumo de energia para aplicações que exigem alta segurança, como assinaturas digitais e criptografia em transações eletrônicas. 

A implementação proposta utiliza a ferramenta Vivado HLS da Xilinx para realizar a Síntese de Alto Nível (HLS), permitindo que o código em C seja convertido em uma descrição de hardware, facilitando o desenvolvimento e melhorando o desempenho. Essa abordagem permite aplicar diretivas de otimização, como \textit{pipeline} e \textit{unroll}, que reduzem a latência e aumentam a taxa de transferência \textit{(throughput)}, aspectos essenciais para a mineração de Bitcoin.

Os resultados obtidos indicam uma redução de 73\% no consumo de energia e um aumento de 17\% na velocidade de execução em comparação com uma implementação puramente em software. Além disso, uma arquitetura otimizada alcançou um incremento de 66\% na taxa de transferência em relação a estudos anteriores, demonstrando a eficácia do uso de FPGA para tarefas intensivas em computação.

O estudo também explora uma arquitetura híbrida SW/HW, combinando o processador ARM Cortex-A9 e o acelerador em FPGA, permitindo que o FPGA seja utilizado exclusivamente para o cálculo do SHA-256, enquanto o processador ARM gerencia outras operações. Esse modelo híbrido proporciona um bom equilíbrio entre flexibilidade e eficiência, confirmando o potencial do FPGA para aplicações de mineração de Bitcoin com eficiência energética e alta taxa de \textit{hash}.

A \autoref{fig:kammoun_execution} apresenta os resultados do tempo de execução do algoritmo SHA-256 em duas abordagens: uma implementação puramente em software (SW) e uma implementação híbrida de software e hardware (SW/HW HLS) utilizando Síntese de Alto Nível (HLS). Observa-se que a abordagem SW/HW HLS proporciona uma redução significativa no tempo de execução em comparação com a solução apenas em software. Essa melhoria em picossegundos destaca a vantagem de utilizar técnicas de HLS para otimizar o desempenho, o que é particularmente relevante em aplicações de alta demanda computacional, como a mineração de criptomoedas.

\begin{figure}[H]
    \centering
    \caption{Resultados dos tempos de execução do algorítimos SHA-256.}
    \includegraphics[width=0.75\linewidth]{trabalhos//img/kammoun_execution.png}
    %\caption{Execution time results of the SHA-256 algorithm 
    \caption*{Fonte: \cite{kammoun2020}.}
    \label{fig:kammoun_execution}
\end{figure}

Enquanto na \autoref{fig:kammoun_energy} exibe o consumo de energia das duas abordagens de implementação do algoritmo SHA-256: a implementação puramente em software (SW) e a implementação híbrida de software e hardware (SW/HW HLS). A abordagem SW/HW HLS demonstra uma considerável redução no consumo energético em relação à solução apenas em software. Esse resultado evidencia a eficiência energética da utilização de FPGAs com HLS, mostrando-se uma alternativa mais sustentável para operações intensivas em processamento, como a mineração de Bitcoin, onde o consumo de energia é um fator decisivo.

\begin{figure}[H]
    \centering
    \caption{Power consumption estimation.}
    \includegraphics[width=0.75\linewidth]{trabalhos//img/kammoun_energy.png}
    \caption*{Fonte: \cite{kammoun2020}.}
    \label{fig:kammoun_energy}
\end{figure}


Os testes experimentais validaram a eficácia da implementação, destacando o FPGA como uma solução viável para melhorar o desempenho e reduzir o consumo de energia em comparação com arquiteturas baseadas apenas em software.


%%%%%%%%%%%%%%%%%%%%%%%%%%%%%%%%%%%%%%%%
%%%%%%%%%%% TERCEIRO TRABALHO %%%%%%%%%%
%%%%%%%%%%%%%%%%%%%%%%%%%%%%%%%%%%%%%%%%

\subsection{The Evolution of Bitcoin Hardware}

O estudo realizado por \citeonline{bedford2017} explora a evolução do hardware de mineração de Bitcoin, desde o uso inicial de CPUs até o desenvolvimento de circuitos integrados específicos para aplicações (ASICs). O autor examina o impacto do crescimento do valor do Bitcoin no aprimoramento da tecnologia de mineração, destacando como a eficiência energética e o desempenho foram melhorados ao longo das gerações de hardware.

Taylor detalha as transições tecnológicas que ocorreram no hardware de mineração, com foco especial nos ASICs, que atualmente dominam a mineração de Bitcoin. Ele descreve a ascensão dos datacenters especializados, conhecidos como “ASIC clouds”, que permitem uma escala de computação em nível planetário, otimizada para reduzir custos e maximizar a eficiência. O autor também aborda como a integração vertical da indústria permitiu que empresas controlassem cada aspecto do processo, desde o design dos chips até a manutenção dos datacenters.

Com o tempo, a dificuldade continuou a subir, levando ao uso de FPGAs, que proporcionaram uma melhoria significativa em desempenho e eficiência energética em relação às GPUs. No entanto, a demanda crescente e a competitividade da mineração resultaram na criação de máquinas baseadas em ASICs, dispositivos projetados especificamente para a mineração de Bitcoin. Os ASICs passaram por avanços rápidos na miniaturização de transistores, com processos de fabricação em diferentes nós de VLSI (Very-Large-Scale Integration), como 130 nm, 110 nm, 65 nm, 55 nm, 28 nm, 22 nm, 20 nm e, finalmente, 16 nm.

A \autoref{fig:bedford_diff} ilustra o aumento exponencial da dificuldade de mineração de Bitcoin ao longo dos anos, desde 2009 até 2018, e mostra como a introdução de novas tecnologias de hardware foi essencial para acompanhar esse crescimento. Inicialmente, a mineração era realizada em CPUs, mas rapidamente se tornou inviável devido ao aumento da dificuldade. Como resposta, as GPUs foram adotadas, oferecendo maior poder de processamento para os cálculos do algoritmo SHA-256.

\begin{figure}[H]
    \centering
    \caption{Nível de dificuldade da mineração ao longo dos anos.}
    \includegraphics[width=0.70\linewidth]{trabalhos//img/bedford_diff.png}
    \caption*{Fonte: \cite{bedford2017}}
    \label{fig:bedford_diff}
\end{figure}

Essas evoluções tecnológicas permitiram que a mineração de Bitcoin acompanhasse o aumento da dificuldade, tornando-se cerca de 850 bilhões de vezes mais difícil do que no início da rede. A figura destaca como cada inovação em hardware foi introduzida em resposta ao aumento da dificuldade, enfatizando a importância da adaptação tecnológica para manter a competitividade e a eficiência energética na mineração.

Outro ponto importante discutido é a viabilidade econômica da mineração, onde os custos de operação e manutenção de \textit{rigs} de mineração são comparados com o retorno financeiro da mineração. Taylor destaca como a dificuldade crescente da mineração exige melhorias contínuas no hardware para manter a lucratividade, forçando os mineradores a substituir equipamentos com frequência para acompanhar a evolução tecnológica.

Esses tópicos são particularmente relevantes para o contexto da mineração de criptomoedas, ao mostrarem a importância da inovação em hardware e da eficiência energética, aspectos centrais para a viabilidade de uma arquitetura baseada em FPGA para a mineração de Bitcoin.







%%%%%%%%%%%%%%%%%%%%%%%%%%%%%%%%%%%%%%%%
%%%%%%%%%%% ANALISE %%%%%%%%%%
%%%%%%%%%%%%%%%%%%%%%%%%%%%%%%%%%%%%%%%%





\subsection{Análise Comparativa}

Nesta seção, são apresentadas as principais características dos estudos discutidos anteriormente. As informações principais de cada trabalho, como propósito, tecnologias utilizadas, implementação e resultados obtidos, são dispostas no \autoref{Q:Trabalhos_Relacionados}. Os trabalhos \citeonline{dang2017}, \citeonline{kammoun2020} e \citeonline{bedford2017} abordam diferentes aspectos da utilização de FPGAs e tecnologias associadas para otimização e eficiência em tarefas intensivas de processamento.

\begin{quadro}[H]
    \centering
    \caption{Análise comparativa dos trabalhos relacionados.}
    \vspace{-6pt}
    \resizebox{\textwidth}{!}{%
    \begin{tabular}{|p{3.0cm}|p{3.5cm}|p{3.5cm}|p{3.5cm}|p{3.5cm}|}
\hline
\rowcolor[HTML]{EFEFEF} 
\textbf{Trabalho} &
  \textbf{Propósito do Trabalho} &
  \textbf{\begin{tabular}[c]{@{}c@{}}Tecnologias Usadas\end{tabular}} &
  \textbf{\begin{tabular}[c]{@{}c@{}}Implementação\end{tabular}} &
  \textbf{\begin{tabular}[c]{@{}c@{}}Resultados Obtidos\end{tabular}} \\ \hline

\textbf{Dang e Skadron (2017)} &
  Acelerar algoritmos de mineração de conjuntos frequentes (Frequent Itemset Mining) em FPGA &
  Xilinx FPGA XCKU115: Lógica Reconfigurável e SPP &
  Implementação do algoritmo de expansão de fronteira (Frontier Expansion - FE) em C/C++ com HLS &
  Aceleração de até 3,2x em relação à CPU, com eficiência energética superior à da GPU \\ \hline

\textbf{Kammoun et al. (2020)} &
  Implementação do algoritmo SHA-256 em FPGA para aplicações de segurança e eficiência energética &
  %Vivado HLS, FPGA, ARM Cortex A9 
  FPGA: Lógica Reconfigurável e SPP; ARM Cortex A9: Full Custon/ASIC e GPP &
  Arquitetura híbrida SW/HW utilizando HLS para SHA-256 com pipeline e unroll &
  Redução de 73\% no consumo de energia e aumento de 17\% na velocidade em relação ao software puro \\ \hline

\textbf{Taylor (2017)} &
  Explorar a evolução do hardware para mineração de Bitcoin, desde CPUs até ASICs &
  CPUs, GPUs, FPGAs, ASICs em diferentes nós de VLSI &
  Estudo comparativo e histórico dos avanços de hardware para mineração &
  Identificação da eficiência dos ASICs em data centers e a necessidade de inovação contínua para acompanhar a dificuldade de mineração \\ \hline
\textbf{Este Trabalho} & Explorar o uso de FPGA para a mineração de Bitcoin utilizando o algoritmo SHA-256 & FPGA: Lógica Reconfigurável. Simulação em IDE & Utilização da arquitetura HW/SW com técnica em HLS & Previsão: aumentar o hashrate em relação à CPU com baixo custo de energia \\\hline
    \end{tabular}%
    }
\label{Q:Trabalhos_Relacionados}
\caption*{Fonte: O autor (2024).}
\end{quadro}


Para uma análise comparativa detalhada, cada trabalho contribui com perspectivas específicas no uso de FPGAs e outras tecnologias de hardware:

\begin{itemize}
     


\item O trabalho de \citeonline{dang2017} foca na aceleração de algoritmos de mineração de conjuntos frequentes utilizando FPGAs, onde foram aplicadas técnicas de HLS com SDAccel e Vivado HLS para melhorar a desempenho de processamento. A implementação do algoritmo FE em FPGA resultou em uma aceleração de até 3,2 vezes em comparação com a CPU, além de oferecer uma eficiência energética superior à da GPU.

\item No estudo de \citeonline{kammoun2020}, a ênfase está na implementação do SHA-256, o algoritmo base da mineração de Bitcoin, com uma arquitetura híbrida SW/HW em FPGA. Utilizando o Vivado HLS e um processador ARM, a abordagem adotada otimizou o uso de pipeline e unroll para aumentar a taxa de transferência e reduzir o consumo de energia, resultando em uma economia de 73\% de energia e um aumento de 17\% na velocidade de execução em relação a uma solução de software puro.

\item Por fim, o trabalho de \citeonline{bedford2017} aborda a evolução dos hardwares de mineração de Bitcoin, descrevendo a transição das CPUs para GPUs, FPGAs e, finalmente, ASICs. Taylor destaca a eficiência dos ASICs em data centers dedicados, os chamados "ASIC clouds", que permitem uma escala massiva de mineração. Esse trabalho é fundamental para compreender a importância da inovação em hardware e eficiência energética, especialmente à medida que a dificuldade da mineração aumenta exponencialmente.
\end{itemize}

Esses estudos reforçam a relevância do uso de FPGAs e outras tecnologias específicas para o contexto da mineração de Bitcoin e outras aplicações intensivas, destacando os avanços em desempenho e eficiência energética proporcionados por implementações dedicadas.









%Fazer um levamento dos pontos positivos e negavos de cada trabalho dentro do contexto do trabalho
%Tabela comparativa com cada coluna sendo um atributado pro seu trabalho

% Trabalho | Tecnologia Circuito Usada | Algoritmos Usados | Métricas Usadas \clearpage
\chapter{Projeto}\label{Sec:Projeto}

Neste capítulo, é apresentado o projeto de implementação do trabalho que será desenvolvido no TCC III. 




%%%%%%%%%%%%%%%%%%%%%%%%%%%%%%%%%%%%%%%%%%%%%%%%%%%%%%%%%%%%%%%%%%%%%

\section{Premissas e Visão Geral do Sistema}\label{sec:premissas}

Neste trabalho, propõe-se a utilização de aceleração em hardware para a mineração de Bitcoin, explorando o algoritmo SHA-256 implementado em FPGA com a utilização de HLS. O objetivo é validar a eficiência e a viabilidade da mineração de criptomoedas em dispositivos de hardware reconfiguráveis.

A \autoref{fig:visao_geral} ilustra a arquitetura proposta para a mineração de Bitcoin utilizando FPGA. No topo, o \textit{Node} representa a estrutura que registra todas as transações. O node interage diretamente com o Pool de Mineração, que organiza e distribui o trabalho de mineração entre diferentes participantes.


\begin{figure}[H]
    \centering
    \caption{Visão geral do sistema de mineração em FPGA com aceleração de hardware executando um software minerador com acesso ao núcleo SHA-256.}
    \fbox{
        \includegraphics[width=0.8\textwidth]{projeto/visao_geral2.png}
    }
    \caption*{Fonte: O autor (2024).}
    \label{fig:visao_geral}
\end{figure}

A parte inferior da figura é dividida entre o Hardware e o Software:

Hardware (à esquerda): Um FPGA configurado para executar o algoritmo SHA-256, responsável pelo cálculo dos hashes necessários para a validação de blocos. A lógica programável do FPGA inclui um PLL para gerar os sinais de \textit{clock} e uma unidade dedicada ao cálculo do SHA-256.

Software (à direita): Um sistema baseado em ARM, executando um sistema operacional Linux, que roda o software de mineração (\textit{fpga miner}). Esse software gerencia a comunicação com o FPGA e interage com o Pool de Mineração para receber e enviar os blocos validados.

As setas indicam a comunicação entre os componentes:

\begin{itemize}
    \item O Node Blockchain interage com o Pool de Mineração para distribuir tarefas de mineração e validar blocos.
    \item O Pool de Mineração fornece as tarefas ao FPGA via o software de mineração.
    \item O FPGA processa os cálculos do SHA-256 e retorna os resultados ao software para validação e envio ao pool.
\end{itemize}

Esse sistema explora a aceleração por hardware (FPGA) e a integração com um software de controle, buscando otimizar a eficiência e a desempenho na mineração de Bitcoin.




Hoje existem várias implementações de algoritmos de mineração de Bitcoin, incluindo para CPUs, GPUs, FPGAs e ASICs. Cada uma com seu programa e implementação diferentes, seja via software, seja via firmware. Este trabalho foca exclusivamente na implementação de mineração em FPGA, utilizando o SHA-256, devido ao objetivo inicial do trabalho. A arquitetura do sistema, incluindo as adaptações e configurações específicas para este projeto, está descrita na \autoref{sec:arq_software}.

As principais premissas e delimitadores de escopo para o projeto são:

\begin{itemize}
    \item O trabalho irá se restringir a avaliar o algoritmo SHA-256 em operações por segundo em diferentes tecnologias de circuito e processador;
    \item Não será proposto ou desenvolvido nenhum algoritmo novo de SHA-256 ou derivado do mesmo; 
    \item O SHA-256 usado é baseado em implementações já disponibilizados na literatura; 
    \item Os kits de prototipação que envolvem os testes serão os disponíveis em laboratórios de ensino e pesquisa ou adquiridos pelo aluno;
    \item As linguagens de desenvolvimento se resumirão a C/C++ e VHDL;
    \item Os dispositivos de processamento alvos serão de prateleira (do inglês Comercial Off-the-Shelf - COTS).
    
\end{itemize}

%%%%%%%%%%%%%%%%%%%%%%%%%%%%%%%%%%%%%%%%%%%%%%%%%%%%%%%%%%%%%%%%%%%%%%%%%%%%%%%%%%

\section{Análise de Requisitos}\label{sec:analise_requisitos}

Dadas as premissas apresentadas na \autoref{sec:premissas}, foram elaborados os requisitos do projeto para descrever as principais características e especificações da implementação do algoritmo de mineração em SHA-256 utilizando FPGAs e HLS. 

Nesta seção, são detalhados os requisitos funcionais e não funcionais, fornecendo uma base para a análise e implementação do sistema de mineração. Os requisitos funcionais e não funcionais são apresentados no \autoref{qua:Requisitos_funcionais} e no \autoref{qua:Requisitos_nao_funcionais}, respectivamente.

\begin{quadro}[H]
\caption{Requisitos funcionais do sistema de mineração em FPGA.}
\small
\renewcommand{\arraystretch}{1.15}
\begin{tabular}{|L{2.5cm}|L{12.6cm}|}
\rowcolor[HTML]{EFEFEF}  
\hline
\textbf{Requisito} & \textbf{Descrição} \\ \hline
\textbf{RF01} & O sistema deve utilizar o algoritmo SHA-256 para a mineração de Bitcoin, tendo como arquitetura de FPGA para acelerar o processamento.
\\ \hline
\textbf{RF02} & O sistema deve realizar operações de \textit{hashrate} para garantir a produtividade em uma \textit{pool} de mineração.
\\ \hline
\textbf{RF03} & O sistema deve permitir o monitoramento do consumo de energia e temperatura do FPGA durante a execução da mineração.
\\ \hline
\textbf{RF04} & O sistema deve exibir informações de desempenho em tempo real, como o \textit{hashrate} e o status da mineração.
\\ \hline
\end{tabular}
\label{qua:Requisitos_funcionais}
\caption*{Fonte: O autor (2024).}
\end{quadro}


\begin{quadro}[H]
\small
\caption{Requisitos não funcionais do sistema de mineração em FPGA.}
\renewcommand{\arraystretch}{1.15}
\begin{tabular}{|L{2.5cm}|L{12.6cm}|}
\rowcolor[HTML]{EFEFEF} 
\hline
\textbf{Requisito} & \textbf{Descrição} 
\\ \hline
\textbf{RNF01} & O sistema deve ser implementado em um FPGA compatível com HLS, para facilitar o desenvolvimento e a síntese do algoritmo de mineração.
\\ \hline

\textbf{RNF02} & O ambiente de desenvolvimento e testes deve incluir uma infraestrutura para monitoramento de consumo energético e temperatura.
\\ \hline

\textbf{RNF03} & O desenvolvimento será realizado em ambiente Linux, utilizando ferramentas de desenvolvimento compatíveis com FPGA.
\\ \hline

\textbf{RNF04} & O sistema deverá operar com um consumo de energia otimizado, mantendo a eficiência energética durante a mineração.
\\ \hline

\textbf{RNF05} & O projeto será testado e validado em uma bancada de hardware que permita medir o desempenho e eficiência energética em condições controladas.
\\ \hline

\end{tabular}
\label{qua:Requisitos_nao_funcionais}
\caption*{Fonte: O autor (2024).}
\end{quadro}


%%%%%%%%%%%%%%%%%%%%%%%%%%%%%%%%%%%%%%%%%%%%%%%%%%%%%%%%%%%%%%%%%%%%%%%%%%%%%%%%%%%%
\section{Arquitetura de Software}\label{sec:arq_software}

Nesta seção, são apresentados os detalhes de implementação do sistema de mineração de Bitcoin em FPGA utilizando o algoritmo SHA-256.
O diagrama apresentado na \autoref{fig:diagrama_sequencia} ilustra o fluxo de interação entre os componentes principais do sistema de mineração de Bitcoin utilizando FPGA e o algoritmo SHA-256. O processo é composto por diversas entidades que colaboram para garantir a validação das transações e a inclusão de blocos na blockchain. 

\begin{figure}[H]
    \centering
    \caption{Diagrama de Sequência do processo de Mineração}
    \includegraphics[width=1\linewidth]{projeto/diagrama_sequencia.png}
    \caption*{Fonte: O autor (2024).}
    \label{fig:diagrama_sequencia}
\end{figure}

\subsection{Node (Nó)}\label{subsec:node}

Um nó de Bitcoin é um participante da rede que valida e propaga transações e blocos, mantendo uma cópia completa ou parcial da blockchain. Os nós desempenham um papel essencial na segurança e descentralização da rede, ao verificarem as transações e blocos conforme o protocolo do Bitcoin. Cada nó independente reforça a resistência da rede a ataques, garantindo que o sistema opere de maneira segura e confiável \cite{nakamoto2008bitcoin}.

\subsection{Pool}\label{subsec:pool}

Um \textit{pool} de mineração é uma colaboração de mineradores que unem seus recursos computacionais para aumentar a probabilidade de encontrar blocos e receber recompensas em redes de criptomoedas que utilizam o algoritmo de consenso Proof of Work (PoW), como o Bitcoin. Ao trabalhar em conjunto, os participantes compartilham proporcionalmente as recompensas obtidas, conforme a contribuição de cada um para o poder de processamento total do \textit{pool}. Essa abordagem é especialmente vantajosa para mineradores individuais, que, sozinhos, teriam chances significativamente menores de sucesso devido à alta dificuldade de mineração presente nas principais criptomoedas \cite{coinext2024pool}.

\begin{quote}
    Então “pool”, além de significar “piscina” em inglês, significa “agrupamento”, “conjunto”; e o “pool de mineração” consiste em um conjunto de nós — ou “nodes” — de computadores funcionando coletivamente na mineração, aumentando o poder computacional e as chances de obter maior sucesso na resolução dos cálculos \cite{coinext2024pool}.
\end{quote}


\subsection{Softwares de Mineração}\label{subsec:cgminer}

\textbf{CGMiner}

O CGMiner é um software de mineração de criptomoedas de código aberto, desenvolvido principalmente para mineração de Bitcoin e baseado em C. Este software é amplamente utilizado pela comunidade de mineração devido à sua flexibilidade, robustez e suporte a diversos dispositivos de mineração, como CPUs, GPUs, FPGAs e ASICs. CGMiner permite aos usuários configurar e monitorar seus dispositivos de mineração, ajustando parâmetros importantes como a frequência de operação e a intensidade, além de suportar algoritmos de mineração como o SHA-256, essencial para a mineração de Bitcoin. Com uma interface de linha de comando e suporte para \textit{pools} de mineração, o CGMiner oferece uma ampla gama de funcionalidades para mineradores que buscam maximizar seu desempenho e eficiência energética, especialmente quando executado em conjunto com hardware dedicado, como FPGAs e ASICs \cite{cgminer}.

\textbf{Open-Source FPGA Bitcoin Miner}

O \textit{Open-Source FPGA Bitcoin Miner} é uma solução de código aberto projetada especificamente para mineração de Bitcoin utilizando FPGAs. Desenvolvido com foco em desempenho e flexibilidade, este software fornece uma base modular para implementações personalizadas, permitindo que engenheiros e entusiastas otimizem a eficiência de seus projetos de mineração em hardware. O minerador é altamente configurável, suportando diversas arquiteturas de FPGA, além de oferecer suporte ao algoritmo SHA-256, utilizado no processo de mineração.

Uma das principais vantagens do \textit{Open-Source FPGA Bitcoin Miner} é sua capacidade de personalização, permitindo ajustes específicos para maximizar a taxa de \textit{hash} (\textit{hashrate}) e minimizar o consumo energético. Este software é amplamente utilizado em projetos acadêmicos e industriais como referência para desenvolvimento de sistemas de mineração de criptomoedas baseados em FPGA \cite{OpenSourceFPGA2024}.


\subsection{Sistema de Mineração}\label{subsec:sistema}

O sistema de software é dividido nos seguintes módulos principais:

\begin{itemize}
    \item \textbf{Controlador de Mineração}: Implementado em C/C++, este módulo gerencia o processo de mineração, controlando o início, interrupção e verificação dos blocos processados.
    \item \textbf{Algoritmo SHA-256}: Implementado em VHDL para o FPGA, este módulo realiza o cálculo dos \textit{hashes} necessários para validar os blocos na blockchain.
    \item \textbf{Monitoramento de Desempenho}: Responsável por coletar e registrar dados de desempenho, como \textit{hashrate}, consumo de energia e temperatura do FPGA, para análise posterior.
    \item \textbf{Interface de Monitoramento}: Permite a visualização das informações de desempenho em tempo real e a configuração de parâmetros, utilizando comunicação serial para interação com o FPGA.
\end{itemize}

%%%%%%%%%%%%%%%%%%%%%%%%%%%%%%%%%%%%%%%%%%%%%%%%%%%%%%%%%%%%%%%%%%%%%%%%%%%%%%%%%%%%
\section{Arquitetura de Hardware}\label{sec:arq_hardware}

Nesta seção, são detalhados os aspectos do hardware utilizado para a implementação do sistema de mineração de Bitcoin em FPGA com o algoritmo SHA-256. A \autoref{sec:arq_hardware} apresenta o FPGA e discute suas características de reconfiguração e paralelismo que otimizam a taxa de \textit{hash}. A implementação do sistema é feita na ferramenta Vivado Design Suite, usando HLS para converter código em C/C++ para descrição em hardware, explorando o paralelismo e permitindo adaptações rápidas para aprimorar o desempenho e a eficiência energética do sistema.

\subsection{HLS como ferramenta}

O uso de High-Level Synthesis (HLS) é essencial para a implementação do algoritmo SHA-256 em FPGA, ao permitir que o desenvolvedor escreva o código em linguagens de alto nível, como C/C++, convertidas em descrição de hardware, como VHDL ou Verilog. Essa abordagem facilita a exploração de paralelismo e a otimização da arquitetura, reduzindo o tempo de desenvolvimento e simplificando a adaptação a novos requisitos de desempenho.

A abordagem do uso de HLS para aceleração de códigos em C/C++, permite o desenvolvimento eficiente de funções complexas para FPGAs, possibilitando melhorias significativas no desempenho da mineração. Este código é uma adaptação baseada na implementação padrão do algoritmo SHA-256 descrito no documento \textit{FIPS PUB 180-4 (Federal Information Processing Standards Publication 180-4)} pelo \textit{NIST} \cite{nist_sha256}. Além disso, estruturas de código SHA-256 semelhantes são comuns em repositórios e recursos de código aberto, como o \cite{bcon_crypto_algorithms} e sites de documentação de algoritmos criptográficos, onde são disponibilizadas implementações de referência para estudo e desenvolvimento.

Na Figura \ref{fig:diagrama_hls_rtl}, observa-se o fluxo de conversão de HLS para RTL (Register Transfer Level). A ferramenta HLS transforma o código em C/C++ em uma representação RTL, permitindo que a implementação do algoritmo seja sintetizada em hardware real no FPGA. Isso proporciona um desenvolvimento mais ágil e flexível para o projeto de descrição em hardware no FPGA.

\begin{figure}[H]
    \centering
    \caption{Diagrama do fluxo de síntese entre HLS e RTL.}
    \includegraphics[width=1\linewidth]{projeto/diagrama_Fluxo_hardware.png}

    \caption*{Fonte: O autor (2024).}
    \label{fig:diagrama_hls_rtl}
\end{figure}



\subsection{Diagrama de Hardware}\label{sec:hardware_diagram}

A \autoref{fig:diagrama_hardware} ilustra a arquitetura proposta para o sistema de mineração, destacando as principais etapas do fluxo de dados e os componentes de hardware utilizados. O sistema é composto por um FPGA com recursos configuráveis, como a PLL para geração de \textit{clock}, FIFO para gerenciamento de dados e controladores digitais de lógica que orquestram o fluxo de dados necessário para o processamento do algoritmo SHA-256. Além disso, o FPGA é conectado ao HPS (Hard Processor System) através de um barramento AXI, que permite a comunicação entre os módulos de hardware e o processador ARM.

No HPS, o software de mineração, incluindo o binário do Miner, é executado em um ambiente Linux, onde o processador ARM lida com tarefas de controle e gerenciamento do sistema. O HPS também fornece interfaces essenciais, como GPIO e \textit{Ethernet}, permitindo a integração do sistema com redes externas e dispositivos periféricos. Esse conjunto de componentes permite que o sistema combine o hardware dedicado do FPGA para a execução do algoritmo SHA-256, com a flexibilidade do software executado no HPS, garantindo assim um desempenho otimizado para a mineração de Bitcoin.

\begin{figure}[H]
    \caption{Diagrama dos detalhes lógicos do hardware de um FPGA e o núcleo SHA-256.}
    \centering
    \fbox{
        \parbox{1\textwidth}{
            \centering
            \includegraphics[width=0.9\textwidth]{projeto/diagrama_hardware.png}}}
        \caption*{Fonte: Adaptado de \cite{framework_digital_control_2020}.}
        \label{fig:diagrama_hardware}
\end{figure}


\subsection{Arquitetura do Altera Cyclone V}\label{sec:hardware_arq}
O FPGA utilizado neste projeto possui uma arquitetura com LUTs e Flip-Flops, essenciais para executar o algoritmo SHA-256 de maneira eficiente e paralela, maximizando o \textit{hashrate}. Blocos de memória e multiplicadores integrados também serão explorados para acelerar cálculos específicos e reduzir o consumo de energia. Sensores de temperatura e monitoramento de consumo serão integrados para garantir a operação segura e ajustes em tempo real, otimizando o desempenho e a viabilidade econômica do sistema \cite{vahid2001embedded}.

%%%%%%%%%%%%%%%%%%%%%%%%%%%%%%%%%%%%%%%%%%%%%%%%%%%%%%%%%%%%%%%%%%%%%%%%%%%%%%%%%%%%
\section{Materiais e Métodos}\label{sec:materiais_metodos}

Essa seção apresenta os materiais e métodos que serão utilizados na implementação do algoritmo de mineração que será executado dentro do FPGA, assim como as técnicas de aceleração em hardware.

\subsection{Materiais}

\textbf{Linguagem e Ferramentas}

Para as implementações, serão utilizadas as linguagens de programação C/C++ e para a geração do hardware será usado o fluxo de desenvolvimento de High-Level Synthesis (HLS) da Intel.

A principal ferramenta de desenvolvimento utilizada será a IDE Altera Quartus, que permite a síntese e a implementação de projetos para FPGA da família Intel. Essa IDE oferece recursos robustos para design, simulação e verificação, facilitando o processo de implementação de lógica digital em FPGAs e proporcionando uma análise aprofundada de desempenho e consumo de energia. Além disso, serão utilizadas as ferramentas de diagnóstico e monitoramento da Altera Quartus para avaliar a eficiência energética e o comportamento térmico do FPGA durante a mineração.

\textbf{Dispositivo Alvo}

O FPGA será da linha Altera Cyclone V, servirá como o dispositivo de processamento central, onde o algoritmo SHA-256 será implementado. Além do FPGA, um computador base será utilizado para realizar o desenvolvimento, síntese e configuração do algoritmo em C/C++ e VHDL, além de gerenciar o processo de monitoramento e coleta de dados em tempo real. 

\textbf{Monitoramento Térmico}

Para avaliar o desempenho térmico do FPGA durante a operação, será usada uma câmera térmica do tipo pistola, possibilitando o monitoramento visual da temperatura e garantindo que o sistema opere nos limites seguros, disponível na instituição.

\textbf{Wattímetro}

Também será utilizado um adaptador de tomada com medidor de consumo em watts, permitindo o acompanhamento em tempo real do consumo energético do sistema e a coleta de dados para análise de eficiência.

Esse conjunto de materiais permitirá uma análise completa do desempenho do FPGA, desde a implementação do algoritmo até a avaliação de sua eficiência energética e comportamento térmico.


\subsection{Métodos}

Para avaliar o desempenho e a viabilidade do sistema de mineração em FPGA, são adotados métodos que analisam diversos aspectos essenciais para uma operação eficiente e sustentável. Esses métodos visam medir a capacidade de processamento do FPGA em termos de taxa de \textit{hash}, além de monitorar o consumo energético e a eficiência da operação. Também será realizada a análise térmica para garantir que o dispositivo opere em condições seguras, evitando o superaquecimento e preservando a integridade do hardware. A seguir, cada um desses métodos é descrito em detalhes.

\textbf{Hashrate}

O \textit{hashrate} é uma métrica que mede a quantidade de operações de \textit{hash} realizadas por segundo, sendo fundamental para avaliar o desempenho da mineração de Bitcoin. Uma taxa de \textit{hash} elevada indica uma maior capacidade de processamento do FPGA, melhorando a probabilidade de sucesso na validação de blocos e, consequentemente, a eficiência do sistema de mineração.

\textbf{Consumo Energético}

O consumo energético monitora a quantidade de energia utilizada pelo FPGA durante a operação de mineração. Esse método permite avaliar o impacto do sistema na infraestrutura elétrica, além de auxiliar na análise da viabilidade econômica da mineração. Um adaptador de tomada com medidor de consumo em watts será utilizado para coletar esses dados em tempo real, permitindo a comparação entre diferentes configurações de hardware e software.

\textbf{Eficiência Energética}

A eficiência energética é uma métrica derivada da relação entre a taxa de \textit{hashrate} e o consumo energético, expressa em \textit{Hash/Watt}. Essa medida permite avaliar a eficácia do sistema em realizar operações de mineração com o menor consumo de energia possível. A eficiência energética é fundamental para determinar o custo-benefício da operação, ao indicar o quão bem o sistema converte energia elétrica em trabalho computacional útil.

\textbf{Temperatura de Operação}

A temperatura de operação do FPGA será monitorada durante todo o processo de mineração para garantir que o dispositivo funcione em limites térmicos seguros. Uma câmera térmica será utilizada para visualizar e registrar a temperatura do hardware, prevenindo possíveis danos causados por superaquecimento. Esse método é essencial para avaliar a estabilidade do sistema e a necessidade de soluções de resfriamento adicionais.

Esses métodos fornecem uma visão abrangente do desempenho do sistema, considerando tanto a eficácia na mineração quanto a eficiência energética e a estabilidade térmica, fundamentais para uma implementação sustentável e de alto desempenho.

%%%%%%%%%%%%%%%%%%%%%%%%%%%%%%%%%%%%%%%%%%%%%%%%%%%%%%%%%%%%%%%%%%%%%%%%%%%%%%%%%%%%
\section{Análise de Riscos} \label{sec:analise_riscos}

No \autoref{Q:Analise_Riscos}, são listados alguns dos riscos mais relevantes para o desenvolvimento do projeto, bem como estratégias de resposta para mitigá-los. Essas ações preventivas e reativas não apenas fortalecem a capacidade de o projeto enfrentar imprevistos, mas também contribuem para um gerenciamento mais técnico do trabalho, aumentando a probabilidade de conclusão bem-sucedida nos prazos estipulados.

\begin{quadro}[H]
\caption{Análise de riscos do projeto.}
\vspace{-6pt}
\renewcommand{\arraystretch}{1.10}
\begin{tabular}{|L{0.22\textwidth}|C{0.16\textwidth}|C{0.10\textwidth}|L{0.18\textwidth}|L{0.20\textwidth}|}
\rowcolor[HTML]{EFEFEF}
\hline

\textbf{Risco} & \textbf{Probabilidade} & \textbf{Impacto} & \textbf{Gatilho} & \textbf{Plano de Contingência} 
\\ \hline

Falha na implementação do algoritmo SHA-256 & Médio & Alto & Problemas no código ou inconsistências nos testes de unidade & Realizar revisão detalhada do código, implementar testes e consultar documentação técnica
\\ \hline

Problemas na conversão de código HLS & Médio & Alto & Erros durante a conversão do código ou desempenho abaixo do esperado & Utilizar ferramentas de diagnóstico e debug da IDE, otimizar o código C/C++, mudar a ferramenta de HLS
\\ \hline


Superaquecimento do FPGA & Médio & Alto & Temperatura excedendo limites seguros durante a operação & Implementar sistema de resfriamento adequado, ativo ou passivo, monitorar temperatura constantemente e ajustar frequência de operação
\\ \hline

Indisponibilidade de ferramentas de desenvolvimento & Baixa & Médio & Ferramentas de desenvolvimento ou ambiente de simulação fora de operação & Ter um ambiente de backup local e versões alternativas de ferramentas configuradas
\\ \hline

\end{tabular}
\label{Q:Analise_Riscos}
\caption*{Fonte: O autor (2024).}
\end{quadro}


%%%%%%%%%%%%%%%%%%%%%%%%%%%%%%%%%%%%%%%%%%%%%%%%%%%%%%%%%%%%%%%%%%%%%%%%%%%%%%%%%%%%

\section{Cronograma do TCC III} \label{sec:cronograma_tcc3}

A seção tem o objetivo de apresentar o \autoref{Q:Cronograma}, que ilustra o cronograma com as atividades a serem realizadas ao longo do TCC III (tempo de execução é dividido em semanas de trabalho).


\begin{quadro}[H]
\caption{Cronograma de atividades do TCC III.}
\renewcommand{\arraystretch}{1.10}
\begin{tabular}{|L{0.38\textwidth}|C{0.09\textwidth}|C{0.09\textwidth}|C{0.09\textwidth}|C{0.09\textwidth}|C{0.09\textwidth}|C{0.09\textwidth}|}
\rowcolor[HTML]{EFEFEF} 
\hline
\textbf{Atividades}& \textbf{02/2025}& \textbf{03/2025}& \textbf{04/2025}& \textbf{05/2025}& \textbf{06/2025}
\\ \hline

Implementação & XXXX & XXXX & XX - - & &
\\ \hline
Simulação & & XXXX & XXXX & &
\\ \hline
Verificação funcional & & XXXX & XXXX & XXXX & X - - -
\\ \hline
Benchmarking (hashrate) & & & & XXXX & X - - -
\\ \hline
Análise de Consumo & & & & - - - X & X - - -
\\ \hline 
Monografia & & XXXX & XXXX & XXXX & X - - -
\\ \hline 

\end{tabular}
\label{Q:Cronograma}
\caption*{Fonte: O autor (2024).}
\end{quadro} \clearpage
\chapter{CONSIDERAÇÕES FINAIS}\label{Sec:Conclusoes}

Neste trabalho, foi apresentado um estudo detalhado sobre a viabilidade da implementação do algoritmo SHA-256 em FPGA, utilizando técnicas de High-Level Synthesis (HLS) para mineração de Bitcoin. Foram explorados os aspectos teóricos relacionados à criptografia, às tecnologias de circuito e processadores, a aceleração em hardware, assim como a arquitetura de hardware e software proposta para o sistema.

O projeto desenvolvido até o momento fornece uma base sólida para o avanço para a próxima etapa, a ser realizada no TCC 3. No próximo semestre, será iniciada a implementação prática do sistema proposto em um FPGA da linha \textit{Digilent ZedBoard Zynq-7000 Development Board}, cedido pela UNIVALI. A programação será realizada utilizando a ferramenta \textit{Vivado Design Suite}, enquanto os testes de desempenho incluirão medições de \textit{hashrate} com base em logs, consumo energético mediante um Wattímetro Digital, consequentemente, o cálculo da eficiência energética e a preocupação térmica do dispositivo em funcionamento.

A fase de implementação também permitirá validar os conceitos apresentados neste trabalho, com a análise comparativa do desempenho do FPGA em relação a outras tecnologias de mineração, como CPUs e equipamentos baseados em ASICs. Adicionalmente, serão realizadas melhorias na arquitetura do sistema com base nos resultados obtidos, buscando otimizar o desempenho e a eficiência energética. \clearpage

\postextual
\bibliography{refs.bib}
%\begin{apendicesenv}
\appendix
\chapter{Arquivo sha256.h}\label{apendice:sha256_h}
\begin{lstlisting}[language=C]
#ifndef SHA256_H
#define SHA256_H

#include <stdbool.h>
#include <stdio.h>
#include <stdint.h>

// Tipos
#define uchar unsigned char // 8-bit byte
#define uint uint32_t 

// Macro para tratar adicao de inteiros de 64 bits
#define DBL_INT_ADD(a,b,c) if (a > 0xffffffff - (c)) ++b; a += c;

// Macros para rotacoes
#define ROTLEFT(a,b) ((uint32_t)(((uint64_t)(a) << (b)) | ((uint64_t)(a) >> (32 - (b)))))
#define ROTRIGHT(a,b) ((uint32_t)(((uint64_t)(a) >> (b)) | ((uint64_t)(a) << (32 - (b)))))


static inline uint32_t rotr32(uint32_t value, uint32_t shift) {
 shift &= 31;
 return ((value >> shift) | ((value << (32 - shift)) & 0xFFFFFFFF));
}

// Macros para funcoes logicas do SHA-256
#define CH(x,y,z) (((x) & (y)) ^ (~(x) & (z)))
#define MAJ(x,y,z) (((x) & (y)) ^ ((x) & (z)) ^ ((y) & (z)))
#define EP0(x) (ROTRIGHT(x,2) ^ ROTRIGHT(x,13) ^ ROTRIGHT(x,22))
#define EP1(x) (ROTRIGHT(x,6) ^ ROTRIGHT(x,11) ^ ROTRIGHT(x,25))
#define SIG0(x) (ROTRIGHT(x,7) ^ ROTRIGHT(x,18) ^ ((x) >> 3))
#define SIG1(x) (ROTRIGHT(x,17) ^ ROTRIGHT(x,19) ^ ((x) >> 10))

// Constantes do SHA-256
static const uint k[64] = {
    0x428a2f98, 0x71374491, 0xb5c0fbcf, 0xe9b5dba5, 0x3956c25b, 0x59f111f1, 0x923f82a4, 0xab1c5ed5,
    0xd807aa98, 0x12835b01, 0x243185be, 0x550c7dc3, 0x72be5d74, 0x80deb1fe, 0x9bdc06a7, 0xc19bf174,
    0xe49b69c1, 0xefbe4786, 0x0fc19dc6, 0x240ca1cc, 0x2de92c6f, 0x4a7484aa, 0x5cb0a9dc, 0x76f988da,
    0x983e5152, 0xa831c66d, 0xb00327c8, 0xbf597fc7, 0xc6e00bf3, 0xd5a79147, 0x06ca6351, 0x14292967,
    0x27b70a85, 0x2e1b2138, 0x4d2c6dfc, 0x53380d13, 0x650a7354, 0x766a0abb, 0x81c2c92e, 0x92722c85,
    0xa2bfe8a1, 0xa81a664b, 0xc24b8b70, 0xc76c51a3, 0xd192e819, 0xd6990624, 0xf40e3585, 0x106aa070,
    0x19a4c116, 0x1e376c08, 0x2748774c, 0x34b0bcb5, 0x391c0cb3, 0x4ed8aa4a, 0x5b9cca4f, 0x682e6ff3,
    0x748f82ee, 0x78a5636f, 0x84c87814, 0x8cc70208, 0x90befffa, 0xa4506ceb, 0xbef9a3f7, 0xc67178f2
};

void sha256_transform(uint state[8], const uchar data[64]);
void sha256_init(uint state[8]);
void sha256_update(uchar in_data[64], uint *datalen, uint bitlen[2], const uchar header[80], uint len, uint state[8]);
void sha256_final(uchar in_data[64], uint *datalen, uint bitlen[2], uint state[8], uchar final_hash[32]);

void sha256_top(
  uint32_t *header,          // 80 bytes = 20 palavras
  uint32_t *hash_result,     // 32 bytes = 8 palavras
  volatile bool *ap_busy_out // 'true' quando computando, 'false' quando ocioso
);
#endif // SHA256_H
\end{lstlisting}


%%%%%%%%%%%%%%%%%%%%%%%%%%%%%%%%%%%%%%%%%%%%%%%%%%%%%%%%%%%%%%%%%%%%%%%%%%%%%%%%%%%%%%%%
%%%%%%%%%%%%%%%%%%%%%%%%%%%%%%%%%%%%%%%%%%%%%%%%%%%%%%%%%%%%%%%%%%%%%%%%%%%%%%%%%%%%%%%%
%%%%%%%%%%%%%%%%%%%%%%%%%%%%%%%%%%%%%%%%%%%%%%%%%%%%%%%%%%%%%%%%%%%%%%%%%%%%%%%%%%%%%%%%

\chapter{Arquivo sha256.c} \label{apendice:sha256_c}

\begin{lstlisting}[language=C]
#include <stdio.h>
#include <stdbool.h>
#include "sha256.h"

/*Recebe um bloco de 64 bytes (512 bits) como data[]
Expande esse bloco para 64 palavras de 32 bits (m[64])
Aplica os 64 rounds do algoritmo SHA-256 usando as constantes k[]
Atualiza state[8] com o resultado final acumulado*/
void sha256_transform(uint state[8], const uchar data[64]) {
  uint a, b, c, d, e, f, g, h, i, j, t1, t2, m[64];
  #pragma HLS ARRAY_PARTITION variable=data complete dim=1
  #pragma HLS ARRAY_PARTITION variable=m complete dim=1

  for (i = 0, j = 0; i < 16; ++i, j += 4) {
    #pragma HLS unroll
    m[i] = (data[j] << 24) | (data[j+1] << 16) | (data[j+2] << 8) | (data[j+3]);
  }

  for (; i < 64; ++i)  {
    #pragma HLS PIPELINE II=1
    m[i] = SIG1(m[i-2]) + m[i-7] + SIG0(m[i-15]) + m[i-16];
  }

  a = state[0];
  b = state[1];
  c = state[2];
  d = state[3];
  e = state[4];
  f = state[5];
  g = state[6];
  h = state[7];

  for (i = 0; i < 64; ++i) {
    #pragma HLS pipeline II=1
    t1 = h + EP1(e) + CH(e, f, g) + k[i] + m[i];
    t2 = EP0(a) + MAJ(a, b, c);
    h = g;
    g = f;
    f = e;
    e = d + t1;
    d = c;
    c = b;
    b = a;
    a = t1 + t2;
  }

  state[0] += a;
  state[1] += b;
  state[2] += c;
  state[3] += d;
  state[4] += e;
  state[5] += f;
  state[6] += g;
  state[7] += h;
}

/*Inicializa o vetor state[8] com os valores padrao do SHA-256, conforme a especificacao (FIPS 180-4).
Isso prepara o estado inicial para o hashing.*/
void sha256_init(uint state[8]) {
  #pragma HLS inline
  state[0] = 0x6a09e667;
  state[1] = 0xbb67ae85;
  state[2] = 0x3c6ef372;
  state[3] = 0xa54ff53a;
  state[4] = 0x510e527f;
  state[5] = 0x9b05688c;
  state[6] = 0x1f83d9ab;
  state[7] = 0x5be0cd19;
}

/*Vetor de entrada (header[]) e seu tamanho (len)
Buffer interno in_data[64] (para armazenar dados parciais)
datalen acumulado no buffer atual
comprimento total (bitlen[2])
E state[8] do hash em progresso*/
void sha256_update(uchar in_data[64], uint *datalen, uint bitlen[2], const uchar header[], uint len, uint state[8]) {
  for (uint i = 0; i < len; ++i) {
    in_data[*datalen] = header[i];
    (*datalen)++;
    if (*datalen == 64) {
      sha256_transform(state, in_data);
      DBL_INT_ADD(bitlen[0], bitlen[1], 512);
      *datalen = 0;
    }
  }
  DBL_INT_ADD(bitlen[0], bitlen[1], (*datalen) * 8);
}




/*padding SHA-256 (bit 1, zeros ate completar 56 bytes)
codificacao do tamanho da mensagem (bitlen) nos ultimos 8 bytes
Se ainda tiver dados no buffer (in_data), processa com sha256_transform
Extrai os 8 valores de state[8] e transforma em final_hash[32] (Big Endian)*/
void sha256_final(uchar in_data[64], uint *datalen, uint bitlen[2], uint state[8], uchar final_hash[32]) {
  uint i, j;
  i = *datalen;  
  // Padding
  if (*datalen < 56) {
    in_data[i++] = 0x80;
    while (i < 56) in_data[i++] = 0x00;
  } else {
    in_data[i++] = 0x80;
    while (i < 64) in_data[i++] = 0x00;

    sha256_transform(state, in_data);

    // Zera e prepara novo bloco
    for (j = 0; j < 56; j++) in_data[j] = 0;
    i = 56;
  }
  in_data[63] = bitlen[0];
  in_data[62] = bitlen[0] >> 8;
  in_data[61] = bitlen[0] >> 16;
  in_data[60] = bitlen[0] >> 24;
  in_data[59] = bitlen[1];
  in_data[58] = bitlen[1] >> 8;
  in_data[57] = bitlen[1] >> 16;
  in_data[56] = bitlen[1] >> 24;

  sha256_transform(state, in_data);

  // Para converter de Big Endian (state[0]) para Little Endian (final_hash)
  sha256_final_final_loop:
  for (i = 0; i < 4; ++i) {
    #pragma HLS UNROLL
    final_hash[i]     = (state[0] >> (24 - i * 8)) & 0xff;
    final_hash[i + 4] = (state[1] >> (24 - i * 8)) & 0xff;
    final_hash[i + 8] = (state[2] >> (24 - i * 8)) & 0xff;
    final_hash[i +12] = (state[3] >> (24 - i * 8)) & 0xff;
    final_hash[i +16] = (state[4] >> (24 - i * 8)) & 0xff;
    final_hash[i +20] = (state[5] >> (24 - i * 8)) & 0xff;
    final_hash[i +24] = (state[6] >> (24 - i * 8)) & 0xff;
    final_hash[i +28] = (state[7] >> (24 - i * 8)) & 0xff;
  }
}
\end{lstlisting}


%%%%%%%%%%%%%%%%%%%%%%%%%%%%%%%%%%%%%%%%%%%%%%%%%%%%%%%%%%%%%%%%%%%%%%%%%%%%%%%%%%%%%%%%
%%%%%%%%%%%%%%%%%%%%%%%%%%%%%%%%%%%%%%%%%%%%%%%%%%%%%%%%%%%%%%%%%%%%%%%%%%%%%%%%%%%%%%%%
%%%%%%%%%%%%%%%%%%%%%%%%%%%%%%%%%%%%%%%%%%%%%%%%%%%%%%%%%%%%%%%%%%%%%%%%%%%%%%%%%%%%%%%%

\chapter{Arquivo sha256\_top.c}\label{apendice:sha256_top_c}
\begin{lstlisting}[language=C]
#include "sha256.h"


void sha256_top(
  uint32_t header[20],       // 80 bytes = 20 palavras
  uint32_t hash_result[8],   // 32 bytes = 8 palavras
  volatile bool *ap_busy_out // 'true' quando computando, 'false' quando ocioso
){
  #pragma HLS INTERFACE s_axilite port=header      bundle=control offset=0x100
  #pragma HLS INTERFACE s_axilite port=hash_result bundle=control offset=0x200
  #pragma HLS INTERFACE s_axilite port=return      bundle=control

  #pragma HLS INTERFACE ap_none    port=ap_busy_out

  unsigned char hash1[32];
  unsigned char hash2[32];
  unsigned char header_bytes[80];

  #pragma HLS ARRAY_PARTITION variable=hash1        complete dim=1
  #pragma HLS ARRAY_PARTITION variable=hash2        complete dim=1
  #pragma HLS ARRAY_PARTITION variable=header_bytes complete dim=1
  
  #pragma HLS DEPENDENCE      variable=header       inter false
  #pragma HLS DEPENDENCE      variable=hash_result  inter false

    *ap_busy_out = true;  // Ativa o sinal de ocupado assim que a funcao comeca a ser executada no hardware.

  // Converte header (uint32_t[20]) para bytes
  for (int i = 0; i < 20; i++) {
    #pragma HLS UNROLL
    header_bytes[i*4 + 0] = (header[i] >>  0) & 0xFF;
    header_bytes[i*4 + 1] = (header[i] >>  8) & 0xFF;
    header_bytes[i*4 + 2] = (header[i] >> 16) & 0xFF;
    header_bytes[i*4 + 3] = (header[i] >> 24) & 0xFF;
  }

  // Zera temporarios
  for (int i = 0; i < 32; i++) hash1[i] = 0;
  for (int i = 0; i < 32; i++) hash2[i] = 0;
  
  uchar in_data[64] = {0};
  uint datalen = 0;
  uint bitlen[2] = {0};
  uint state[8] = {0};

  // SHA256 ROUND 1
  sha256_init(state);
  sha256_update(in_data, &datalen, bitlen, header_bytes, 80, state);
  sha256_final(in_data, &datalen, bitlen, state, hash1);

  for (int i = 0; i < 64; i++) in_data[i] = 0;
  datalen = 0;
  bitlen[0] = 0;
  bitlen[1] = 0;

  // SHA256 ROUND 2
  sha256_init(state);
  sha256_update(in_data, &datalen, bitlen, hash1, 32, state);
  sha256_final(in_data, &datalen, bitlen, state, hash2);


  // Converte hash2 (bytes) para hash_result (uint32_t[8])
  for (int i = 0; i < 8; i++) {
    #pragma HLS UNROLL
    hash_result[i] = ((uint32_t)hash2[i*4 + 3] << 24) |
                     ((uint32_t)hash2[i*4 + 2] << 16) |
                     ((uint32_t)hash2[i*4 + 1] << 8)  |
                     ((uint32_t)hash2[i*4 + 0]);              
  }

  *ap_busy_out = false;  // Desativa o sinal de ocupado quando a computacao principal termina.


}
\end{lstlisting}

%%%%%%%%%%%%%%%%%%%%%%%%%%%%%%%%%%%%%%%%%%%%%%%%%%%%%%%%%%%%%%%%%%%%%%%%%%%%%%%%%%%%%%%%
%%%%%%%%%%%%%%%%%%%%%%%%%%%%%%%%%%%%%%%%%%%%%%%%%%%%%%%%%%%%%%%%%%%%%%%%%%%%%%%%%%%%%%%%
%%%%%%%%%%%%%%%%%%%%%%%%%%%%%%%%%%%%%%%%%%%%%%%%%%%%%%%%%%%%%%%%%%%%%%%%%%%%%%%%%%%%%%%%
\chapter{Arquivo cycle\_counter.vhdl}\label{apendice:cycle_counter}

\begin{lstlisting}[language=vhdl]
library IEEE;
use IEEE.STD_LOGIC_1164.ALL;
use IEEE.NUMERIC_STD.ALL;

entity cycle_counter is
  Port (
    clk           : in  std_logic;
    reset         : in  std_logic;
    ap_busy_in    : in  std_logic;
    cycle_count   : out std_logic_vector(31 downto 0)
  );
end cycle_counter;

architecture Behavioral of cycle_counter is
  signal counter    : unsigned(31 downto 0) := (others => '0');
  signal reg_out    : unsigned(31 downto 0) := (others => '0');
  signal prev_ap_busy : std_logic := '0'; -- Para detectar transicao de 'ocioso' a 'ocupado'
begin
  process(clk)
  begin
    if rising_edge(clk) then
      if reset = '0' then
        counter   <= (others => '0');
        reg_out   <= (others => '0');
        prev_ap_busy <= '0';
      else
        if prev_ap_busy = '0' and ap_busy_in = '1' then
          counter <= (others => '0'); -- Reseta o contador ao iniciar a computacao
          -- prev_ap_busy = '0' e ap_busy_in = '1' detecta a borda de subida.
        elsif ap_busy_in = '1' then -- Enquanto o IP esta ocupado (ap_busy_in e '1')
          counter <= counter + 1;
        elsif prev_ap_busy = '1' and ap_busy_in = '0' then
          -- Detecta o fim da execucao (ap_busy_in vai de '1' para '0')
          reg_out <= counter; -- valor final quando a computacao termina
        end if;
        prev_ap_busy <= ap_busy_in; -- Atualiza ap_busy_in para a proxima deteccao de borda
      end if;
    end if;
  end process;
  cycle_count <= std_logic_vector(reg_out);
end Behavioral;
\end{lstlisting}

%%%%%%%%%%%%%%%%%%%%%%%%%%%%%%%%%%%%%%%%%%%%%%%%%%%%%%%%%%%%%%%%%%%%%%%%%%%%%%%%%%%%%%%%
%%%%%%%%%%%%%%%%%%%%%%%%%%%%%%%%%%%%%%%%%%%%%%%%%%%%%%%%%%%%%%%%%%%%%%%%%%%%%%%%%%%%%%%%
%%%%%%%%%%%%%%%%%%%%%%%%%%%%%%%%%%%%%%%%%%%%%%%%%%%%%%%%%%%%%%%%%%%%%%%%%%%%%%%%%%%%%%%%

\chapter{Arquivo main.c}\label{apendice:main_c}

\begin{lstlisting}[language=C]
#include "xil_io.h"
#include "xparameters.h"
#include "xil_printf.h"
#include "xtime_l.h"

#include <stdio.h> // para getchar(), fgets
#include <string.h>
#include "sha256.h"

// Base address do IP sha256_top_0 (mapeado via Address Editor)
#define SHA256_BASE       0xA0000000            // s_axi_control
#define GPIO_BASE         0xA0010000            // Enderecos para o AXI GPIO

#define CYCLE_COUNT_READ  (GPIO_BASE + 0x0)     // Data register para leitura do contador
#define REG_CONTROL       (SHA256_BASE + 0x00)  // Controle (ap_start, ap_done, ap_idle, etc.)
#define REG_HEADER_BASE   (SHA256_BASE + 0x100) // Offset inicial para os 20 registros de entrada (uint32_t[20])
#define REG_RESULT_BASE   (SHA256_BASE + 0x200) // Offset inicial para os 8 registros de saida (uint32_t[8])
#define CLOCK_HZ	      250000000ULL          // Para clock em 250 MHz no PL


// Variaveis globais
uint32_t header[20];       // 80 bytes
uint32_t hash_result[8];   // 32 bytes
u64 nanos_cpu, nanos_fpga;
u64 hashrate_cpu, hashrate_fpga;
int match_cpu = 0, match_fpga = 0;
unsigned char hash_cpu[32], hash_fpga[32];
unsigned char header_bytes[80];
unsigned char hash_expected[32];
char hash_string_cpu[65];
char hash_string_fpga[65];


unsigned char hex_to_byte(char a, char b) {
  unsigned char byte = 0;
  if (a >= '0' && a <= '9') byte |= (a - '0') << 4;
  else if (a >= 'a' && a <= 'f') byte |= (a - 'a' + 10) << 4;
  else if (a >= 'A' && a <= 'F') byte |= (a - 'A' + 10) << 4;

  if (b >= '0' && b <= '9') byte |= (b - '0');
  else if (b >= 'a' && b <= 'f') byte |= (b - 'a' + 10);
  else if (b >= 'A' && b <= 'F') byte |= (b - 'A' + 10);

  return byte;
}

void hex_string_to_bytes(const char *hex_str, unsigned char *bytes, int len) {
  for (int i = 0; i < len; i++) {
    bytes[i] = hex_to_byte(hex_str[i * 2], hex_str[i * 2 + 1]);
  }
}

void PS_Mining(const char *header_hex, const char *hash_real){
  hex_string_to_bytes(header_hex, header_bytes, 80);
  hex_string_to_bytes(hash_real, hash_expected, 32);

  for (int i = 0; i < 20; i++) {
    header[i] = ((uint32_t)header_bytes[i*4 + 0] << 0) |
                ((uint32_t)header_bytes[i*4 + 1] << 8) |
                ((uint32_t)header_bytes[i*4 + 2] << 16) |
                ((uint32_t)header_bytes[i*4 + 3] << 24);
  }

  XTime t1, t2;
  XTime_GetTime(&t1);
  sha256_top(header, hash_result);
  XTime_GetTime(&t2);

  nanos_cpu = (t2 - t1) * 1000000000ULL / COUNTS_PER_SECOND;
  hashrate_cpu = 1000000000ULL / nanos_cpu;

  for (int i = 0; i < 8; i++) {
    hash_cpu[i*4 + 0] = (hash_result[i] >> 0) & 0xFF;
    hash_cpu[i*4 + 1] = (hash_result[i] >> 8) & 0xFF;
    hash_cpu[i*4 + 2] = (hash_result[i] >> 16) & 0xFF;
    hash_cpu[i*4 + 3] = (hash_result[i] >> 24) & 0xFF;
  }

  match_cpu = 1;
  for (int i = 0; i < 32; i++) {
    if (hash_cpu[i] != hash_expected[i]) {
      match_cpu = 0;
      break;
    }
  }

  for (int i = 0; i < 32; i++) sprintf(&hash_string_cpu[i*2], "%02X", hash_cpu[i]);
}


void IP_Mining(const char *header_hex, const char *hash_real) {
  hex_string_to_bytes(header_hex, header_bytes, 80);
  hex_string_to_bytes(hash_real, hash_expected, 32);

  for (int i = 0; i < 20; i++) {
    header[i] = ((uint32_t)header_bytes[i*4 + 0] << 0) |
                ((uint32_t)header_bytes[i*4 + 1] << 8) |
                ((uint32_t)header_bytes[i*4 + 2] << 16) |
                ((uint32_t)header_bytes[i*4 + 3] << 24);
  }

  for (int i = 0; i < 20; i++) {
    Xil_Out32(REG_HEADER_BASE + i * 4, header[i]);
  }

  Xil_Out32(REG_CONTROL, 0x01);  				// start IP
  while ((Xil_In32(REG_CONTROL) & 0x2) == 0);   // espera ap_done

  for (int i = 0; i < 8; i++) {
    hash_result[i] = Xil_In32(REG_RESULT_BASE + i * 4);
  }

  for (int i = 0; i < 8; i++) {
    hash_fpga[i*4 + 0] = (hash_result[i] >> 0) & 0xFF;
    hash_fpga[i*4 + 1] = (hash_result[i] >> 8) & 0xFF;
    hash_fpga[i*4 + 2] = (hash_result[i] >> 16) & 0xFF;
    hash_fpga[i*4 + 3] = (hash_result[i] >> 24) & 0xFF;
  }

  match_fpga = 1;
  for (int i = 0; i < 32; i++) {
    if (hash_fpga[i] != hash_expected[i]) {
      match_fpga = 0;
      break;
    }
  }

  u32 cycles = Xil_In32(CYCLE_COUNT_READ);
  nanos_fpga = ((u64)cycles * 1000000000ULL) / CLOCK_HZ;
  hashrate_fpga = (nanos_fpga > 0) ? 1000000000ULL / nanos_fpga : 0;

  for (int i = 0; i < 32; i++) sprintf(&hash_string_fpga[i*2], "%02X", hash_fpga[i]);
}

void resultados(const char *hash_real) {
  char hash_cpu_be[65];  // string para hash em BE
  char hash_fpga_be[65]; // string para hash em BE

  // Inverte hash_cpu (LE -> BE)
  for (int i = 0; i < 32; i++) {
    sprintf(&hash_cpu_be[i * 2], "%02X", hash_cpu[31 - i]);
  }
  hash_cpu_be[64] = '\0'; // null terminator

  // Inverte hash_fpga (LE -> BE)
  for (int i = 0; i < 32; i++) {
    sprintf(&hash_fpga_be[i * 2], "%02X", hash_fpga[31 - i]);
  }
  hash_fpga_be[64] = '\0'; // null terminator


  xil_printf("\r\nIniciando Double SHA256 no CPU (Cortex-A53 arch:ARMv8-A)...");
  xil_printf("\r\nDouble Hash gerado (em LE): %s", hash_string_cpu);

  xil_printf("\r\n\nIniciando Double SHA256 no FPGA (Zynq UltraScale+ MPSoC ZCU104)...");
  xil_printf("\r\nDouble Hash gerado (em LE): %s", hash_string_fpga);

  xil_printf("\r\n\n--- COMPARATIVO ---");
  xil_printf("\r\nHash Final do bloco: %s", hash_real);
  xil_printf("\r\n[CPU]  corresponde ao esperado? %s", match_cpu ? "SIM" : "NAO");
  xil_printf("\r\n[FPGA] corresponde ao esperado? %s", match_fpga ? "SIM" : "NAO");

  xil_printf("\r\n\n--------- RESULTADOS: ---------");
  xil_printf("\r\n[CPU] Double Hash gerado (em BE): %s", hash_cpu_be);
  xil_printf("\r\n[CPU] Tempo de execucao (preciso): %llu.%09llu segundos (%llu ns)",
    nanos_cpu / 1000000000ULL, nanos_cpu % 1000000000ULL, nanos_cpu);
  xil_printf("\r\n[CPU] Hashrate (estimado) = (numero de hashes / tempo total): %llu H/s", hashrate_cpu);

  xil_printf("\r\n\n[FPGA] Double Hash gerado (em BE): %s", hash_fpga_be);
  xil_printf("\r\n[FPGA] Tempo de execucao (preciso): %llu.%09llu segundos (%llu ns)",
    nanos_fpga / 1000000000ULL, nanos_fpga % 1000000000ULL, nanos_fpga);
  xil_printf("\r\n[FPGA] Hashrate (estimado) = (numero de hashes / tempo total): %llu H/s", hashrate_fpga);

  xil_printf("\r\n\nGanho de aceleracao comparado (FPGA/CPU) ~ %llux\r\n", nanos_cpu / nanos_fpga);
}

int main() {
  char resposta;
  char header_input[161];
  char hash_input[65];
  const char *header_hex =
    "0100000050120119172a610421a6c3011dd330d9df07b63616c2cc1f1cd00200"
    "000000006657a9252aacd5c0b2940996ecff952228c3067cc38d4885efb5a4ac"
    "4247e9f337221b4d4c86041b0f2b5710";

  const char *hash_real =
    "06e533fd1ada86391f3f6c343204b0d278d4aaec1c0b20aa27ba030000000000";

  xil_printf("\r\n--------- TESTE DOUBLE SHA256 ---------");
  xil_printf("\r\n\n--- ENTRADA DE DADOS ---");
  xil_printf("\r\nHeader (hex) com nonce:\r\n%s\r\n", header_hex);

  // Execucao
  PS_Mining(header_hex, hash_real);
  IP_Mining(header_hex, hash_real);
  resultados(hash_real);

  // Pergunta se deseja novo teste
	xil_printf("\r\nDeseja testar outro bloco ao vivo? (s/n): ");
	while ((resposta = getchar()) == '\n'); // ignora ENTER

  if (resposta == 's' || resposta == 'S') {
	xil_printf("\r\n--- ENTRADA DE DADOS ---");
    xil_printf("\r\nInsira o header (hex) com nonce [160 caracteres]:\r\n");
    scanf("%160s", header_input);
    xil_printf("\r\nHeader (hex) com nonce:\r\n%s\r\n", header_input);
    xil_printf("\r\nInsira o hash esperado (LE, 64 caracteres):\r\n");
    scanf("%64s", hash_input);

    PS_Mining(header_input, hash_input);
    IP_Mining(header_input, hash_input);
    resultados(hash_input);
  }
  return 0;
}
\end{lstlisting}



%%%%%%%%%%%%%%%%%%%%%%%%%%%%%%%%%%%%%%%%%%%%%%%%%%%%%%%%%%%%%%%%%%%%%%%%%%%%%%%%%%%%%%%%
%%%%%%%%%%%%%%%%%%%%%%%%%%%%%%%%%%%%%%%%%%%%%%%%%%%%%%%%%%%%%%%%%%%%%%%%%%%%%%%%%%%%%%%%
%%%%%%%%%%%%%%%%%%%%%%%%%%%%%%%%%%%%%%%%%%%%%%%%%%%%%%%%%%%%%%%%%%%%%%%%%%%%%%%%%%%%%%%%

\chapter{Script Python mempool.py} \label{apendice:mempool_py}

\begin{lstlisting}[language=Python]
import requests
import hashlib
import struct
import hashlib

global height
height = 100000

def sha256(data): # Realiza o SHA256 localmente para comparacao
  return hashlib.sha256(data).digest()

def bits_to_target(bits):
  exponent = bits >> 24
  mantissa = bits & 0xffffff
  target = mantissa * (1 << (8 * (exponent - 3)))
  # Altere para little-endian para compatibilidade direta com C
  return target.to_bytes(32, byteorder='little')

def get_block_info(height):
  r = requests.get(f"https://mempool.space/api/block-height/{height}")
  block_hash = r.text.strip()
  r = requests.get(f"https://mempool.space/api/block/{block_hash}")
  b = r.json()

  return {
    "version": struct.pack("<L", b["version"]),
    "prev_hash": bytes.fromhex(b["previousblockhash"])[::-1],
    "merkle_root": bytes.fromhex(b["merkle_root"])[::-1],
    "timestamp": struct.pack("<L", b["timestamp"]),
    "bits": struct.pack("<L", b["bits"]),
    "nonce": struct.pack("<L", b["nonce"]),
    "bits_int": b["bits"],
    "nonce_int": b["nonce"],
    "block_hash": block_hash
  }

def mempool():
  global height
  b = get_block_info(height)

  version     = b["version"]
  prev_hash   = b["prev_hash"]
  merkle_root = b["merkle_root"]
  timestamp   = b["timestamp"]
  bits        = b["bits"]
  nonce       = b["nonce"]

  header = version + prev_hash + merkle_root + timestamp + bits + nonce

  print("Version: ", version.hex())
  print("Prev Block: ", prev_hash.hex())
  print("Merkle Root: ", merkle_root.hex())
  print("Timestamp: ", timestamp.hex())
  print("Bits: ", bits.hex())
  print("Nonce: ", nonce.hex())
  print("Header = version + prev_hash + merkle_root + timestamp + bits + nonce")

  print("----- BLOCK HEADER (80 bytes, LE, hex) -----")
  print(header.hex())

  # Double SHA256 para verificar o hash final, utilizando a lib  hashlib do Python
  first_hash_bytes = sha256(header)
  final_hash_bytes = sha256(first_hash_bytes)

  print("\n----- VERIFICACAO DO HASH FINAL (DO PYTHON) -----")
  print(f"Bloco numero: {height}")
  print(f"Double Hash Final (LE, Hexa): \t\t{final_hash_bytes.hex()}") # do python
  print(f"Double Hash Final (BE, Hexa invertido): {final_hash_bytes[::-1].hex()}") # do python
  print(f"Hash do Bloco (mempool.space): \t\t{b['block_hash']}") # do mempool.space

if __name__ == "__main__":
  mempool()
\end{lstlisting}



\end{apendicesenv} \clearpage
%\input{anexo/0_anexo.tex} \clearpage

\end{document}
